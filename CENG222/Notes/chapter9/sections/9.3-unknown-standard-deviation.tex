\section{Unknown Standard Deviation}
\label{sec:unknown-standard-deviation}

A rather heavy condition was assumed when we constructed all the confidence intervals. We assumed a \textit{known standard deviation} $\sigma$ and used it in all the derived formulas.

Sometimes this assumption is perfectly valid and we may know the variance. However, generally, the population variance is \textit{unknown}. We'll then estimate it from data and see if we can still apply methods of the previous section.

Two broad situations will be considered:
\begin{itemize}
  \item large samples from any distribution,
  \item samples of any size from a Normal distribution.
\end{itemize}

\subsection{Large Samples}
\label{subsec:large-samples}

A large sample should produce a rather accurate estimator of a variance. We can then
replace the true standard error $\sigma(\hat{\theta})$ in (3) by its estimator $s(\hat{\theta})$, and obtain an approximate confidence interval
\begin{equation*}
  \hat{\theta}\ \pm\ z_{\alpha/2} \cdot s(\hat{\theta})
\end{equation*}

\begin{example}{ (Delays at nodes)}
  Internet connections are often slowed by delays at nodes. Let us determine if the delay time increases during heavy-volume times.
  
  Five hundred packets are sent through the same network between 5 pm and 6 pm (sample $\bs{X}$), and three hundred packets are sent between 10 pm and 11 pm (sample $\bs{Y}$). The early sample has a mean delay time of 0.8 sec with a standard deviation of 0.1 sec whereas the second sample has a mean delay time of 0.5 sec with a standard deviation of 0.08 sec. Construct a $99.5\%$ confidence interval for the difference between the mean delay times. \\

  \textbf{Solution:}
  We have $n = 500$, $\bar{X}$, $s_X = 0.1$; $m = 500$, $\bar{Y}$, $s_Y = 0.1$. Large sample sizes allow us to replace unknow population standard deviationsa by their estimates and use an approximately Normal distribution of sample means.

  For a confidence level of $1 - \alpha = 0.995$, we need
  \begin{equation*}
    z_{\alpha/2} = z_{0.0025} = q_{0.9975}
  \end{equation*}
  Look for the \textit{probability} 0.9975 in the body of Table A4 (from textbook) and find the corresponding value of $z$,
  \begin{equation*}
    z_{0.0025} = 2.81
  \end{equation*}
  Then, a $99.5\%$ confidence interval for the difference of mean execution times is
  \begin{align*}
    \bar{X} - \bar{Y} \ &\pm\ z_{0.0025} \sqrt{\frac{s_X^2}{n} + \frac{s_Y^2}{m}}\\
    &= (0.8 - 0.5) \ \pm\ (2.81) \sqrt{\frac{(0.1)^2}{500} + \frac{(0.08)^2}{300}}\\
    &= 0.3 \ \pm\  \textnormal{ or } \left[ 0.282,\ 0.318 \right]
  \end{align*}
\end{example}


\subsection{Confidence Intervals for Proportions}
\label{subsec:conf-inter-for-proportions}

In particular, we surely don't know the variance when we estimate a population proportion.
\begin{definition}{}
  We assume a subpopulation $A$ of items that have a certain \textit{attribute}. By the \textbf{population proportion} we mean the probability
  \begin{equation*}
    p = \prob{i \in A}
  \end{equation*}
  for a randomly selected item i to have this attribute.

  A \textbf{sample proportion}
  \begin{equation*}
    \hat{p} = \frac{\textnormal{number of sampled items from A}}{n}
  \end{equation*}
  is used to estimate $p$.
\end{definition}
\noindent Let us use the \textit{indicator} variables
\begin{equation*}
  X_i = \begin{cases}
    1 &\textnormal{if } i \in A\\
    0 &\textnormal{if } i \notin A\\
  \end{cases}
\end{equation*}
Each $X_i$ has Bernoulli distribution with parameter $p$. In particular
\begin{align*}
  \expc{X_i} = p &&\textnormal{and}& &\var{X_i} = p (1 - p)
\end{align*}
Also,
\begin{equation*}
  \hat{p} = \frac{1}{n} \sum_{i=1}^{n} X_i
\end{equation*}
is nothing but a sample of $X_i$.

\noindent Therefore,
\begin{align*}
  \expc{\hat{p}} = p &&\textnormal{and}& &\var{\hat{p}} = \frac{p (1 - p)}{n}
\end{align*}

\vspace*{\fill}
\columnbreak

\noindent We conclude that
\begin{enumerate}
  \item a sample proportion $\hat{p}$ is unbiased for the population proportion $p$;
  \item it has approximately Normal distribution for large samples, because it has a form of
  a sample mean;
  \item when we construct a confidence interval for $p$, we do not know the standard deviation
  Std($\hat{p}$).
\end{enumerate}

Indeed, knowing the standard deviation is equivalent to knowing p, and if we know p, why
would we need a confidence interval for it?

Thus, we estimate the unknown standard error
\begin{equation*}
  \sigma(\hat{p}) = \sqrt{\frac{p (1-p)}{n}}
\end{equation*}
by
\begin{equation*}
  s(\hat{p}) = \sqrt{\frac{\hat{p} (1 - \hat{p})}{n}}
\end{equation*}
and use it in the general formula
\begin{equation*}
  \hat{p}\ \pm\ z_{\alpha/2} \cdot s(\hat{p})
\end{equation*}
to construct an approximate $(1 - \alpha)100\%$ confidence interval.
\begin{formula}{Confidence interval for a population proportion}
  \begin{equation*}
    \hat{p} \ \pm\ z_{\alpha/2} \sqrt{\frac{\hat{p} (1 - \hat{p})}{n}}
  \end{equation*}
\end{formula}

Similarly, we can construct a confidence interval for the difference between two proportions. In two populations, we have proportions $p_1$ and $p_2$ of items with an attribute. Independent samples of size $n_1$ and $n_2$ are collected, and both parameters are estimated by sample proportions $\hat{p_1}$ and $\hat{p_2}$.

Summarizing, we have
\begin{align*}
  \textnormal{Parameter of interest:} &\theta = p_1 - p_2 \\
  \textnormal{Estimated by:}          &\hat{\theta} = \hat{p_1} - \hat{p_2} \\
  \textnormal{Its standard error:}    &\sigma(\hat{\theta}) = \sqrt{\frac{p_1 (1 - p_1)}{n_1} + \frac{p_2 (1- p_2)}{n_2}} \\
  \textnormal{Estimated by:}          &s(\hat{\theta}) = \sqrt{\frac{\hat{p_1} (1 - \hat{p_1})}{n_1} + \frac{\hat{p_2} (1- \hat{p_2})}{n_2}}
\end{align*}

\begin{formula}{Confidence interval for the difference of proportions}
  \begin{equation*}
    \hat{p_1} - \hat{p_2} \ \pm \ z_{\alpha/2} \sqrt{\frac{\hat{p_1} (1 - \hat{p_1})}{n_1} + \frac{\hat{p_2} (1 - \hat{p_2})}{n_2}}
  \end{equation*}
\end{formula}

\begin{example}{ (Pre-election poll)}
  A candidate prepares for the local elections. During his campaign, 42 out of 70 randomly selected people in town A and 59 out of 100 randomly selected people in town B showed they would vote for this candidate. Estimate the difference in support that this candidate is getting in towns A and B with $95\%$ confidence. Can we state affirmatively that the candidate gets a stronger support in town A? \\

  \textbf{Solution:}
  We have $n_1 = 70$, $n_2 = 100$, $\hat{p_1} = 42/70 = 0.6$, and $\hat{p_2} = 59/100 = 0.59$. For the confidence interval, we have
  \begin{equation*}
    \textnormal{center} = \hat{p_1} - \hat{p_2} = 0.01
  \end{equation*}
  and
  \begin{align*}
    \textnormal{margin} &= z_{0.05/2} \sqrt{\frac{\hat{p_1} (1 - \hat{p_1})}{n_1} + \frac{\hat{p_2} (1 - \hat{p_2})}{n_2}}\\
    &= (1.960) \sqrt{\frac{(0.6) (0.4)}{70} + \frac{(0.59) (0.41)}{100}} = 0.15
  \end{align*}
  Then
  \begin{equation*}
    0.01 \ \pm \ 0.15 = \left[ -0.14,\ 0.16 \right]
  \end{equation*}
  is a $95\%$ confidence interval for the difference in support ($p_1 - p_2$) in the two towns.\\

  So, is the support stronger in town A? On one hand, the estimator $\hat{p_1} - \hat{p_2} = 0.01$ suggests that the support is $1\%$ higher in town A than in town B. On the other hand, the difference could appear positive just because of a sampling error. As we see, the $95\%$ confidence interval includes a large range of negative values too. Therefore, the obtained data \textit{does not} indicate affirmatively that the support in town A is stronger.
\end{example}


\subsection{Estimating Proportions with a Given Precision}
\label{subsec:est-propor-given-precision}

Our confidence interval for a population proportion has a margin
\begin{equation*}
  \textnormal{margin} = z_{\alpha/2} \sqrt{\frac{\hat{p} (1 - \hat{p})}{n}}
\end{equation*}
A standard way of finding the sample size that provides the desired margin $\Delta$ is to solve the inequality
\begin{align*}
  \textnormal{margin} \leq \Delta &&\textnormal{or}&& n \geq \hat{p} (1 - \hat{p}) \left( \frac{z_{\alpha/2}}{\Delta} \right)^2
\end{align*}
However, this inequality includes $\hat{p}$. To know $\hat{p}$, we first need to collect a sample, but to know the sample size, we first need to know $\hat{p}$!

Since $\hat{p} (1 - \hat{p})$ does not exceed 0.25, we can replace this unknown by 0.25 and find a sample size $n$, perhaps larger than we actually need, that will ensure that we estimate $\hat{p}$ with a margin not exceeding $\Delta$. That is, choose a sample size
\begin{equation*}
  n \geq 0.25 \left( \frac{z_{\alpha / 2}}{\Delta} \right)^2
\end{equation*}

\begin{example}{}
  A sample of size
  \begin{equation*}
    n \geq 0.25 \left( \frac{1.960}{0.1} \right)^2 = 96.04
  \end{equation*}
  that is, at least 97 observations always guarantees that a populaiton proportion is estimated with an error of at most 0.1 with a $95\%$ confidence.
\end{example}


\subsection{Small Samples: Student's \textit{t} distribution}
\label{subsec:small-samples}

Having a small sample, we can no longer pretend that a sample standard deviation $s$ is an accurate estimator of the population standard deviation $\sigma$. Then, how should we adjust the confidence interval when we replace $\sigma$ by $s$, or more generally, when we replace the standard error $\sigma(\hat{\theta})$ by its estimator $s(\hat{\theta})$?

A famous solution was proposed by \textit{William Gosset} (1876–1937), he derived the \textbf{T-distribution}.

He replaced the true but unknown standard error of $\hat{\theta}$ by its estimator $s(\hat{\theta})$ and concluded that the \textbf{T-ratio}
\begin{equation*}
  t = \frac{\hat{\theta} - \theta}{s(\hat{\theta})}
\end{equation*}
the \textit{ratio} of two random variables, no longer has a Normal distribution!

For the problem of estimating the mean based on $n$ Normal observations $X_1,\ \ldots,\ X_n$, this was \textbf{T-distribution} with $(n - 1)$ \textit{degrees of freedom}. Table A5 (from textbook) gives critical values $t_{\alpha}$ of the T-distribution that we'll use for confidence intervals.

So, using \textit{T-distribution} instead of Standard Normal and estimated standard error instead of the unknown true one, we obtain the confidence interval for the population mean.

\begin{formula}{Confidence interval for the mean $\sigma$ is unknown}
  \begin{equation}
    \bar{X} \ \pm\ t_{\alpha/2} \frac{s}{\sqrt{n}}
  \end{equation}
  where $t_{\alpha/2}$ is a critical value from T-distribution with $n - 1$ degrees of freedom.
\end{formula}

The density of \textit{Student's T-distribution} is a bell-shaped symmetric curve that can be easily confused with Normal. Comparing with the Normal density, \textit{its peak is lower and its tails are thicker}. Therefore, a larger number $t_{\alpha}$ is generally needed to cut area $\alpha$ from the right tail. That is
\begin{equation*}
  t_{\alpha} > z_{\alpha}
\end{equation*}
for small $\alpha$. As a consequence, the confidence interval (9) is wider than the interval (5) for the case of known $\sigma$. This wider margin is the price paid for not knowing the standard deviation $\sigma$. When we lack a certain piece of information, we cannot get a more accurate estimator.

\vspace*{\fill}
\columnbreak

However, we see in Table A5 that
\begin{equation*}
  t_{\alpha} \ra z_{\alpha}
\end{equation*}
as the number of degrees of freedom $\nu$ tends to infinity. Indeed, having a large sample (hence, large $\nu = n - 1$), we can count on a very accurate estimator of $\sigma$, and thus, the confidence interval is almost as narrow as if we knew $\sigma$ in this case.

\textbf{Degrees of freedom $\nu$} is the parameter of T-distribution controlling the shape of the T-density curve. Its meaning is the dimension of a vector used to estimate the variance. Here we estimate $\sigma^2$ by a sample variance
\begin{equation*}
  s^2 = \frac{1}{n - 1} \sum_{i=1}^{n} (X_i - \bar{X})^2
\end{equation*}
and thus, we use a vector
\begin{equation*}
  \bs{X'} = (X_1 - \bar{X},\ \ldots,\ X_n - \bar{X})
\end{equation*}
The initial vector $\bs{X} = (X_1,\ \ldots,\ X_n)$ has dimension $n$; therefore, it has $n$ degrees of freedom. However, when the sample mean $\bar{X}$ is subtracted from each observation, there appears a linear relation among the elements,
\begin{equation*}
  \sum_{i=1}^{n} (X_i - \bar{X}) = 0
\end{equation*}
We lose 1 degree of freedom due to this constraint; the vector $\bs{X'}$ belongs to an $(n - 1)$.

In many similar problems, degrees of freedom can be computed as
\begin{table}[H]
  \centering
  \begin{tabular}{p{2cm} c p{2cm} c p{2cm} r}
    number of degrees of freedom & $=$ & sample size & $-$ & number of estimated location parameters & (10) \\
  \end{tabular}
\end{table}


\subsection{Comparison of Two Populations with Unknown Variances}
\label{subsec:comp-two-pop-unknown-var}
\setcounter{equation}{10}

We now construct a confidence interval for the difference of two means $\mu_X - \mu_Y$, comparing the population of $X$'s with the population of $Y$'s.

Again, independent random samples are collected,
\begin{align*}
  \bs{X} = (X_1,\ \ldots,\ X_n) &&\textnormal{and}&& \bs{Y} = (Y_1,\ \ldots,\ Y_m)
\end{align*}
one from each population, as in Figure 4. This time, however, population variances $\sigma^2_X$ and $\sigma^2_Y$ are unknown to us, and we use their estimates.

Two important cases need to be considered here. In one case, there exists an exact and simple solution based on T-distribution. The other case suddenly appears to be a famous Behrens-Fisher problem, where no exact solution exists, and only approximations are available.

\vspace*{\fill}
\columnbreak

\subsubsection{Case 1. Equal Variances}

Suppose there are reasons to assume that the two populations have equal variances,
\begin{equation*}
  \sigma^2_X = \sigma^2_Y = \sigma^2
\end{equation*}
For example, two sets of data are collected with the same measurement device, thus, measurements have different means but the same precision. In this case, there is only one variance $\sigma^2$ to estimate instead of two. We should use both samples $\bs{X}$ and $\bs{Y}$ to estimate their common variance. This estimator of $\sigma^2$ is called a \textbf{pooled sample variance}, and it is computed as
\begin{align}
  s_p^2 &= \frac{\dsum_{i=1}^{n} (X_i - \bar{X})^2 + \dsum_{i=1}^{n} (Y_i - \bar{Y})^2}{n + m - 2} \nonumber \\
  &= \frac{(n-1) s_X^2 + (n-1) s_Y^2}{n + m - 2}
\end{align}
Substituting this variance estimator in (6) for $\sigma^2_X$ and $\sigma^2_Y$, we get the following confidence interval.
\begin{formula}{Confidence interval for the difference of means; equal, unknown standard deviations}
  \begin{equation*}
    \bar{X} - \bar{Y} \ \pm\ t_{\alpha/2} s_p \sqrt{\frac{1}{n} + \frac{1}{m}}
  \end{equation*}
  where $s_p$ is the pooled standard deviation, a root of the pooled variance in (11) and $t_{\alpha/2}$ is a critical value from T-distribution with $(n + m - 2)$ degrees of freedom.
\end{formula}

\subsubsection{Case 2. Unequal Variances}

The most difficult case is when both variances are unknown and unequal. Confidence estimation of $\mu_X - \mu_Y$ in this case is known as the \textit{Behrens-Fisher} problem. Certainly, we can replace unknown variances $\sigma^2_X$, $\sigma^2_Y$ by their estimates $s^2_X$, $s^2_Y$ and form a T-ratio
\begin{equation*}
  t = \frac{(\bar{X} - \bar{Y}) - (\mu_X - \mu_Y)}{\sqrt{\dfrac{s_X^2}{n} + \dfrac{s_Y^2}{m}}}
\end{equation*}
However, it won't have a T-distribution.

An approximate solution was proposed in the 1940s by Franklin E. Satterthwaite. Satterthwaite used the method of moments to estimate degrees of freedom $\nu$ of a T-distribution that is ``closest'' to this T-ratio. This number depends on unknown variances. Estimating them by sample variances, he obtained the formula that is now known as \textit{Satterthwaite approximation}
\begin{equation}
  \nu = \dfrac{\left( \dfrac{s_X^2}{n} + \dfrac{s_Y^2}{m} \right)^2}{\dfrac{s_X^4}{n^2(n-1)} + \dfrac{s_Y^4}{m^2(m-1)}}
\end{equation}
This number of degrees of freedom often appears non-integer. There are T-distributions with non-integer $\nu$. To use Table A5, just take the closest $\nu$ that is given in that table.

\vspace*{\fill}
\columnbreak

Formula (12) is widely used for t-intervals and t-tests.
\begin{formula}{Confidence interval for the difference of means; unequal, unknown standard deviation}
  \begin{equation*}
    \bar{X} - \bar{Y} \ \pm \ t_{\alpha/2} \sqrt{\frac{s_X^2}{n} + \frac{s_Y^2}{m}}
  \end{equation*}
  where $t_{\alpha/2}$ is a critical value from T-distribution with $\nu$ degrees of freedom given by formula (12).
\end{formula}

\textbf{\textit{Check the examples of 9.3.4 from book.}}

