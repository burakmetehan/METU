\section{Commands}
\label{sec:commands}

A \textit{\textbf{command}} is a program construct that will be \textit{\textbf{executed}} in order \textit{to update variables}. Commands are a characteristic feature of imperative, object-oriented, and concurrent languages. Commands are often called \textit{statements}.

Commands may be formed in various ways. We survey the following fundamental forms of commands. Some commands are primitive:
\begin{itemize}
  \item \textit{skips}
  \item \textit{assignments}
  \item \textit{proper procedure calls}
\end{itemize}
Others are composed from simpler commands:
\begin{itemize}
  \item \textit{sequential commands}
  \item \textit{collateral commands}
  \item \textit{conditional commands}
  \item \textit{iterative commands}
\end{itemize}
There are also \textit{block commands} and \textit{exception-handling commands}.

A well-designed imperative, object-oriented, or concurrent language should provide \textit{all or most of the above forms of command}; it is impoverished if it omits (or arbitrarily restricts) any important forms. Conversely, a language that provides additional forms of command is probably bloated; the additional ones are likely to be unnecessary accretions rather than genuine enhancements to the language’s expressive power.

All the above commands exhibit \textit{single-entry single-exit} control flow. This pattern of control flow is adequate for most practical purposes. But sometimes it is too restrictive, so modern imperative and object-oriented languages also provide \textit{sequencers} (such as exits and exceptions) that allow us to program \textit{single-entry multi-exit} control flows.

\subsection{Skips}
\label{sec:skips}

The simplest possible kind of command is the \textit{\textbf{skip}} command, which has no effect whatsoever. In \texttt{C}, \texttt{C++} and \texttt{JAVA}, the skip command is written simply ``;''.

\newpage

\subsection{Assignments}
\label{sec:assignments}

We have already encountered the concept of \textit{\textbf{assignment}}. The assignment command typically has the form ``\texttt{$V$ = $E$}''. Here $E$ is an expression which yields a value, and $V$ is a variable access which yields a reference to a variable.

More general kinds of assignment are possible. A \textit{\textbf{multiple assignment}}, typically written in the form ``$V_1 = \ldots = V_n = E$;'', causes the same value to be assigned to several variables.

\subsection{Proper Procedure Call}
\label{sec:proper-procedure-call}

A \textit{\textbf{proper procedure call}} is a command that achieves its effect by applying a proper procedure (or method) to some arguments. The call typically has the form ``$P(E_1, \ldots, E_n)$;'', where $P$ determines the procedure to be applied, and the expressions $E_1, \ldots, E_n$ are evaluated to determine the arguments.

\subsection{Sequential Commands}
\label{sec:seq-com}

Since commands update variables, the order in which commands are executed
is important. 

A \textit{\textbf{sequential command}} specifies that \textit{two (or more) commands are to be executed in sequence}. A sequential command might be written in the form:
\begin{equation*}
  C_1\ ;\ C_2
\end{equation*}
meaning that command $C_1$ is executed before command $C_2$.

\subsection{Collateral Commands}
\label{sec:col-com}

A \textit{\textbf{collateral command}} specifies that \textit{two (or more) commands may be executed in any order}. A collateral command might be written in the form:
\begin{equation*}
  C_1\ ,\ C_2
\end{equation*}
where both $C_1$ and $C_2$ are to be executed, but in no particular order.

An unwise collateral command would be:
\begin{minted}{cpp}
  n = 7,  n = n + 1;
\end{minted}
The net effect of this collateral command depends on the order of execution. Let us suppose that \texttt{n} initially contains 0:
\begin{itemize}
  \item If ``\texttt{n = 7}'' is executed first, \texttt{n} will end up containing 8.
  \item If ``\texttt{n = 7}'' is executed last, \texttt{n} will end up containing 7.
  \item If ``\texttt{n = 7}'' is executed between evaluation of ``\texttt{n + 1}'' and assignment of its value to \texttt{n}, \texttt{n} will end up containing 1.
\end{itemize}

Collateral commands are said to be \textit{nondeterministic}. \textit{A computation is \textbf{deterministic} if the sequence of steps it will perform is entirely predictable; otherwise the computation is \textbf{nondeterministic}}.

If we perform a deterministic computation over and over again, with the same input, it will always produce the same output. But if we perform a nondeterministic computation over and over again, with the same input, it might produce different output every time.

Although the sequence of steps performed by a nondeterministic computation is unpredictable, its output might happen to be predictable. We call such a computation \textit{\textbf{effectively deterministic}}. A collateral command is effectively deterministic if no subcommand inspects a variable updated by another subcommand.

\subsection{Concurrent Commands}
\label{sec:concur-com}

\begin{itemize}
  \item Concurrent paradigm languages:
  \begin{equation*}
    \{\ C_1\ |\ C_2\ |\ \ldots\ |\ C_n\ \}
  \end{equation*}
  \item All commands start concurrently in parallel. Block finishes when the last active command finishes.
  \item Real parallelism in multi-core/multi-processor machines.
  \item Transparently handled by only a few languages. Thread libraries required in languages like \texttt{Java, C, C++}.
\end{itemize}

\subsection{Conditional Commands}
\label{sec:conditional-com}

A \textit{\textbf{conditional command}} has two or more subcommands, of which exactly one is chosen to be executed.

The most elementary form of conditional command is the \textit{\textbf{if-command}}, in which a choice between two subcommands is based on a boolean value. The if-command is found in every imperative language, and typically looks like this:

$\begin{aligned}
  &\textbf{if}\ (E)\ C_1\\
  &\textbf{else}\ C_2
\end{aligned}$

\noindent If the boolean expression $E$ yields \textit{true}, $C_1$ is chosen; if it yields \textit{false}, $C_2$ is chosen.

The if-command can be generalized to allow choice among several subcommands:

$\begin{aligned}
  &\textbf{if}\ (E_1)\ C_1\\
  &\textbf{else if}\ (E_2)\ C_2\\
  &...\\
  &\textbf{else if}\ (E_n)\ C_n\\
  &\textbf{else}\ C_0
\end{aligned}$

\noindent Here the boolean expressions $E_1, E_2, \ldots, E_n$ are evaluated sequentially.

\newpage

The above conditional commands are \textit{deterministic}: in each case we can predict which subcommand will be chosen. A \textit{nondeterministic} conditional command is also sometimes useful, and might be written in the following notation:

$\begin{aligned}
  &\textbf{if}\ (E_1)\ C_1\\
  &\textbf{or if}\ (E_2)\ C_2\\
  &...\\
  &\textbf{or if}\ (E_n)\ C_n\\
\end{aligned}$

\noindent Here the boolean expressions $E_1, E_2, \ldots, E_n$ would be evaluated collaterally, and any $E_i$ that yields \textit{true} would cause the corresponding subcommand $C_i$ to be chosen. If no $E_i$ yields \textit{true}, the command would fail.

Nondeterministic conditional commands tend to be available only in concurrent languages, where nondeterminism is present anyway, but their advantages are not restricted to such languages.

A more general form of conditional command is the \textit{\textbf{case command}}, in which a choice between several subcommands is typically based on an integer (or other) value.

The nearest equivalent to a case command in \texttt{C}, \texttt{C++}, and \texttt{JAVA} is the \textbf{switch} command.

$\begin{aligned}
  &\textbf{switch}\ (E):\\
  &\quad \textbf{case}\ v_1:\ C_1;\ \texttt{break};\\
  &\quad...\\
  &\quad \textbf{case}\ v_n:\ C_n;\ \texttt{break};\\
  &\quad \textbf{case}\ \textbf{default:}\ C_0;\ \texttt{break};
\end{aligned}$

\subsection{Iterative Commands}
\label{sec:iter-com}

An \textit{\textbf{iterative command}} (commonly known as a \textit{loop}) has a subcommand that is executed repeatedly. The latter subcommand is called the loop \textit{\textbf{body}}. Each execution of the loop body is called an \textit{\textbf{iteration}}.

We can classify iterative commands according to when the number of iterations is fixed:

\begin{itemize}
  \item \textit{\textbf{Indefinite iteration:}} the number of iterations is not fixed in advance.
  \item \textit{\textbf{Definite iteration:}} the number of iterations is fixed in advance.
\end{itemize}

Indefinite iteration is typically provided by the \textit{\textbf{while-command}}. The while-command typically looks like this:

\begin{minted}[escapeinside=@@]{cpp}
  while (@$E$@) @$C$@  
\end{minted}
The meaning of the while-command can be defined by the following equivalence:
\begin{minted}[escapeinside=@@]{cpp}
  while (@$E$@) @$C$@  @$\equiv$@ if (@$E$@) {
                      @$C$@
                      while (@$E$@) @$C$@
                    }
\end{minted}
This definition makes clear that the loop condition in a while-command is tested \textit{before} each iteration of the loop body.

Note that this definition of the \textit{while-command is recursive}. In fact, \textit{iteration is just a special form of recursion}. \texttt{C}, \texttt{C++}, and \texttt{JAVA} also have a \textbf{do-while-command}, in which the loop condition is tested \textit{after} each iteration.

Definite iteration is characterized by a \textit{\textbf{control sequence}}, a predetermined sequence of values that are successively assigned (or bound) to a \textit{\textbf{control variable}} such as \textbf{\textit{for-command}} in \texttt{ADA}.
\begin{minted}{ada}
  for V in T loop
    C
  end loop;
\end{minted}

The control sequence consists of all components of the collection. Iteration over an array or list is deterministic, since the components are visited in order. Iteration over a set is nondeterministic, since the components are visited in no particular order.

Note that the for-command of \texttt{C} and \texttt{C++} (and the old-style for-command of \texttt{JAVA}) is nothing more than syntactic shorthand for a while-command, and thus supports indefinite iteration:
\begin{minted}[escapeinside=@@]{cpp}
  for (@$C_1$@; @$E_1$@; @$E_2$@) @$C_2$@; @$\equiv$@ @$C_1$@ while (@$E_1$@) {
                                  @$C_2$@;
                                  @$E_2$@;
                                }
\end{minted}
