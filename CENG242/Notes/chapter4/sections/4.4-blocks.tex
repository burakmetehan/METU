\newpage
\section{Blocks}
\label{sec:blocks}

If we allow a command to contain a local declaration, we have a \textit{block command}. If we allow an expression to contain a local declaration, we have a\textit{block expression}.

\subsection{Block Commands}

A block command is a form of command that contains a local declaration (or group of declarations) \textit{D} and a subcommand \textit{C}. The bindings produced by \textit{D} are used only for executing \textit{C}.

In other words, declarations done inside a block command is available only during the block. Statements inside work in this environment. The declarations lost outside of the block.

\begin{listing}[H]
\begin{minted}{c}
int x = 3, i = 2;
x += i;
while (x > i) {
  int i = 0;
  ...
  i++;
}
/* i is 2 again */
\end{minted}
\caption{}
\label{code:code7}
\end{listing}


\subsection{Block Expressions}

A \textit{\textbf{block expression}} is a form of expression that contains a local declaration (or group of declarations) \textit{D} and a subexpression \textit{C}. The bindings produced by \textit{D} are used only for evaluating \textit{E}.

In other words, it allows an expression to be evaluated in a special local environment. Declarations done in the block is not available outside.

\begin{listing}[H]
\begin{minted}{Haskell}
x=5
t=let xsquare=x*x
      factorial n = if n<2 then 1
                    else n*factorial (n-1)
      xfact = factorial x
  in  (xsquare+1)*xfact/(xfact*xsquare+2)
\end{minted}
\caption{}
\label{code:code8}
\end{listing}

\vspace*{\fill}
\columnbreak

\noindent Hiding works in block expressions as expected:
\begin{listing}[H]

\begin{minted}{Haskell}
x=5 ; y=6 ; z = 3
t=let x=1 
  in let y=2 
     in   x+y+z
{--
t is 1+2+3 here. 
local x and y hides the ones above
--}
\end{minted}
\caption{}
\label{code:code9}
\end{listing}

\noindent GCC (only GCC) block expressions has the last expression in block as the value:
\begin{listing}[H]

\begin{minted}{c}
double min ;
...
min =  ({ double tmp;
          if (b < a) then {
            tmp = a;  a = b ; b = tmp;
          }
          a; // this is the value of the block
        });
\end{minted}
\caption{}
\label{code:code10}
\end{listing}

\subsection{Block Declaration}

A declaration is made in a local environment of declarations. Local declarations are not made available to the outer environment.

In \texttt{Haskell}: \texttt{D$_{exp}$ where D$_1$; D$_2$;  ... ; D$_n$}

\noindent Only \texttt{D$_{exp}$} is added to environment. Body of \texttt{D$_{exp}$} has all local declarations available in its environment.

\begin{listing}[H]

\begin{minted}{Haskell}
fifthpower x = (forthpowerx) * x where
               squarex = x*x;
               forthpowerx = squarex*squarex
\end{minted}
\caption{}
\label{code:code11}
\end{listing}



