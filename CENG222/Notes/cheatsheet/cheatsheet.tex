\documentclass{article}
\usepackage[utf8]{inputenc}
\usepackage{geometry}
\usepackage{graphicx}
\usepackage{amsmath}
\usepackage{amsfonts}
\usepackage{amsthm}
\usepackage{amssymb}
\usepackage[most]{tcolorbox}
\usepackage{array}
\usepackage{latexsym}
\usepackage{alltt}
\usepackage{hyperref}
\usepackage{color}
\usepackage{float}
\usepackage{pdfpages}
\usepackage{algpseudocode}
\usepackage{multicol}
\usepackage{multirow}
\usepackage{caption}
\usepackage{xparse}
\usepackage{setspace}
\usepackage{enumitem}


\geometry
{
    a4paper,
    left=10mm,
    right=10mm,
    top=10mm,
    bottom=12mm,
}

% mybox
\newtcolorbox{mybox}[3][]
{
  colframe = #2!25,
  colback  = #2!10,
  coltitle = #2!20!black,  
  title    = {#3},
  #1,
}

% New environments that use mybox
\newcounter{example}
\newenvironment{example}[1]{\begin{mybox}{green}{\refstepcounter{example}\textbf{Example~\theexample #1}}}{\end{mybox}}

\newenvironment{example_break}[1]{\begin{mybox}[breakable]{green}{\refstepcounter{example}\textbf{Example~\theexample #1}}}{\end{mybox}}

\newcounter{definition}
\newenvironment{definition}[1]{\refstepcounter{definition}\begin{mybox}{blue}{\textbf{Definition~\thedefinition #1}}}{\end{mybox}}

\newcounter{theorem}
\newenvironment{theorem}[1]{\begin{mybox}{red}{\refstepcounter{theorem}\textbf{Theorem~\thetheorem #1}}}{\end{mybox}}

\newenvironment{formula}[1]{\begin{mybox}{cyan}{\textbf{#1}}}{\end{mybox}}

% Changing maketitle
\makeatletter         
\renewcommand\maketitle{
{\raggedright % Note the extra {
\begin{center}
{\Large \bfseries \@title}\\[2ex] 
{\large \@author \ - \@date}\\[2ex]
\end{center}}} % Note the extra }
\makeatother

% \onehalfspacing % adjust spacing
\setlength{\parskip}{0.5\baselineskip}

% macros
\newcommand{\prob}[1]{\textbf{\textit{P}}\left\{#1\right\}}
\newcommand{\expc}[1]{\mathbf{E}\left(#1\right)}
\newcommand{\expcs}[1]{\mathbf{E}^2\left(#1\right)}
\newcommand{\var}[1]{\text{Var}\left( #1 \right)}
\newcommand{\std}[1]{\text{Std}\left( #1 \right)}

\NewDocumentCommand{\dsum}{%
    e{^_}
}{%
  {% 
    \displaystyle\sum
    \IfValueT{#1}{^{#1}}
    \IfValueT{#2}{_{#2}}
  }
}%

% maketitle variables
\title{CENG 222 - Cheatsheet}
\author{Burak Metehan Tunçel}
\date{April 2022}

\begin{document}

\maketitle

% \setcounter{section}{1}

\documentclass{article}
\usepackage[utf8]{inputenc}
\usepackage{geometry}
\usepackage{graphicx}
\usepackage{wrapfig}
\usepackage{subfig}
\usepackage{subfigure}
\usepackage{amsmath}
\usepackage{amsfonts}
\usepackage{amsthm}
\usepackage[most]{tcolorbox}
\usepackage{fancybox}
\usepackage{verbatim}
\usepackage{array}
\usepackage{latexsym}
\usepackage{alltt}
\usepackage{hyperref}
\usepackage{textcomp}
\usepackage{color}
\usepackage{float}
\usepackage{pdfpages}
\usepackage{algorithm}
\usepackage[noend]{algpseudocode}
\usepackage{multicol}
\usepackage{minted}

\usepackage[T1]{fontenc}
\usepackage{libertine}%% Only as example for the romans/sans fonts
\usepackage[scaled=0.85]{beramono}

\usepackage{colortbl}
\usepackage{pifont}
\newcommand{\cmark}{\ding{51}}
\newcommand{\xmark}{\ding{55}}

\geometry
{
 a4paper,
 left=15mm,
 top=15mm,
}

\newtcolorbox{mybox}[3][]
{
  colframe = #2!25,
  colback  = #2!10,
  coltitle = #2!20!black,  
  title    = {#3},
  #1,
}

\renewcommand\qedsymbol{$\triangle$}

\newenvironment{example}{\begin{mybox}{green}{\textbf{Example}}}{\end{mybox}}
\newenvironment{definition}[1]{\begin{mybox}{blue}{\textbf{Definition #1}}}{\end{mybox}}
\newenvironment{theorem}[1]{\begin{mybox}{red}{\textbf{Theorem #1}}}{\end{mybox}}

\def\bsq#1{%both single quotes
\lq{#1}\rq}

\makeatletter         
\renewcommand\maketitle{
{\raggedright % Note the extra {
\begin{center}
{\Large \bfseries \@title}\\[2ex] 
{\large \@author \ - \@date}\\[2ex]
\end{center}}} % Note the extra }
\makeatother

\title{CENG 242 - Chapter 2: Values and Types}
\author{Burak Metehan Tunçel}
\date{March 2022}

\begin{document}

\maketitle

\section{Types}

A \textbf{\textit{value}} is any entity that can be manipulated by a program. Values can be \textit{evaluated, stored, passed} as \textit{arguments, returned as function results}, and so on. Different programming languages support different types of values: \texttt{C} supports integers, real numbers, structures, arrays, unions, pointers to variables, and pointers to functions; \texttt{C++} supports all the above types of values plus objects. 

Most programming languages group values into \textit{types}. For instance, nearly all languages make a clear distinction between integer and real numbers. Most languages also make a clear distinction between booleans and integers: integers can be added and multiplied, while booleans can be subjected to operations like not, and, and or.

\begin{multicols}{2}
\begin{itemize}
    \item \textbf{C Types:}
        \begin{itemize}
            \item int, char, long,...
            \item float, double
            \item pointers
            \item structures: struct, union
             \item arrays
        \end{itemize}
    
    \vfill\null
    \columnbreak
    
    \item \textbf{Haskell Types:}
        \begin{itemize}
            \item Bool, Int, Float, ...
            \item Char, String
            \item tuples,(N-tuples), records
            \item lists
            \item functions
        \end{itemize}
\end{itemize}
\end{multicols}


What exactly is a type? The most obvious answer, perhaps, is that a type is a set of values. When we say that $v$ is a value of type $T$, we mean simply that $v \in T$. When we say that an expression $E$ is of type $T$, we are asserting that the result of evaluating $E$ will be a value of type $T$.

However, not every set of values is suitable to be regarded as a type. We insist that each operation associated with the type behaves uniformly when applied to all values of the type. Thus \{$false$, $true$\} is a type because the operations not, and, and or operate uniformly over the values $false$ and $true$. Also, $\{...,\ -2,\ -1,\ 0,\ +1,\ +2,...\}$ is a type because operations such as addition and multiplication operate uniformly over all these values. 

But \{13, $true$, Monday\} is not a type, since there are no useful
operations over this set of values. Thus we see that a type is characterized not only by its set of values, but also by the operations over that set of values.

Therefore we define a \textbf{\textit{type}} to be a set of values, equipped with one or more operations that can be applied uniformly to all these values. 

Every programming language supports both primitive types and composite types. Some languages also have recursive types, a recursive type being one whose values are composed from other values of the same type.

\section{Primitive Types}

A \textbf{\textit{primitive value}} is one that cannot be decomposed into simpler values. A \textbf{\textit{primitive type}} is one whose values are primitive.

Every programming language provides built-in primitive types. Some languages also allow programs to define new primitive types.

\subsection{Built-in Primitive Types}

One or more primitive types are built-in to every programming language. The choice of built-in primitive types tells us much about the programming language’s intended application area. 

Nevertheless, certain primitive types crop up in a variety of languages, often under different names. For the sake of consistency, we shall use \texttt{Boolean, Character, Integer}, and \texttt{Float} as names for the most common primitive types:
\begin{itemize}
    \item \texttt{Boolean} = \{$false$, $true$\}
    \item \texttt{Character} = \{..., \bsq a, ..., \bsq z, ..., \bsq 0, ..., \bsq 9, ..., \bsq ?, ...\}
    \item \texttt{Integer} = \{..., -2, -1, 0, +1, +2, ...\}
    \item \texttt{Float} = \{..., -1.0, ..., 0.0, ..., +1.0, ...\}
\end{itemize}

The \textbf{\textit{cardinality}} of a type $T$, written \#$T$, is the number of distinct values in $T$. For example:
\begin{itemize}
    \item \#Boolean = 2
    \item \#Character = 256 (ISO LATIN character set)
    \item \#Character = 65 536 (UNICODE character set)
\end{itemize}

\newline

If some types are implementation-defined, the behavior of programs may vary from one computer to another, even programs written in high-level languages. This gives rise to portability problems: a program that works well on one computer might fail when moved to a different computer.

One way to avoid such portability problems is for the programming language to define all its primitive types precisely. As we have seen, this approach is taken by JAVA.

\subsection{Defined Primitive Types}

Another way to avoid portability problems is to allow programs to define their own integer and floating-point types, stating explicitly the desired range and/or precision for each type. For example, the following declaration can be used in \texttt{ADA}: \lq \texttt{\textbf{type} Population \textbf{is range} 0 ... 1e10;}\rq. In ADA, we can define a completely new primitive type by enumerating its
values. Such a type is called an \textbf{\textit{enumeration type}}, and its values are called \textit{\textbf{enumerands}}.

\subsection{Discrete Primitive Types}

A \textbf{\textit{discrete primitive type}} is a primitive type whose values have a \textit{one-to-one relationship with a range of integers}.

This is an important concept in \texttt{ADA}, in which values of any discrete primitive type may be used for array indexing, counting, and so on. The discrete primitive types in \texttt{ADA} are \texttt{Boolean, Character}, integer types, and enumeration types.

Most programming languages allow only integers to be used for counting and array indexing. \texttt{C} and \texttt{C++} allow enumerands also to be used for counting and array indexing, since they classify enumeration types as integer types.


\section{Composite Types}

A \textbf{\textit{composite value}} (or \textit{data structure}) is a value that is composed from simpler values. A \textbf{\textit{composite type}} is a type whose values are composite.

Programming languages support a huge variety of composite values: tuples, structures, records, arrays, algebraic types, discriminated records, objects, unions, strings, lists, trees, sequential files, direct files, relations, etc. The variety might seem bewildering, but in fact nearly all these composite values can be understood in terms of a small number of structuring concepts, which are:
\begin{itemize}
    \item Cartesian products (tuples, records)
    \item mappings (arrays)
    \item disjoint unions (algebraic types, discriminated records, objects)
    \item recursive types (lists, trees).
\end{itemize}

Each programming language provides its own notation for describing composite types. Here, mathematical notation that is concise, standard, and suitable for defining sets of values structured as Cartesian products, mappings, and disjoint unions will be used.

\subsection{Cartesian Products, Structures, and Records}

In a \textbf{\textit{Cartesian product}}, values of several (possibly different) types are grouped into tuples.

We use the notation $(x,\ y)$ to stand for the pair whose first component is $x$ and whose second component is $y$. We use the notation $S \times T$ to stand for the set of all pairs $(x,\ y)$ such that $x$ is chosen from set $S$ and $y$ is chosen from set $T$. Formally:

\begin{equation}
    S \times T = \{(x,\ y)\ |\ x \in S;\ y \in T\}
\end{equation}

\newpage

The basic operations on pairs are:
\begin{itemize}
    \item \textbf{\textit{construction}} of a pair from two component values;
    \item \textbf{\textit{selection}} of the first or second component of a pair.
\end{itemize}

We can easily infer the cardinality of a Cartesian product:
\begin{equation}
    \text{\#}(S \times T) = \text{\#}S \times \text{\#}T
\end{equation}
This equation motivates the use of the notation ‘‘$\times$’’ for Cartesian product.

We can extend the notion of Cartesian product from pairs to tuples with any number of components. In general, the notation $S_1 \times S_2 \times ... \times S_n$ stands for the set of all $n$-tuples, such that the first component of each $n$-tuple is chosen from $S_1$, the second component from $S_2$, ..., and the $n^{th}$ component from $S_n$.

C \texttt{struct}, Pascal \texttt{record}, functional languages \texttt{tuple}
\begin{multicols}{3}

\textbf{in C:} \texttt{string} $\times$ \texttt{int}

\begin{minted}
[
%frame=lines,
%framesep=2mm,
baselinestretch=1.2,
bgcolor=lightgray,
fontsize=\footnotesize,
%linenos
]
{c}
struct Person {
    char name[20];
    int no;
} x = {" Osman Hamdi " ,23141};
\end{minted}

\vfill\null
\columnbreak

\textbf{in Haskell:} \texttt{string} $\times$ \texttt{int}   
\begin{minted}
[
%frame=lines,
%framesep=2mm,
baselinestretch=1.2,
bgcolor=lightgray,
fontsize=\footnotesize,
%linenos
]
{haskell}
type People = (String, Int)
...
x = ("Osman Hamdi", 23141)::People
\end{minted}

\vfill\null
\columnbreak

\textbf{in Python:} \texttt{string} $\times$ \texttt{int} 
\begin{minted}
[
%frame=lines,
%framesep=2mm,
baselinestretch=1.2,
bgcolor=lightgray,
fontsize=\footnotesize,
%linenos
]
{py}
x = ("Osman Hamdi", 23141)
type (x)
<type 'tuple'>
\end{minted}
\end{multicols}

A special case of a Cartesian product is one where all tuple components are chosen from the same set. The tuples in this case are said to be \textbf{\textit{homogeneous}}. For example:
\begin{equation}
    S^2 = S \times S
\end{equation}
means the set of homogeneous pairs whose components are both chosen from set $S$. More generally we write:
\begin{equation}
    S^n = S \times S \times \cdots \times S
\end{equation}
to mean the set of homogeneous $n$-tuples whose components are all chosen from set $S$.

The cardinality of a set of homogeneous $n$-tuples is given by:
\begin{equation}
    \text{\#}(S^n) = (\text{\#}S)^n
\end{equation}
This motivates the superscript notation.

Finally, let us consider the special case where $n = 0$. Equation 5 tells us that $S_0$ should have exactly one value. This value is the empty tuple $()$, which is the unique tuple with no components at all. We shall find it useful to define a type that has the empty tuple as its only value:
\begin{equation}
    \text{Unit} = \{()\}
\end{equation}
This type’s cardinality is:
\begin{equation}
    \text{\#Unit} = 1
\end{equation}
Note that Unit is \textit{not} the empty set (whose cardinality is 0). Unit corresponds to the type named \texttt{\textbf{void}} in \texttt{C}, \texttt{C++}, and \texttt{JAVA}, to the type \textbf{null record} in \texttt{ADA}, and to the  \texttt{\textbf{None}} in \texttt{Python}.

\subsection{Mappings, Arrays, and Functions}

The notion of a \textbf{\textit{mapping}} from one set to another is extremely important in programming languages. This notion in fact underlies two apparently different language features: \textit{arrays} and \textit{functions}.

We write:
\begin{equation}
    m : S \rightarrow T
\end{equation}
to state that $m$ is a mapping from set $S$ to set $T$. In other words, $m$ maps every value in $S$ to a value in $T$. (Read the symbol ‘‘$\rightarrow$’’ as ‘‘maps to’’.)

If $m$ maps value $x$ in set $S$ to value $y$ in set $T$, we write $y = m(x)$. The value $y$ is called the image of $x$ under $m$.

Two different mappings from $S = \{u,\ v\}$ to $T = \{a,\ b,\ c\}$ are illustrated in Figure 1. We use notation such as $\{u \rightarrow a,\ v \rightarrow c\}$ to denote the mapping that maps $u$ to $a$ and $v$ to $c$.

\begin{figure}[h!]
    \centering
    \includegraphics[width=.5\textwidth]{img/Fig2.2.png}
    \caption{Two different mappings in $S \rightarrow T$}
    \label{fig:my_label}
\end{figure}

The notation $S \rightarrow T$ stands for the set of all mappings from $S$ to $T$. Formally:
\begin{equation}
    S \rightarrow T = \{m\ |\ x \in S \Rightarrow m(x) \in T\}
\end{equation}
This is illustrated in Figure 2.

\begin{figure}[h!]
    \centering
    \includegraphics[width=.5\textwidth]{img/Fig2.3.png}
    \caption{Set of all mappings in $S \rightarrow T$}
    \label{fig:my_label}
\end{figure}

Let us deduce the cardinality of $S \rightarrow T$. Each value in $S$ has \#$T$ possible images under a mapping in $S \rightarrow T$. There are \#$S$ such values in $S$. Therefore there are \#$T$ $\times$ \#$T$ $\times \cdots \times$ \#$T$ possible mappings (\#$S$ copies of \#$T$ multiplied together). In short:
\begin{equation}
    \text{\#}(S \rightarrow T) = (\text{\#}T)^{\text{\#}S}
\end{equation}

An \textbf{\textit{array}} is an indexed sequence of components. An array has one component of type $T$ for each value in type $S$, so the array itself has type $S \rightarrow T$. The \textbf{\textit{length}} of the array is its number of components, which is \#$S$. Arrays are found in all imperative and object-oriented languages.

The type $S$ must be finite, so an array is a \textit{finite} mapping. In practice, $S$ is always a range of consecutive values, which is called the array’s \textit{\textbf{index range}}. The limits of the index range are called its \textbf{\textit{lower bound}} and \textbf{\textit{upper bound}}.

The basic operations on arrays are:
\begin{itemize}
    \item \textbf{\textit{construction}} of an array from its components;
    \item \textbf{\textit{indexing}}, i.e., selecting a particular component of an array, given its index.
\end{itemize}
The index used to select an array component is a computed value. Thus array indexing differs fundamentally from Cartesian-product selection (where the component to be selected is always explicit).

\texttt{C} and \texttt{C++} restrict an array’s index range to be a range of integers whose lower bound is zero. ($S$ is integers whose lower bound is 0.)

Mappings occur in programming languages, not only as arrays, but also as \textbf{\textit{function procedures}} (more usually called simply \textit{functions}). We can implement a mapping in $S \rightarrow T$ by means of a function procedure, which takes a value in $S$ (the \textbf{\textit{argument}}) and computes its image in $T$ (the \textbf{\textit{result}}). Here the set $S$ is not necessarily finite.

\subsubsection{Array and Function Difference}

\begin{multicols}{2}
\textbf{Arrays:}
\begin{itemize}
    \item Values stored in memory
    \item Restricted: only integer domain
    \item double $\rightarrow$ double ?
\end{itemize}

\vfill\null
\columnbreak

\textbf{Functions:}
\begin{itemize}
    \item Defined by algorithms
    \item Efficiency, resource usage
    \item All types of mappings possible
    \item Side effect, output, error, termination problem.
\end{itemize}
\end{multicols}

\subsection{Disjoint Unions, Discriminated Records, and Objects}

Another kind of composite value is the \textbf{\textit{disjoint union}}, whereby a value is chosen from one of several (usually different) sets.

We use the notation $S + T$ to stand for a set of disjoint-union values, each of which consists of a \textbf{\textit{tag}} together with a \textbf{\textit{variant}} chosen from either set $S$ or set $T$. The tag indicates the set from which the variant was chosen. Formally:
\begin{equation}
    S + T = \{left\ x\ |\ x \in S\} \cup \{right\ y\ |\ y \in T\}
\end{equation}
Here $left\ x$ stands for a disjoint-union value with tag $left$ and variant $x$ chosen from $S$, while $right\ x$ stands for a disjoint-union value with tag $right$ and variant $y$ chosen from $T$. This is illustrated in Figure 2.4.
\begin{figure}[h!]
    \centering
    \includegraphics[width=.5\textwidth]{img/Fig2.4.png}
    \caption{Disjoint union of sets $S$ and $T$}
    \label{fig:my_label}
\end{figure}

When we wish to make the tags explicit, we will use the notation $left\ S + right\ T$:
\begin{equation}
    left\ S + right\ T = \{left\ x\ |\ x \in S\} \cup \{right\ y\ |\ y \in T\}
\end{equation}
When the tags are irrelevant, we will still use the simpler notation $S + T$.

Note that the tags serve only to distinguish the variants. They must be distinct, but otherwise they may be chosen freely. 

The basic operations on disjoint-union values in $S + T$ are:
\begin{itemize}
    \item \textbf{\textit{construction}} of a disjoint-union value, by taking a value in either $S$ or $T$ and tagging it accordingly;
    \item \textbf{\textit{tag test}}, determining whether the variant was chosen from $S$ or $T$;
    \item \textbf{\textit{projection}} to recover the variant in $S$ or the variant in $T$ (as the case may be).
\end{itemize}
For example, a tag test on the value $right\ b$ determines that the variant was chosen from $T$, so we can proceed to project it to recover the variant $b$.

We can easily infer the cardinality of a disjoint union:
\begin{equation}
    \text{\#}(S + T) = \text{\#}S + \text{\#}T
\end{equation}
This motivates the use of the notation ‘‘$+$’’ for disjoint union.

We can extend disjoint union to any number of sets. In general, the notation $S_1 + S_2 + \cdots + S_n$ stands for the set in which each value is chosen from one of $S_1,\ S_2,\ \cdots$, or $S_n$.

The functional language \texttt{HASKELL} has \textbf{\textit{algebraic types}}, which we can understand in terms of disjoint unions. In fact, the \texttt{HASKELL} notation is very close to our mathematical disjoint-union notation.\newline


How should we understand \textbf{\textit{objects}}? Simplistically we could view each object of a particular class as a tuple of components. However, any object-oriented language allows objects of different classes to be used interchangeably (to a certain extent), and therefore provides an operation to test the class of a particular object. Thus each object must have a tag that identifies its class. So we shall view each object as a \textit{tagged} tuple.


It is \textit{important not to confuse disjoint union with ordinary set union}. The tags in a disjoint union $S + T$ allow us to test whether the variant was chosen from $S$ or $T$. This is not necessarily the case in the ordinary union $S \cup T$. In fact, if $T = \{a,\ b,\ c\}$, then:
\begin{align}
    T \cup T &= \{a,\ b,\ c\} = T\\
    T + T &= \{left\ a,\ left\ b,\ left\ c,\ right\ a,\ right\ b,\ right\ c\} \neq T
\end{align}

The \textbf{unions} of C and C++ are not disjoint unions, since they have no tags. This obviously makes tag test impossible, and makes projection unsafe. In practice, therefore, C programmers enclose each union within a structure that also contains a tag.

\subsection{Power Set}

The set of all subsets. $\mathcal{P}(S) = \{T\ |\ T \subseteq S\}$. Cardinality of set \#$\mathcal{P}(S) = 2^{\text{\#}S}$ \newpage


\section{Recursive Types}

A \textbf{\textit{recursive type}} is one defined in terms of itself. In this section we discuss two common recursive types, lists and strings, as well as recursive types in general.

\subsection{Lists}

A \textbf{\textit{list}} is a sequence of values. A list may have any number of components, including none. The number of components is called the \textbf{\textit{length}} of the list. The unique list with no components is called the \textbf{\textit{empty list}}.

A list is \textbf{\textit{homogeneous}} if all its components are of the same type; otherwise it is \textbf{\textit{heterogeneous}}. 

Typical list operations are:
\begin{itemize}
    \item length
    \item emptiness test
    \item head selection (i.e., selection of the list’s first component)
    \item tail selection (i.e., selection of the list consisting of all but the first component)
    \item concatenation
\end{itemize}

Suppose that we wish to define a type of \textit{integer-lists}, whose values are lists of integers. We may define an integer-list to be a value that is either empty or a pair consisting of an integer (its head) and a further integer-list (its tail). This definition is recursive. We may write this definition as a set equation:
\begin{equation}
    \texttt{Integer-List} = nil\ \texttt{Unit} + cons(\texttt{Integer} × \texttt{Integer-List})
\end{equation}
or, in other words:
\begin{equation}
    \texttt{Integer-List} = \{nil()\} \cup \{cons(i, l)\ |\ i \in \texttt{Integer};\ l \in \texttt{Integer-List}\}
\end{equation}
where we have chosen the tags \textit{nil} for an empty list and \textit{cons} for a nonempty list. Henceforth we shall abbreviate \textit{nil}() to \textit{nil}.

\textbf{Polymorphic lists:} a single definition defines lists of many types. So, the general idea:
\begin{align}
    &L = \text{Unit} + (T \times L) &\texttt{List} \alpha = \alpha \times (\texttt{List} \alpha) + \{\textit{empty}\}
\end{align}
has a \textit{least solution} for \textit{L} that corresponds to the set of all finite lists of values chosen from \textit{T}. Every other solution is a superset of the least solution.

Lists (or sequences) are so ubiquitous that they deserve a notation of their
own: $T^*$ stands for the set of all finite lists of values chosen from $T$. Thus:
\begin{equation}
    T^* = \text{Unit} + (T \times T^*)
\end{equation}

In imperative languages (such as \texttt{C, C++}, and \texttt{ADA}), recursive types must be defined in terms of pointers. In functional languages (such as \texttt{HASKELL}) and in some object-oriented languages (such as \texttt{JAVA}), recursive types can be defined directly.

\subsection{Strings}

A \textbf{\textit{string}} is a sequence of characters. A string may have any number of characters, including none. The number of characters is called the \textbf{\textit{length}} of the string. The unique string with no characters is called the \textbf{\textit{empty string}}.

Strings are supported by all modern programming languages. Typical string operations are:
\begin{itemize}
    \item length
    \item equality comparison
    \item lexicographic comparison
    \item character selection
    \item substring selection
    \item concatenation
\end{itemize}

How should we classify strings? No consensus has emerged among programming language designers. 
\begin{itemize}
    \item One approach is to classify strings as \textit{primitive values}. 
    \item Another approach is to treat strings as \textit{arrays of characters}.
    \item A slightly different and more flexible approach is to treat strings as \textit{pointers to arrays of characters}.
    \item In a programming language that supports lists, the most natural approach is to treat strings as \textit{lists of characters}.
    \item In an object-oriented language, the most natural approach is to treat strings as \textit{objects}.
\end{itemize}

\subsection{Recursive Types in General}

As we have seen, a \textit{recursive type} is one defined in terms of itself. Values of a recursive type are composed from values of the same type.

In general, the set of values of a recursive type, \textit{R}, will be defined by a recursive set equation of the form:
\begin{equation}
    R = ... + (...R...R...)
\end{equation}
A recursive set equation may have many solutions. Fortunately, a recursive set equation always has a \textit{least solution} that is a subset of every other solution. In computation, the least solution is the one in which we are interested.

The cardinality of a recursive type is \textit{infinite}, even if every individual value of the type is finite.


\section{Type Systems}

A programming language’s \textbf{\textit{type system}} groups values into types. This allows programmers to describe data effectively. It also helps prevent programs from performing nonsensical operations, such as multiplying a string by a boolean. Performing such a nonsensical operation is called a \textbf{\textit{type error}}.

\subsection{Static vs Dynamic Typing}

Types are required to provide data processing, integrity checking, efficiency, access controls. Type compatibility on operators is essential. Before any operation is to be performed, the types of its operands must be checked in order to prevent a type error. For example, before an integer multiplication is performed, both operands must be checked to ensure that they are integers.  Such checks are called \textbf{\textit{type checks}}.

The type check must be performed before the operation itself is performed. However, there may still be some freedom in the timing: the type check could be performed either at \textit{compile-time} or at \textit{run-time}. This seemingly pragmatic issue in fact underlies an important classification of programming languages, into statically typed and dynamically typed languages.

\begin{itemize}
    \item In a \textbf{\textit{statically typed}} language, each variable and each expression has a \textit{fixed type} (which is either explicitly stated by the programmer or inferred by the compiler). Using this information, all operands can be type-checked \textit{at compile-time}.
    \item In a \textbf{\textit{dynamically typed}} language, values have fixed types, but variables and expressions have no fixed types. Every time an operand is computed, it could yield a value of a different type. So operands must be type-checked after they are computed, but before performing the operation, \textit{at run-time}.
\end{itemize}

The choice between static and dynamic typing is essentially pragmatic:
\begin{itemize}
    \item \textit{Static typing is more efficient}. Dynamic typing requires (possibly repeated) run-time type checks, which slow down the program’s execution. Static typing requires only compile-time type checks, whose cost is minimal (and one-off). Moreover, dynamic typing forces all values to be tagged (in order to make run-time type checks possible), and these tags take up storage space. Static typing requires no such tagging.
    \item \textit{Static typing is more secure}: the compiler can certify that the program contains no type errors. Dynamic typing provides no such security. (This point is important because type errors account for a significant proportion of programming errors.)
    \item \textit{Dynamic typing provides greater flexibility}, which is needed by some applications where the types of the data are not known in advance.
\end{itemize}

In practice the greater security and efficiency of static typing outweigh the
greater flexibility of dynamic typing in the vast majority of applications. It is no coincidence that most programming languages are statically typed.

\subsection{Type Equivalence}

Consider some operation that expects an operand of type $T_1$. Suppose that it is given instead an operand whose type turns out to be $T_2$. Then we must check whether $T_1$ is equivalent to $T_2$, written $T_1 \equiv T_2$. What exactly this means depends on the programming language.

One possible definition of type equivalence is \textbf{\textit{structural equivalence}}: $T_1 \equiv T_2$ if and only if $T_1$ and $T_2$ have the same set of values. 

Structural equivalence is so called because it may be checked by comparing the \textit{structures} of the types $T_1$ and $T_2$. (It is unnecessary, and in general even impossible, to enumerate all values of these types.)

The following rules illustrate how we can decide whether types $T_1$ and
$T_2$, defined in terms of \textit{Cartesian products}, \textit{disjoint unions}, and \textit{mappings}, are structurally equivalent or not.
\begin{itemize}
    \item If $T_1$ and $T_2$ are both primitive, then $T_1 \equiv T_2$ if and only if $T_1$ and $T_2$ are identical. For example:
        \begin{align*}
            &\texttt{Integer} \equiv \texttt{Integer}&&&&\\
            &\texttt{Integer} \not\equiv \texttt{Float} &&&&\\
            &\text{(The symbol ``$\not\equiv$'' means ‘‘is not equivalent to’’.)}&&&&
        \end{align*}
    
    \item If $T_1 = A_1 \times B_1$ and $T_2 = A_2 \times B_2$, then $T_1 \equiv T_2$ if and only if $A_1 \equiv A_2$ and $B_1 \equiv B_2$. For example:
        \begin{align*}
            &\texttt{Integer} \times \texttt{Float} \equiv \texttt{Integer} \times \texttt{Float} &&&&\\
            &\texttt{Integer} \times \texttt{Float} \not\equiv \texttt{Float} \times \texttt{Integer} &&&&
        \end{align*}
    
    \item If $T = A_1 \mapsto B_1$ and $T_2 = A_2 \mapsto B_2$, then $T_1 \equiv T_2$ if and only if $A_1 \equiv A_2$ and $B_1 \equiv B_2$. For example:
        \begin{align*}
            &\texttt{Integer} \mapsto \texttt{Float} \equiv \texttt{Integer} \mapsto \texttt{Float}&&&&\\
            &\texttt{Integer} \mapsto \texttt{Float} \not\equiv \texttt{Integer} \mapsto \texttt{Boolean}
        \end{align*}
    
    \item If $T_1 = A_1 + B_1$ and $T_2 = A_2 + B_2$, then $T_1 \equiv T_2$ if and only if either $A_1 \equiv A_2$ and $B_1 \equiv B_2$, or $A_1 \equiv B_2$ and $B_1 \equiv A_2$. For example:
        \begin{align*}
            &\texttt{Integer} + \texttt{Float} \equiv \texttt{Integer} + \texttt{Float} &&&&\\
            &\texttt{Integer} + \texttt{Float} \equiv \texttt{Float} + \texttt{Integer} &&&&\\
            &\texttt{Integer} + \texttt{Float} \not\equiv \texttt{Integer} + \texttt{Boolean}
        \end{align*}
    
    \item Otherwise $T_1 \not\equiv T_2$
\end{itemize}

Although these rules are simple, it is not easy to see whether two recursive
types are structurally equivalent. In other words, it is harder to implement structural equality, especially in recursive cases. Consider the following:
\begin{align*}
    T_1 &= \text{Unit} + (S \times T_1)\\
    T_2 &= \text{Unit} + (S \times T_2)\\
    T_3 &= \text{Unit} + (S \times T_3)
\end{align*}
Intuitively, these three types are all structurally equivalent. However, the reasoning needed to decide whether two arbitrary recursive types are structurally equivalent makes type checking uncomfortably hard.

Another possible definition of type equivalence is \textbf{\textit{name equivalence}}: $T_1 \equiv T_2$ if and only if $T_1$ and $T_2$ were defined in the same place.

The following summarizes the advantages and disadvantages of structural and
name equivalence:
\begin{itemize}
    \item \textit{Name equivalence forces each distinct type to be defined in one and only one place}. This is sometimes inconvenient, but it helps to make the program more maintainable. (If the same type is defined in several places, and subsequently it has to be changed, the change must be made consistently in several places.)
    \item \textit{Structural equivalence allows confusion between types that are only coincidentally similar}.
\end{itemize}

\textit{Most programming languages use name equivalence.}

\subsection{The Type Completeness Principle}

First order values needs to have the following:
\begin{itemize}
    \item Assignment
    \item Function parameter
    \item Take part in compositions
    \item Return value from a function
\end{itemize}

Most imperative languages (\texttt{Pascal}, \texttt{Fortran}) classify functions as second order value (\texttt{C} represents function names as pointers). Functions are first order values in most functional languages like \texttt{Haskell} and \texttt{Scheme}.

We can characterize a language’s class-consciousness in terms of its adherence
to the \textbf{\textit{Type Completeness Principle}}:
\begin{quote}
    No operation should be arbitrarily restricted in the types of its operands.
\end{quote}
For another definition,
\begin{quote}
    First order values should take part in all operations above, no arbitrary restrictions should exist.
\end{quote}
The word \textit{arbitrarily} is important here. Insisting that the operands of the and operation are booleans is not an arbitrary restriction, since it is inherent in the nature of this operation. But insisting that only values of certain types can be assigned is an arbitrary restriction, as is insisting that only values of certain types can be passed as arguments or returned as function results.

% C Types
\begin{table}[h!]
\centering
\begin{tabular}{|l|c c c c|}
\hline
\rowcolor[HTML]{96FFFB} 
{\color[HTML]{000000} \textbf{C Types}} & {\color[HTML]{000000} Primitive} & {\color[HTML]{000000} Array} & {\color[HTML]{000000} Struct} & {\color[HTML]{000000} Functions} \\ \hline
Assigment                               & \cmark            & \xmark        & \cmark         & \xmark            \\ \hline
Functions parameter                     & \cmark            & \xmark        & \cmark         & \xmark            \\ \hline
Function return                         & \cmark            & \xmark        & \cmark         & \xmark            \\ \hline
In compositions                         & \cmark            & \cmark        & \cmark         & \xmark            \\ \hline
\end{tabular}
\end{table}

% Haskell Types
\begin{table}[h!]
\centering
\begin{tabular}{|l|c c c c|}
\hline
\rowcolor[HTML]{67FD9A} 
{\color[HTML]{000000} \textbf{Haskell Types}} & {\color[HTML]{000000} Primitive} & {\color[HTML]{000000} Array} & {\color[HTML]{000000} Struct} & {\color[HTML]{000000} Functions} \\ \hline
Assigment                                     & \cmark            & \cmark        & \cmark         & \cmark            \\ \hline
Functions parameter                           & \cmark            & \cmark        & \cmark         & \cmark            \\ \hline
Function return                               & \cmark            & \cmark        & \cmark         & \cmark            \\ \hline
In compositions                               & \cmark            & \cmark        & \cmark         & \cmark            \\ \hline
\end{tabular}
\end{table}

% Pascal Types
\begin{table}[h!]
\centering
\begin{tabular}{|l|c c c c|}
\hline
\rowcolor[HTML]{FFCCC9} 
{\color[HTML]{000000} \textbf{Pascal Types}} & {\color[HTML]{000000} Primitive} & {\color[HTML]{000000} Array} & {\color[HTML]{000000} Struct} & {\color[HTML]{000000} Functions} \\ \hline
Assigment                                    & \cmark            & \cmark        & \cmark         & \xmark            \\ \hline
Functions parameter                          & \cmark            & \cmark        & \cmark         & \xmark            \\ \hline
Function return                              & \cmark            & \xmark        & \xmark         & \xmark            \\ \hline
In compositions                              & \cmark            & \cmark        & \cmark         & \xmark            \\ \hline
\end{tabular}
\end{table}


\section{Expression}

An \textbf{\textit{expression}} is a construct that will be \textbf{\textit{evaluated}} to yield a value. Expressions may be formed in various ways. In this section we shall survey the fundamental forms of expression:
\begin{multicols}{2}
\begin{itemize}
    \item Literals
    \item Constructions
    \item Function Calls
    \item Conditional Expression
\end{itemize}

\begin{itemize}
    \item Iterative Expression
    \item Constant and Variable Accesses
    \item Aggregates
    \item Block Expression
\end{itemize}
\end{multicols}
    

\subsection{Literals}

The simplest kind of expression is a \textbf{\textit{literal}}, which denotes a fixed value of some type.

\subsection{Constructions}

A \textbf{\textit{construction}} is an expression that constructs a composite value from its component values. In some languages the component values must be literals; in others, the component values are computed by evaluating subexpressions.

\subsection{Function Calls}

A \textbf{\textit{function call}} computes a result by applying a function procedure (or method) to one or more arguments. The function call typically has the form ``$F(E)$'', where $F$ determines the function procedure to be applied, and the expression $E$ is evaluated to determine the argument.

An \textbf{\textit{operator}} may be thought of as denoting a function. Applying a unary or binary operator to its operand(s) is essentially equivalent to a function call with one or two argument(s):
$\oplus E$ is essentially equivalent to $\oplus(E)$ (where $\oplus$ is a unary operator)
$E_1 \otimes E_2$ is essentially eqivalent to $\otimes(E_1, E_2)$ (where $\otimes$ is a binary operator.

\subsection{Conditional Expressions}

A \textbf{\textit{conditional expression}} computes a value that depends on a condition. It has two or more subexpressions, from which exactly one is chosen to be evaluated.

\subsection{Iterative Expressions}

An \textbf{\textit{iterative expression}} is one that performs a computation over a series of values (typically the components of an array or list), yielding some result. Iterative expressions are rather more unusual, but they are a prominent feature of the functional language \texttt{HASKELL}, in the form of list comprehensions.

\subsection{Constant and Variable Accesses}

A \textbf{\textit{constant access}} is a reference to a named constant, and yields the value of that constant. A \textbf{\textit{variable access}} is a reference to a named variable, and yields the current value of that variable. (``\texttt{\#define}'' can be example of that in \texttt{C})

\subsection{Aggregates}

Used to construct composite values without any declaration/definition.

\begin{multicols}{3}

\begin{minted}
[
baselinestretch=1.2,
bgcolor=lightgray,
fontsize=\footnotesize,
]
{haskell}
x = (12 , "ali", True)
y = {name = "ali", no = 12}
f = \x -> x*x
l = [1 ,2 ,3 ,4]
\end{minted}

\vfill\null
\columnbreak

\begin{minted}
[
baselinestretch=1.2,
bgcolor=lightgray,
fontsize=\footnotesize,
]
{py}
x = (12 , "ali", True)
y = [1, 2, [2, 3], "a"]
f = {"name": "ali", "no": "12"}
l = lambda x:x+1
\end{minted}

\vfill\null
\columnbreak

\begin{minted}
[
baselinestretch=1.2,
bgcolor=lightgray,
fontsize=\footnotesize,
]
{c}
struct Person {
    char name[20]; 
    int no;
} p = {"Ali Cin", 332314};
\end{minted}
\end{multicols}

\subsection{Block Expressions}

Some languages allow multiple/statements in a block to calculate a value. GCC extension for compound statement expressions:
\begin{minted}
[
baselinestretch=1.2,
bgcolor=lightgray,
fontsize=\footnotesize,
]
{c}
double s, i, arr[10];
s = ( { double t = 0;
        for (i = 0; i < 10; i++)
            t += arr[i];
        t;} ) + 1;
\end{minted}
Value of the last expression is the value of the block.

\end{document}


\newpage
\section{Chapter 3}

\begin{multicols}{2}
\setlength{\columnsep}{1.5cm}
\setlength{\columnseprule}{0.2pt}

\subsection{Context Free Grammars}

\begin{definition}{}
  A \textbf{context-free grammar} $G$ is a quadruple $(V, \Sigma, R, S)$ where
  \begin{itemize}
    \item $V$ is an alphabet
    \item $\Sigma$ (the set of \textbf{terminals}) is a subset of $V$
    \item $R$ (the set of \textbf{rules}) is a finite subset of $(V - \Sigma) \times V^*$
    \item $S$ (the \textbf{start symbol}) is an element of $V - \Sigma$
  \end{itemize}
\end{definition}

The members of $V - \Sigma$ are called \textbf{nonterminals}. For any $A \in V - \Sigma$ and $u \in V^*$, we write $A \rightarrow_G u$ whenever $(A, u) \in R$. 
  
For any strings $u, v \in V^*$, we write $u \Rightarrow_G v$ if and only if there are strings $x, y \in V^*$ and $A \in V - \Sigma$ such that $u = xAy$, $v = xv'y$, and $A \rightarrow_G v'$. 

The relation $\Rightarrow_G^*$ is the \textit{reflexive}, \textit{transitive closure} of $\Rightarrow_G$. Finally, $L(G)$, the \textbf{language generated} by $G$, is $\left\{ w \in \Sigma^* : S \Rightarrow_G^* w \right\}$; we also say that $G$ \textbf{generates} each string in $L(G)$. A language $L$ is said to be a \textbf{context-free language} if $L = L(G)$ for some context-free grammar $G$.

We call any sequence of the form
\begin{equation*}
  w_0 \Rightarrow_G w_1 \Rightarrow_G \cdots \Rightarrow_G w_n
\end{equation*}
a \textbf{derivation} in $G$ of $w_n$ from $w_0$. Here $w_o, \cdots, w_n$ may be any strings in $V^*$, and $n$, the \textbf{length} of the derivation, may be any natural number, including zero. We also say that the derivation has $n$ \textbf{steps}. 

\end{multicols}

\begin{formula}{}
  \begin{itemize}
    \item $u \Rightarrow v$: $u$ \textbf{\textit{directly yields}} $v$; $A \rightarrow w$: \textbf{\textit{(production) rule}}.
    \item $V$, alphabet, can include symbols such as start symbol, $S$, or $A$.
    \item (From example 3.1.4) The same string may have several derivations in a context-free grammar. Two derivations in this grammar are
    \begin{align*}
      &S \Rightarrow SS \Rightarrow S(S) \Rightarrow S((S)) \Rightarrow S(()) \Rightarrow ()(()) &\textnormal{ and } && S \Rightarrow SS \Rightarrow (S)S \Rightarrow ()S \Rightarrow ()(S) \Rightarrow ()(())
    \end{align*}
    \item Some context-free languages are not regular. However, all regular languages are context-free.
    \item Context-free languages are precisely the languages accepted by certain language acceptors called \textit{\textbf{pushdown automata}}.
  \end{itemize}
\end{formula}


% \begin{multicols}{2}
%   \setlength{\columnsep}{1.5cm}
%   \setlength{\columnseprule}{0.2pt}


% \end{multicols}



\newpage
\documentclass{article}
\usepackage[utf8]{inputenc}
\usepackage{geometry}
\usepackage{graphicx}
\usepackage{amsmath}
\usepackage{amsfonts}
\usepackage{amsthm}
\usepackage{amssymb}
\usepackage[most]{tcolorbox}
\usepackage{array}
\usepackage{latexsym}
\usepackage{alltt}
\usepackage{hyperref}
\usepackage{color, colortbl}
\usepackage{float}
\usepackage{pdfpages}
\usepackage{algpseudocode}
\usepackage{multicol}
\usepackage{multirow}
\usepackage{caption}
\usepackage{xparse}
\usepackage{setspace}
\usepackage{enumitem}
\usepackage{pdflscape}
% \usepackage{parskip}
\usepackage{blindtext}
\usepackage{forest}
\usepackage[newfloat]{minted}


\geometry
{
  a4paper,
  left=12mm,
  right=12mm,
  top=12mm,
  bottom=15mm,
}

% mybox
\newtcolorbox{mybox}[3][]
{
  colframe = #2!25,
  colback  = #2!10,
  coltitle = #2!20!black,  
  title    = {#3},
  #1,
}

\definecolor{bg}{rgb}{0.95,0.95,0.95}
\setminted
{
	mathescape=true,
  escapeinside=@@,
	xleftmargin=\parindent,
	bgcolor=bg
}

\SetupFloatingEnvironment{listing}{name=Code}

\usetikzlibrary{patterns,positioning,fit,arrows,calc,shapes.geometric,shapes.multipart,decorations.pathreplacing}

\newenvironment*{dummyenv}{}{}

% New environments that use mybox
\newcounter{example}[section]
\newenvironment{example}[1]{\begin{mybox}[breakable]{green}{\refstepcounter{example}\textbf{Example \thesection.\theexample #1}}}{\end{mybox}}

\newcounter{definition}[section]
\newenvironment{definition}[1]{\refstepcounter{definition}\begin{mybox}[breakable]{blue}{\textbf{Definition \thesection.\thedefinition #1}}}{\end{mybox}}

\newcounter{theorem}[section]
\newenvironment{theorem}[1]{\begin{mybox}{red}{\refstepcounter{theorem}\textbf{Theorem \thesection.\thetheorem #1}}}{\end{mybox}}

\newenvironment{formula}[1]{\begin{mybox}{cyan}{\textbf{#1}}}{\end{mybox}}

% Changing maketitle
\makeatletter         
\renewcommand\maketitle{
{\raggedright % Note the extra {
\begin{center}
{\Large \bfseries \@title}\\[2ex] 
{\large \@author \ - \@date}\\[2ex]
\end{center}}} % Note the extra }
\makeatother

% \onehalfspacing % adjust spacing
\setlength{\parskip}{0.5\baselineskip}

% macros
\newcommand{\prob}[1]{\textbf{\textit{P}}\left\{#1\right\}}
\newcommand{\expc}[1]{\mathbf{E}\left(#1\right)}
\newcommand{\expcs}[1]{\mathbf{E}^2\left(#1\right)}
\newcommand{\var}[1]{\text{Var}\left( #1 \right)}
\newcommand{\ra}{\rightarrow}
\newcommand{\Ra}{\Rightarrow}

\def\circtxt#1{$\mathalpha \bigcirc \mkern-13mu \mathtt #1$}

\NewDocumentCommand{\dsum}{%
    e{^_}
}{%
  {% 
    \displaystyle\sum
    \IfValueT{#1}{^{#1}}
    \IfValueT{#2}{_{#2}}
  }
}%

% maketitle variables
\title{CENG 242 - Chapter 4: Binding and Scope}
\author{Burak Metehan Tunçel}
\date{May 2022}

\begin{document}

\maketitle

\begin{multicols*}{2}
\setlength{\columnsep}{1.5cm}
\setlength{\columnseprule}{0.2pt}

\section{Bindings and Environment}
\label{sec:bind-env}

The most important feature of high level languages: programmers able to give names to program entities (variable, constant, function, type, ...). These names are called \textit{\textbf{identifiers}}. They are declared once, used $n$ times.

A \textit{\textbf{binding}} is a fixed association between an identifier and an entity such as a value, variable, or procedure. A declaration produces one or more bindings. For binding:
\begin{itemize}
  \item Scope of identifiers should be known: What is the block structure?, Which blocks the identifier is available?
  \item What will happen if we use same identifier name again ``\texttt{C} forbids reuse of same identifier name in the same scope. Same name can be used in different nested blocks. The identifier inside hides the outside identifier''.
\end{itemize}

\setlength{\columnsep}{0.2cm}
\setlength{\columnseprule}{0pt}
\begin{multicols*}{2}
\begin{listing}[H]
\begin{minted}{cpp}
  double f, y;
  int f() { // Error
    ...
  }
  double y; // Error
\end{minted}
\caption{}
\label{code:code1}
\end{listing}

\columnbreak

\begin{listing}[H]
\begin{minted}{cpp}
  double y;
  int f() {
    double f; // OK
    int y; // OK 
  }
\end{minted}
\caption{}
\label{code:code2}
\end{listing}
\end{multicols*}
\setlength{\columnseprule}{0.2pt}
\setlength{\columnsep}{1.5cm}

\textit{An \textbf{environment} (or name space) is a set of bindings occurrences that are accessible at a point in the program}. Each expression or command is interpreted in a particular environment, and all identifiers used in the expression or command must have bindings in that environment. It is possible that expressions and commands in different parts of the program will be interpreted in different environments.

\begin{listing}[H]
\begin{minted}{c}
struct Person { ... } x;
int f(int a) { 
  double y;
  int x;
  ... @\circtxt{1}@
}

int main() {
   double a;
   ... @\circtxt{2}@
}
\end{minted}
\caption{}
\label{code:code3}
\end{listing}

\noindent O(\circtxt{1})=\{struct Person $\mapsto$ type, x $\mapsto$ int, f $\mapsto$ func, a $\mapsto$ int, y $\mapsto$ double\}

\noindent O(\circtxt{2})=\{struct Person $\mapsto$ type, x $\mapsto$ struct Person, f $\mapsto$ func, a $\mapsto$ double, main $\mapsto$ func\}

Usually at most one binding per identifier is allowed in any environment. An environment is then a partial mapping from identifiers to entities.

A \textit{\textbf{bindable}} entity is one that may be bound to an identifier. Programming languages vary in the kinds of entity that are bindable:
\begin{itemize}
  \item \texttt{C}'s bindable entities are types, variables, and function procedures.
  \item \texttt{JAVA}'s bindable entities are values, local variables, instance and class variables, methods, classes, and packages.
\end{itemize}


\section{Scope}
\label{sec:scope}

The \textit{\textbf{scope}} of a declaration is the portion of the program text over which the declaration is effective. Similarly, the \textit{\textbf{scope}} of a binding is the portion of the program text over which the binding applies.

In some early programming languages, the scope of each declaration was the whole program. In modern languages, the scope of each declaration is influenced by the program’s syntactic structure, in particular the arrangement of blocks.

\subsection{Block Structure}

A \textit{\textbf{block}} defines the scope of the identifiers declared
inside (boundary of the definition validity). For variables, they
also define the lifetime. Each programming language has its own forms of blocks:
\begin{itemize}
  \item The blocks of a \texttt{C} program are block commands (\{ ... \}), function bodies, compilation units (source files), and the program as a whole.
  \item The blocks of a \texttt{JAVA} program are block commands ({ ... }), method bodies, class declarations, packages, and the program as a whole.
  \item The block of a \texttt{Haskell} program are `\texttt{let \textit{definitions} in \textit{expression}}' defines a block expression. Also `\textit{expression} \texttt{where} \textit{definitions}' defines a block expression. (the definitions have a local scope and not accessible outside of the expression).
\end{itemize}

Block structure of the language is defined by the organization of the blocks.

\newpage

\subsubsection{Monolithic Block Structure}

In a language with \textit{\textbf{monolithic block structure}}, the only block is the whole program, so the scope of every declaration is the whole program. In other words, all declarations are global.
\begin{figure}[H]
  \centering
  \includegraphics[width=\linewidth]{img/fig-4.1.png}
  \caption{Monolithic block structure.}
  \label{fig:fig1}
\end{figure}
In a long program with many identifiers, they share the same scope and they need to be distinct.

\subsubsection{Flat Block Structure}

In a language with \textit{\textbf{flat block structure}}, the program is partitioned into several non-overlapping blocks. In other words, program contains \textit{the global scope and only a single level local scope of function definitions}. No further nesting is possible.

\begin{figure}[H]
  \centering
  \includegraphics[width=\linewidth]{img/fig-4.2.png}
  \caption{Flat block structure.}
  \label{fig:fig2}
\end{figure}

\subsubsection{Nested Block Structure}

In a language with \textit{\textbf{nested block structure}}, blocks may be nested within other blocks.

\begin{figure}[H]
  \centering
  \includegraphics[width=\linewidth]{img/fig-4.3.png}
  \caption{Nested block structure.}
  \label{fig:fig3}
\end{figure}


\subsection{Scope and Visibility}

Consider all the occurrences of identifiers in a program. We must distinguish two different kinds of identifier occurrences:
\begin{itemize}
  \item A \textit{\textbf{binding occurrence}} of identifier \textit{I} is an occurrence where \textit{I} is bound to some entity \textit{X}.
  \item An \textit{\textbf{applied occurrence}} of \textit{I} is an occurrence where use is made of the entity \textit{X} to which \textit{I} has been bound. At each such applied occurrence we say that \textit{I} denotes \textit{X}.
\end{itemize}

When a program contains more than one block, it is possible for the same
identifier \textit{I} to be declared in different blocks. In general, \textit{I} will denote a different entity in each block.

Consider two nested blocks, such that the inner block lies within the scope of a declaration of identifier \textit{I} in the outer block:
\begin{itemize}
  \item If the inner block \textit{does not} contain a declaration of \textit{I}, then applied occurrences of \textit{I} both inside and outside the inner block correspond to the same declaration of \textit{I}. The declaration of \textit{I} is then said to be \textit{\textbf{visible}} throughout the outer and inner blocks.
  \item If the inner block \textit{does} contain a declaration of \textit{I}, then all applied occurrences of \textit{I} inside the inner block correspond to the inner, not the outer, declaration of \textit{I}. The outer declaration of \textit{I} is then said to be \textit{\textbf{hidden}} by the inner declaration of \textit{I}.
\end{itemize}

\begin{listing}[H]

\begin{minted}{c}
int x, y;

int f(double x) {
  ...             // parameter x hides global x
                  // in f()
}

int g(double a) {
  int y;         // local y hides global y in g()
  double f;      // local f hides global f()
                 // in g()
  ...
}

int main() {
  int y;        // local y hides global y
                // in main()
}
\end{minted}
\caption{}
\label{code:code4}
\end{listing}

\newpage

\end{multicols*}

\begin{multicols}{2}
\setlength{\columnsep}{1.5cm}
\setlength{\columnseprule}{0.2pt}

\subsection{Static vs Dynamic Scoping}

When are the binding and scope resolution done? In compile time or run
time? Two options:
\begin{enumerate}
  \item Static binding, static scope
  \item Dynamic binding, dynamic scope  
\end{enumerate}

The first defines scope and binding based on the lexical structure of the program and binding is done \textit{at compile time}. Second activates the definitions in a block during the execution of the block. The environment changes dynamically \textit{at run time} as functions are called and returned.

\vspace*{\fill}
\columnbreak

\subsubsection{Static Binding}

Programs' shape is significant. Environment is based on the position in the source (lexical scope). Most languages apply static binding (\texttt{C, Haskell, Pascal, Java, ...})

\begin{dummyenv}
\def\X{\color{green!60!black}\bf x}
\def\gX{\color{red!60!black}\bf x}
\def\Y{\color{green!60!black}\bf y}
\def\mY{\color{red!60!black}\bf y}
\def\fY{\color{yellow!60!black}\bf y}
\def\gA{\color{yellow!60!black}\bf a}
\def\mA{\color{red!60!black}\bf a}
\lstset{language=C,
        basicstyle=\footnotesize\ttfamily,
        keywordstyle=\color{blue!50!black}\bfseries,
        identifierstyle=\color{blue!60!green}\sffamily,
        stringstyle=\color{red!70!green}\ttfamily,
	      commentstyle=\color{blue!30!white}\itshape,
        showstringspaces=true}
\begin{listing}[H]
\begin{lstlisting}[language={C},escapechar=\#]
int #\X#=1,#\Y#=2;
int f(int #\fY#) {
      #\fY#=#\X#+#\fY#;     /* #\X# global, #\fY# local */
      return #\X#+#\fY#;
}
int g(int #\gA#) {
    int #\gX#=3;      /* #\gX# local, #\Y# global */
    #\Y#=#\gX#+#\gX#+#\gA#;    #\gX#=#\gX#+#\Y#;    #\Y#=f(#\gX#);
    return #\gX#;
}
int main() {
    int #\mY#=0;      /* #\X# global #\mY# local */
    int #\mA#=10;
    #\X#=#\mA#+#\mY#;    #\mY#=#\X#+#\mA#;    #\mA#=f(#\mA#);    #\mA#=g(#\mA#);
    return 0;
}
\end{lstlisting}
\caption{}
\label{code:code5}
\end{listing}
\end{dummyenv}

\end{multicols}

\subsubsection{Dynamic Binding}

Functions called update their declarations on the environment at run-time. Delete them on return. Current stack of activated blocks is significant in binding. \texttt{Lisp} and some script languages apply dynamic binding.

\begin{dummyenv}

% \noindent
\begin{minipage}[c]{0.35\textwidth}
%%%% Column 1 %%%%

\begin{listing}[H]

\begin{minted}[linenos]{c}
int x = 1, y = 2;

int f(int y) {
  y = x + y;      
  return x + y;
}

int g(int a) {
  int x = 3;       
  y = x + x + a; x = x + y;    
  y = f(x);
  return x;
}

int main() {
  int y = 0;  int a = 10;  
  x = a + y;  y = x + a;    
  a = f(a);   a = g(a);
  return 0;
}
\end{minted}
\caption{}
\label{code:code6}
\end{listing}

\end{minipage} %
\begin{minipage}[c]{0.65\textwidth}
%%%% Column 2 %%%%
\begin{dummyenv}
\def\T{\rule{0pt}{1em}\hspace*{1em}}
\noindent\normalsize\begin{tabular}{rll}
& Trace & Environment (without functions)\\ \hline
& initial & \{x:GL, y:GL \} \\ \rowcolor{blue!5}
12& call main & \{x:GL, y:main, a:main \} \\ \rowcolor{blue!15}
15&\T call f(10)  & \{x:GL, y:f , a:main \} \\ \rowcolor{blue!15}
4 &\T return f : 30 & back to environment before f  \\ \rowcolor{blue!5}
15& in main & \{x:GL, y:main, a:main \} \\ \rowcolor{blue!10}
15&\T call g(30) & \{x:g, y:main, a:g  \} \\ \rowcolor{blue!25}
9&\T\T call f(39) & \{x:g, y:f, a:g  \} \\ \rowcolor{blue!25}
4&\T\T return f : 117 & back to environment before f\\ \rowcolor{blue!10}
9&\T in g  & \{x:g, y:main, a:g  \} \\ \rowcolor{blue!10}
10&\T return g : 39 & back to environment before g \\ \rowcolor{blue!5}
15& in main & \{x:GL, y:main, a:main\} \\ \rowcolor{blue!5}
16& return main & x:GL=10, y:GL=2, y:main=117, a:main=39 \\
\end{tabular}
\end{dummyenv}

\end{minipage}

\end{dummyenv}

\begin{multicols*}{2}
\setlength{\columnsep}{1.5cm}
\setlength{\columnseprule}{0.2pt}


\end{multicols*}

\section{Binding Process}
\label{sec:binding-process}

Language processor keeps track of current environment in a data structure called \textit{\textbf{Symbol Table}} or \textit{\textbf{Identifier Table}}. Symbol table maps identifier strings to their type and binding. Each new block introduces its declarations/bindings to the symbol table and on exit, they are cleared. Usually implemented as a \textit{Hash Table}.

For static binding, Symbol Table is a compile time data structure and maintained during different stages of compilation. For dynamic binding, symbol table is maintained at run time.

\newpage

\begin{multicols*}{2}
\setlength{\columnsep}{1.5cm}
\setlength{\columnseprule}{0.2pt}


\section{Declaration}
\label{sec:declaration}

\textit{\textbf{Definition:}} Creating a new name for an existing binding.

\noindent \textit{\textbf{Declaration:}} Creating a completely new binding.

\begin{itemize}
  \item in \texttt{C}: \texttt{struct Person} is a declaration. \texttt{typedef struct Person persontype} is a definition.
  \item in \texttt{C++}: \texttt{double x} is a declaration. \texttt{double \&y=x;} is a definition.
\end{itemize}

The basic distinction is whether creating a new entity or not. However, usually the distinction is not clear and used interchangeably.

\subsection{Type Declaration}

A \textit{\textbf{type declaration}} binds an identifier to a type. We can distinguish two kinds of type declaration. A \textit{\textbf{type definition}} binds an identifier to an existing type. A \textit{\textbf{new-type declaration}} binds an identifier to a new type that is not equivalent to any existing type.

\subsection{Constant Declaration}

A \textit{\textbf{constant declaration}} binds an identifier to a constant value. A constant declaration typically has the form ``\texttt{const \textit{I} = \textit{E};}''.

\subsection{Variable Declaration}

A \textit{\textbf{variable declaration}}, in its simplest form, creates a single variable and binds an identifier to that variable. A \textit{\textbf{variable renaming definition}} binds an identifier to an \textit{existing} variable. In other words, it creates an \textit{alias}.

\subsection{Procedure Definitions}

A \textit{\textbf{procedure definition}} binds an identifier to a procedure. In most programming languages, we can bind an identifier to either a function procedure or a proper procedure.

\textit{Note: The following subsections are important.}

\subsection{Sequential Declarations}

A \textit{\textbf{sequential declaration}} composes subdeclarations that are to be elaborated one after another. Each subdeclaration can use bindings produced by any \textit{previous} subdeclarations, but not those produced by any \textit{following} subdeclarations.

Declared identifier is not available in preceding declarations but is available in following declaration.

Most programming languages provide only such declarations.

\subsection{Collateral Declarations}

A \textit{\textbf{collateral declaration}} composes subdeclarations that are to be elaborated independently of each other. These subdeclarations may not use bindings produced by each other. The collateral declaration merges the bindings produced by its subdeclarations.

Collateral declarations are uncommon in imperative and object-oriented languages, but they are common in functional and logic languages.

\subsection{Recursive Declarations}

A \textit{\textbf{recursive declaration}} is one that uses the bindings that it produces itself. Such a construct is important because it enables us to define recursive types and procedures.

\textit{Declaration:} \texttt{Name = Body}. The body of the declaration can access the declared identifier. Declaration is available in the body of itself.

\texttt{C} functions and type declarations are recursive. Variable definitions are usually not recursive.

\subsection{Recursive Collateral Declarations}

All declarations can access the others regardless of their order.
\begin{itemize}
  \item All \texttt{Haskell} declarations are recursive collateral (including variables). All declarations are mutually recursive.
  \item \texttt{C++} class members are like this.
  \item In \texttt{C}, a similar functionality can be achieved by prototype
  definition.
\end{itemize}

\subsection{Scopes of Declarations}

\textit{Collateral}, \textit{sequential}, and \textit{recursive declarations} differ in their influence on scope:
\begin{itemize}
  \item In a \textit{collateral declaration}, the scope of each subdeclaration extends from the end of the collateral declaration to the end of the enclosing block.
  \item In a \textit{sequential declaration}, the scope of each subdeclaration extends from the end of that subdeclaration to the end of the enclosing block.
  \item In a \textit{recursive declaration}, the scope of every subdeclaration extends from the start of the recursive declaration to the end of the enclosing block.
\end{itemize}


\newpage
\section{Blocks}
\label{sec:blocks}

If we allow a command to contain a local declaration, we have a \textit{block command}. If we allow an expression to contain a local declaration, we have a\textit{block expression}.

\subsection{Block Commands}

A block command is a form of command that contains a local declaration (or group of declarations) \textit{D} and a subcommand \textit{C}. The bindings produced by \textit{D} are used only for executing \textit{C}.

In other words, declarations done inside a block command is available only during the block. Statements inside work in this environment. The declarations lost outside of the block.

\begin{listing}[H]
\begin{minted}{c}
int x = 3, i = 2;
x += i;
while (x > i) {
  int i = 0;
  ...
  i++;
}
/* i is 2 again */
\end{minted}
\caption{}
\label{code:code7}
\end{listing}


\subsection{Block Expressions}

A \textit{\textbf{block expression}} is a form of expression that contains a local declaration (or group of declarations) \textit{D} and a subexpression \textit{C}. The bindings produced by \textit{D} are used only for evaluating \textit{E}.

In other words, it allows an expression to be evaluated in a special local environment. Declarations done in the block is not available outside.

\begin{listing}[H]
\begin{minted}{Haskell}
x=5
t=let xsquare=x*x
      factorial n = if n<2 then 1
                    else n*factorial (n-1)
      xfact = factorial x
  in  (xsquare+1)*xfact/(xfact*xsquare+2)
\end{minted}
\caption{}
\label{code:code8}
\end{listing}

\vspace*{\fill}
\columnbreak

\noindent Hiding works in block expressions as expected:
\begin{listing}[H]

\begin{minted}{Haskell}
x=5 ; y=6 ; z = 3
t=let x=1 
  in let y=2 
     in   x+y+z
{--
t is 1+2+3 here. 
local x and y hides the ones above
--}
\end{minted}
\caption{}
\label{code:code9}
\end{listing}

\noindent GCC (only GCC) block expressions has the last expression in block as the value:
\begin{listing}[H]

\begin{minted}{c}
double min ;
...
min =  ({ double tmp;
          if (b < a) then {
            tmp = a;  a = b ; b = tmp;
          }
          a; // this is the value of the block
        });
\end{minted}
\caption{}
\label{code:code10}
\end{listing}

\subsection{Block Declaration}

A declaration is made in a local environment of declarations. Local declarations are not made available to the outer environment.

In \texttt{Haskell}: \texttt{D$_{exp}$ where D$_1$; D$_2$;  ... ; D$_n$}

\noindent Only \texttt{D$_{exp}$} is added to environment. Body of \texttt{D$_{exp}$} has all local declarations available in its environment.

\begin{listing}[H]

\begin{minted}{Haskell}
fifthpower x = (forthpowerx) * x where
               squarex = x*x;
               forthpowerx = squarex*squarex
\end{minted}
\caption{}
\label{code:code11}
\end{listing}





\end{multicols*}

\end{document}


\section{Chapter8: Introdcution to Statistics}

\begin{formula}{NOTATION}
    \begin{align*}
        \mu &= \textnormal{population mean}\\
        \bar{X} &= \textnormal{Sample mean, estimator of $\mu$}\\
        &\\
        \sigma &= \textnormal{population standard deviation}\\
        s &= \textnormal{sample standard deviation, estimator of $\sigma$}\\
        &\\
        \sigma^2 &= \textnormal{population variance}\\
        s^2 &= \textnormal{sample variance, estimator of $\sigma$} \\
        &\\
        \sigma(\hat{\theta}) &= \textnormal{standard error of estimator $\hat{\theta}$ of parameter $\theta$}\\
        s(\hat{\theta}) &= \textnormal{estimated standard error = $\hat{\sigma}(\hat{\theta})$}
    \end{align*}
\end{formula}

\subsection{Mean}

\paragraph{\textit{Population Mean:}} $\mu = \expc{X}$

\paragraph{\textit{Sample Mean}}
\begin{equation*}
    \bar{X} = \frac{X_1 + \cdots + X_n}{n}
\end{equation*}

\subsubsection{Unbiasedness}

\paragraph{Unbiased}
\begin{equation*}
    \expc{\hat{\theta}} = \theta
\end{equation*}

Bias of $\hat{\theta}$ is defined as Bias($\hat{\theta}$) = $\expc{\hat{\theta} - \theta}$.

\subsection{Variance and Standard Deviation}

\paragraph{Sample Variance}
\begin{equation*}
    s^2 = \frac{1}{n - 1} \sum_{i = 1}^{n} (X_i - \bar{X})^2
\end{equation*}
Another formula
\begin{equation*}
    s^2 = \frac{\dsum_{i=1}^n X_i^2 - n\bar{X}^2}{n-1}
\end{equation*}

\paragraph{Sample Standard Deviation}
\begin{equation*}
    s = \sqrt{s^2}
\end{equation*}

\paragraph{Variance of $\bar{X}$}
\begin{equation*}
    \var{\bar{X}} = \frac{\sigma^2}{n}
\end{equation*}

\paragraph{Standard Deviation of $\bar{X}$}
\begin{equation*}
    \sigma(\bar{X}) = \sqrt{\var{\bar{X}}} = \sqrt{\frac{\sigma^2}{n}} = \frac{\sigma}{\sqrt{n}}
\end{equation*}

\subsection{Standard Errors of Estimates}

\textbf{Standard error} of an estimator $\hat{\theta}$ is its standard deviation, $\sigma(\hat{\theta}) = \textnormal{Std}(\hat{\theta})$


\end{document}
