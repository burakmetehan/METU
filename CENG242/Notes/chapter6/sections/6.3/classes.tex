\subsection{Classes}
\label{subsec:classes}

A \textit{\textbf{class}} is a set of similar objects. All the objects of a given class have the same variable components, and are equipped with the same operations.

Classes are supported by all the object-oriented languages including \texttt{C++} and \texttt{JAVA}. In object-oriented terminology, an object's variable components are variously called \textit{instance variables} or \textit{member variables}, and its operations are usually called \textit{constructors} and \textit{methods}.

A \textit{\textbf{constructor}} is an operation that creates (and typically initializes) a new object of the class. In both \texttt{C++} and \texttt{JAVA}, a constructor is always named after the class to which it belongs.

A \textit{\textbf{method}} is an operation that inspects and/or updates an existing object of the class.

\noindent Method call,
\mint{cpp}{O.M(E_1, @$\ldots$@, E_n)}
where
\begin{itemize}
  \item $O$ identifies the target object.
  \item $M$ is the name of method. If there is not method named $M$, it causes an error.
  \item $E_i$ are evaluated to yield the arguments.
\end{itemize}
Inside the method's body, \textbf{\texttt{this}} denotes the target object.

An \textit{object} is an \textit{\textbf{instance}} of the class that it belongs to (a counter type instead of a single counter). Classes have similar purposes to abstract data types.
\begin{listing}[H]

\begin{minted}{cpp}
class Counter {
private:   int counter;
public:    Counter() { counter=0; }
           int get() { return counter;}
           void increment() { counter++; }
} men,vehicles;
men.increment(); vehicles.increment();
men.get(); vehicles.get();
\end{minted}
\caption{C++ class declaration}
\label{code:code3}
\end{listing}

\begin{table}[H]
  \centering
  \textcolor{blue}{\large Abstract data type}\\
  \begin{tabular}{
    !{\color{red!70!black}\vrule width 1pt}p{7cm}
    !{\color{red!70!black}\vrule width 1pt}} \colorfline{green!70!black}
    interface {\footnotesize (constructor, functions)} \\ \hline \\
    detail {\footnotesize (\textcolor{red}{data type definition}, auxiliary functions)}  \colorline{red!70!black}
  \end{tabular}
\end{table}


\begin{table}[H]
  \centering
  \textcolor{blue}{\large Object}\\
  \begin{tabular}{
    !{\color{red!70!black}\vrule width 1pt}p{7cm}
    !{\color{red!70!black}\vrule width 1pt}} \colorfline{green!70!black}
    interface {\footnotesize (constructor, functions)}\\ \hline \\
    detail {\footnotesize (\textcolor{red}{variables}, auxiliary functions)}  \colorline{red!70!black}
  \end{tabular}
\end{table}

\begin{formula}{Purpose}
\begin{itemize}
  \item preserving data integrity, 
  \item abstraction, 
  \item re-usable codes.
\end{itemize}
\end{formula}
