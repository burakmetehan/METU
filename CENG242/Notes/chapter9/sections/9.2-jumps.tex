\section{Jumps}
\label{sec:jumps}

A \textit{\textbf{jump}} is a sequencer that transfers control to a specified program point. A jump typically has the form ``\texttt{goto $L$;}'', and this transfers control directly to the program point denoted by $L$, which is a \textit{\textbf{label}}.

\vspace*{\fill}
\columnbreak

\begin{formula}{}
\begin{listing}[H]

\begin{minted}[bgcolor={}]{c}
@\R{l1}{}@L1:  x++;
      if (x>10) goto L2@\R{gl2}{}@;
      j++;
      for (i=0;i<j;j++) {
          x=x*2;
@\R{l2}{}@L2:      if (x>1000) goto L3@\R{gl3}{}@;
          else goto L1@\R{gl1}{}@;
      }
@\R{l3}{}@L3:  printf("out\n");
\end{minted}

\begin{tikzpicture} [thick,remember picture,overlay,rounded corners=3pt]
  \draw [green,->] (gl1) -- +(0,3) -| (l1);
  \draw [violet,->] (gl2) -- +(0,-0.55) -|  (l2);
  \draw [red,->] (gl3) -- +(0,-1.7) -| (l3);
\end{tikzpicture}

\caption{}
\label{code:code1}

\end{listing}
\end{formula}

Unrestricted jumps allow any command to have multiple entries and multiple
exits. They tend to give rise to ``spaghetti'' code, so called because its flowchart is tangled. Also, ``spaghetti'' code tend to be hard to understand and is prone to errors.
\begin{listing}[H]

\begin{minted}{c}
L1: ...
  goto L2;        @\color{violet}\circtxt{1}@
  ...
  for (i=0;i<10;i++) {
    int x=t;
L2: ...
    goto L1;    @\color{violet}\circtxt{2}@
    ...
	  goto L2:    @\color{violet}\circtxt{3}@
  }
\end{minted}

\caption{}
\label{code:code2}
\end{listing}

Also, when we consider the jumps, lifetime and values of local variables and values of index variables become a problem.

In \texttt{C}, labels are local to enclosing block. No jumps allowed into the block or out the block. Newer languages avoid jumps.

Jump is avoided but single entrance multiple exit is still desirable. It can be satisfied by \textbf{escapes}.
