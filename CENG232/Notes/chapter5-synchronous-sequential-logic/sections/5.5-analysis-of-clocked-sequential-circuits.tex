\section{Analysis of Clocked Sequential Circuits}
\label{sec:analysis-clocked-seq-circ}

Analysis describes what a given circuit will do under certain operating conditions. The behavior of a clocked sequential circuit is determined from the inputs, the outputs, and the state of its flip-flops. The outputs and the next state are both a function of the inputs and the present state.

\subsection{State Equations}
\label{subsec:state-equations}

The behavior of a clocked sequential circuit can be described algebraically by means of state equations. A \textit{state equation} (also called a \textit{transition equation}) specifies the next state as a function of the present state and inputs. Consider the sequential circuit shown in Fig. 15
\begin{figure}[H]
  \centering
  \includegraphics[width=\linewidth]{img/fig-5.15.png}
  \caption{Example of sequential circuit}
  \label{fig:5.15}
\end{figure}
It consists of two $D$ flip-flops $A$ and $B$, an input $x$ and an output $y$. Since the $D$ input of a flip-flop determines the value of the next state (i.e., the state reached after the clock transition), it is possible to write a set of state equations directly from the logic diagram in Fig. 15.
\begin{align*}
  A(t + 1) &= A(t)x(t) + B(t)x(t)\\
  B(t + 1) &= A'(t)x(t)
\end{align*}
We can omit the designation ($t$) after each variable for convenience and can express the state equations in the more compact form
\begin{align*}
  A(t + 1) &= Ax + Bx\\
  B(t + 1) &= A'x
\end{align*}
The Boolean expressions for the state equations can be derived directly from the gates that form the combinational circuit part of the sequential circuit, since the $D$ values of the combinational circuit determine the next state. Similarly, the present-state value of the output can be expressed algebraically as
\begin{equation*}
  y(t) = [A(t) + B(t)]x'(t)
\end{equation*}
By removing the symbol ($t$) for the present state, we obtain the output Boolean equation:
\begin{equation*}
  y = (A + B)x'
\end{equation*}

\subsection{State Table}
\label{subsec:state-table}

The time sequence of inputs, outputs, and flip-flop states can be enumerated in a \textit{state table} (sometimes called a \textit{transition table}). The state table for the circuit of Fig. 15 is shown in Table 5.2. The table consists of four sections labeled \textit{present state}, \textit{input}, \textit{next state}, and \textit{output}.
\begin{figure}[H]
  \centering
  \includegraphics[width=\linewidth]{img/table-5.2.png}
  \label{table:5.2}
\end{figure}
\noindent The derivation of a state table requires listing all possible binary combinations of present states and inputs.

In general, a sequential circuit with $m$ flip-flops and $n$ inputs needs $2^{m + n}$ rows in the state table. The binary numbers from 0 through $2^{m + n - 1}$ are listed under the present state and input columns.

It is sometimes convenient to express the state table in a slightly different form having only three sections: \textit{present state}, \textit{next state}, and \textit{output}. The input conditions are enumerated under the next-state and output sections. The state table of Table 5.2 is  repeated in Table 5.3 in this second form.
\begin{figure}[H]
  \centering
  \includegraphics[width=\linewidth]{img/table-5.3.png}
  \label{table:5.3}
\end{figure}

\subsection{State Diagram}
\label{subsec:state-diagram}

The information available in a state table can be represented graphically in the form of a state diagram. In this type of diagram, a state is represented by a circle, and the (clock-triggered) transitions between states are indicated by directed lines connecting the circles.

The state diagram of the sequential circuit of Fig. 15 is shown in Fig. 16. The state diagram provides the same information as the state table and is obtained directly from Table 5.2 or Table 5.3. The binary number inside each circle identifies the state of the flip-flops. The directed lines are labeled with two binary numbers separated by a slash. The input value during the present state is labeled first, and the number after the slash gives the output during the \textit{present} state with the given input.
\begin{figure}[H]
  \centering
  \includegraphics[width=.5\linewidth]{img/fig-5.16.png}
  \caption{State diagram of the circuit of Fig. 15}
  \label{fig:5.16}
\end{figure}

The steps presented in this example are summarized below:
\begin{center}
  Circuit diagram $\ra$ Equations $\ra$ State table $\ra$ State diagram
\end{center}

To analyze sequential circuits:
\begin{itemize}
  \item Find Boolean expressions for the outputs of the circuit and the flip-flop inputs.
  \item Use these expressions to fill in the output and flip-flop input columns in the state table.
  \item Finally, use the characteristic equation or characteristic table of the flip-flop to fill in the next state columns.
\end{itemize}
The result of sequential circuit analysis is a state table or a state diagram describing the circuit.

\textit{\textbf{Note:} There are some examples in book that are realted to analysis. Examine them carefully.}

\subsection{Mealy and Moore Models of Finite State Machine}
\label{subsec:mealy-and-moore-models}

\begin{figure}[H]
  \centering
  \includegraphics[width=.5\linewidth]{img/fig-5.21.png}
  \caption{}
  \label{fig:5.21}
\end{figure}

In a Moore model, the outputs of the sequential circuit are synchronized with the  clock, because they depend only on flip-flop outputs, which are synchronized with the clock.

The output of the Mealy machine is the value that is present immediately before the active edge of the clock.

\textbf{Notes:}
\begin{itemize}
  \item The difference between a Mealy and Moore state machine is that ``\textit{the output of a Moore state machine depends on only the state of the machine; the output of a Mealy machine depends on the present state and the inputs to the machine.}
  \item \textit{The edge of a state machine chart represents a transition of the machine between two states.}
  \item \textit{A transition between the states of a finite state machine occurs at the active edge of the synchronizing signal (clock).}
  \item \textit{A finite state machine may have synchronous or asynchronous reset.}
  \item The reason why it is an important practice to implement a reset  signal in a finite state machine is that ``\textit{ Without a reset signal a finite state machine cannot be driven into a known initial state.}''.
  \item \textit{The outputs of a Mealy state machine may depend on the inputs to the machine.}
\end{itemize}
