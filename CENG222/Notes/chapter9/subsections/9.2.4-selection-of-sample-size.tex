\subsection{Selection of a Sample Size}
\label{subsec:selection-of-a-sample-size}

Formula (3) describes a confidence interval as
\begin{center}
  center $\pm$ margin
\end{center}
\noindent where
\begin{align*}
  \textnormal{center} &= \hat{\theta}\\
  \textnormal{margin} &= z_{\alpha / 2} \cdot \sigma(\hat{\theta})
\end{align*}

We can revert the problem and ask a very practical question: \textit{How large a sample should be collected to provide a certain desired precision of our estimator}?

To answer this question, we only need to solve the inequality
\begin{equation}
  \textnormal{margin} \leq \Delta
\end{equation}
in terms of $n$. Typically, parameters are estimated more accurately based on larger samples,
so that the standard error $\sigma(\hat{\theta})$ and the margin are decreasing functions of sample size $n$. Then, (7) must be satisfied for sufficiently large $n$.
