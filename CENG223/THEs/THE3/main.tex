\documentclass[11pt]{article}
\usepackage[utf8]{inputenc}
\usepackage[dvips]{graphicx}
\usepackage{fancybox}
\usepackage{verbatim}
\usepackage{array}
\usepackage{latexsym}
\usepackage{alltt}
\usepackage{hyperref}
\usepackage{textcomp}
\usepackage{color}
\usepackage{amsmath}
\usepackage{amsfonts}
\usepackage{tikz}
\usepackage{float}
%\usepackage{tcolorbox}
\usepackage[most]{tcolorbox}
\usepackage[hmargin=3cm,vmargin=5.0cm]{geometry}
%\topmargin=0cm
\topmargin=-2cm
\addtolength{\textheight}{6.5cm}
\addtolength{\textwidth}{2.0cm}
%\setlength{\leftmargin}{-5cm}
\setlength{\oddsidemargin}{0.0cm}
\setlength{\evensidemargin}{0.0cm}


\begin{document}

\section*{Student Information } 
%Write your full name and id number between the colon and newline
%Put one empty space character after colon and before newline
Full Name :  Burak Metehan Tunçel \\
Id Number :  2468726 \\


\section*{Q1}
Let $m \in Z^+-\{1\}$ and $f(x)$ is a identity function, which means that $f(x) = x$.

\begin{equation*}
    f(m-1) = f(m) - 1 = m - 1
\end{equation*}
which means $m - 1 \in Z^+$ and $m - 1 < m$ according \textbf{Well-Ordering Property} of the positive integers.

When we insert any number instead of $m$, we acquire another positive integer. However, when we insert $1$, $1 \in Z^+$, instead of m
\begin{equation*}
    f(1 - 1) = f(1) - 1 = 1 - 1 = 0
\end{equation*}
Here, $0 \notin Z^+$. Therefore $1$ is the smallest positive integer


\section*{Q2}
\subsubsection*{Proof of $S(m, 1)$}

For $n = 1$, the function $f(m, n)$ becomes $f(m, 1) = \frac{(1 + m - 1)!}{1! (m-1)!} = \frac{m!}{(m-1)!} = \frac{m (m-1)!}{(m-1)!} = m$

\begin{itemize}
    \item For $m = 1$, $S(1, 1): x_1 = 1$. There is only \textbf{1} solution
    \begin{equation*}
        \{(x_1 = 1)\}
    \end{equation*}
    Since $f(1, 1) = 1$, the function $f$ is valid for $m = 1$.
    \item For $m = 2$, $S(2, 1): x_1 + x_2 = 1$. There are \textbf{2} solutions
    \begin{equation*}
        \{(x_1 = 1, x_2 = 0), (x_1 = 0, x_2 = 1)\}
    \end{equation*}
    Since $f(2, 1) = 1$, the function $f$ is valid for $m = 2$.
    \begin{center}
        $\bullet$\\
        $\bullet$\\
        $\bullet$\\    
    \end{center}
    \item For $m$, $S(m, 1): x_1 + x_2 + \cdot \cdot \cdot + x_m = 1$. There are \textbf{m} solutions
    \begin{align*}
        &\{(x_1, x_2, ..., x_m) | x_i = 1,\ x_j = 0,\ 1 \leq i, j \leq m,\ (i, j \in Z^+),\ i \neq j\}\\
        &= \{(x_1 = 1, x_2 = 0, \cdot \cdot \cdot x_m = 0), (x_1 = 0, x_2 = 1, \cdot \cdot \cdot x_m = 0), \cdot \cdot \cdot, (x_1 = 0, x_2 = 0, \cdot \cdot \cdot x_m = 1)\}
    \end{align*}
    \item For $m$, $S(m+1, 1): x_1 + x_2 + \cdot \cdot \cdot + x_{m+1} = 1$. There are \textbf{m+1} solutions
    \begin{align*}
        &\{(x_1, x_2, ..., x_m) | x_i = 1,\ x_j = 0,\ 1 \leq i, j \leq m,\ (i, j \in Z^+),\ i \neq j\}\\
        &= \{(x_1 = 1, x_2 = 0, \cdot \cdot \cdot x_m = 0), (x_1 = 0, x_2 = 1, \cdot \cdot \cdot x_m = 0), \cdot \cdot \cdot, (x_1 = 0, x_2 = 0, \cdot \cdot \cdot x_{m+1} = 1)\}
    \end{align*}
\end{itemize}
So, it can be told that statement of $S$ is true.\\

\textbf{Basis Step:} $S(1, 1)$ is true and verified above.

\textbf{Inductive Step:} Assume that $S(m, 1)$ is true. show that $S(m, 1) \rightarrow S(m+1, 1)$ is true $\forall m$.\\ 

\textit{It is showed above that $S(m+1, 1)$ is true and verified. Therefore, By induction $S(m, 1)$ is proved.}


\subsubsection*{Proof of $S(1, n)$}

For $m = 1$, the function $f(m, n)$ becomes $f(1, n) = \frac{(n + 1 - 1)!}{n! (1-1)!} = \frac{n!}{(n)!} = 1$

\begin{itemize}
    \item For $n = 1$, $S(1, 1): x_1 = 1$. There is only \textbf{1} solution
    \begin{equation*}
        \{(x_1 = 1)\}
    \end{equation*}
    Since $f(1, 1) = 1$, the function $f$ is valid for $n = 1$.
    \item For $n = 2$, $S(1, 2): x_1 = 2$. There are \textbf{1} solutions
    \begin{equation*}
        \{(x_1 = 2)\}
    \end{equation*}
    Since $f(1, 2) = 1$, the function $f$ is valid for $n = 2$.
    \begin{center}
        $\bullet$\\
        $\bullet$\\
        $\bullet$\\    
    \end{center}
    \item For $n$, $S(1, n): x_1 = n$. There are \textbf{1} solutions
    \begin{align*}
        \{(x_1 = n)\}
    \end{align*}
    \item For $n$, $S(1, n): x_1 = n$. There are \textbf{1} solutions
    \begin{align*}
        \{(x_1 = n+1)\}
    \end{align*}
\end{itemize}

\textbf{Basis Step:} $S(1, 1)$ is true and verified above.

\textbf{Inductive Step:} Assume that $S(1, n)$ is true. Show that $S(1, n) \rightarrow S(1, n+1)$ is true $\forall n$.\\ 

\textit{It is showed above that $S(1, n+1)$ is true and verified. Therefore, By induction $S(1, n)$ is proved.}


\subsubsection*{Proof of $S(m+1, n)$}

\textbf{Basis Step:} $S(1, n)$ is showed that it is true.\\
\textbf{Inductive Step:} Assume that $[S(1, n) \land S(2, n) \land \cdot \cdot \cdot \land S(m, n)]$ is true. Show that $S(m+1, n)$ is true.

According to our assumption, the following equations are true.
\begin{align*}
    S(1, n) &= f(1, n) = 1\\
    S(2, n) &= f(2, n) = (n+1) f(1, n)\\
    S(3, n) &= f(3, n) = \frac{n+2}{2} f(2, n)\\
    &\cdot\\
    &\cdot\\
    &\cdot\\
    S(m, n) &= f(m, n) = \frac{n+m-1}{m-1} f(m-1, n)
\end{align*}

Since
\begin{align}
    S(m+1, n) &= f(m+1, n) = \frac{n+m}{m} f(m, n),
\end{align}$S(m+1, n)$ is also true by Strong Induction


\subsubsection*{Proof of $S(m, n+1)$}

\textbf{Basis Step:} $S(m, 1)$ is showed that it is true.\\
\textbf{Inductive Step:} Assume that $S(m, 1) \land S(m, 2) \land \cdot \cdot \cdot \land S(m, n)$ is true. Show that $S(m, n+1)$ is true.

According to our assumption, the following equations are true.
\begin{align*}
    S(m, 1) &= f(m, 1) = m\\
    S(m, 2) &= f(m, 2) = \frac{m+1}{2} f(m, 1)\\
    S(m, 3) &= f(m, 2) = \frac{m+2}{3} f(m, 2)\\
    &\cdot\\
    &\cdot\\
    &\cdot\\
    S(m, n) &= f(m, n) = \frac{m+n-1}{n} f(m, n-1)
\end{align*}

Since
\begin{align}
    S(m, n+1) &= f(m, n+1) = \frac{n+m}{n+1} f(m, n),
\end{align}$S(m, n+1)$ is also true by Strong Induction


\subsubsection*{Proof of $S(m+1, n+1)$}

\textbf{Basis Step:} $S(1, 1)$ is showed that it is true.\\
\textbf{Inductive Step:} Assume that $S(m+1, n) \land S(m, n+1)$ is true. Show that $S(m+1, n+1)$ is true.\\

According the Equation 1, 
\begin{equation}
    S(m + 1, n + 1) = \frac{n+m+1}{m} f(m, n+1) = \frac{n+m+1}{m} \frac{(n+m)!}{(n+1)! (m-1)!}
\end{equation}

According to Equation 2,
\begin{equation}
    S(m + 1, n + 1) = \frac{n+m+1}{n+1} f(m+1, n) = \frac{n+m+1}{n+1} \frac{(n+m)!}{n! m!}
\end{equation}

Since Equations 3 and 4 are equal, it can be told that $S(m+1, n+1)$ is true.


\newpage
\section*{Q3}
\paragraph{\textbf{a.}}

In picture, \textbf{21} unit square can be formed. For each unit square, \textbf{4} congruent triangles can be drawn. The area that is outside these unit squares (the right side of the picture) there are \textbf{7} congruent triangle.

\begin{equation*}
    21 \times 4 + 7 = 91
\end{equation*}

\textbf{91} congruent triangles can be drawn.

\paragraph{\textbf{b.}}

\begin{equation*}
    4^6 - \binom{4}{1} 3^6 + \binom{4}{2} 2^6 - \binom{4}{3} 1^6 = 1560
\end{equation*}


\section*{Q4}
\paragraph{\textbf{a.}}
\begin{align*}
    &a_n = 2 a_{n-1} + 3^{n-1} &\forall n \geq 3
\end{align*}

\paragraph{\textbf{b.}}
\begin{equation*}
    a_2 = 3
\end{equation*}

\paragraph{\textbf{c.}}
\begin{equation*}
    a_n = a_n^{(h)} + a_n^{(p)}
\end{equation*}

Here, we need to calculate both homogeneous and particular solutions.
\begin{equation*}
    a_n = 2 a_{n-1} + 3^{n-1} \implies a_n - 2 a_{n-1} = 3^{n-1}
\end{equation*}
\\

$a_n^{(h)}$: The associated linear homogeneous equation is  $a_n - 2 a_{n-1} = 0$. Its solutions are $a_n^{(h)} = \alpha 2^n$, where $\alpha$ is constant.\\

$a_n^{(p)}$: We now find a particular solution. Because $F(n) = 3^{n-1}$, a reasonable trial solution is $p_n = A3^n$, where $A$ is constant.

Then equation becomes
\begin{align*}
    a_n - 2 a_{n-1} &= 3^{n-1}\\
    A3^n - 2A3^{n-1} &= 3^{n-1}\\
    3A - 2A &= 1 &\text{when the equation is divided by $3^{n-1}$}\\
    A &= 1\\
\end{align*}

So, $a_n^{(p)} = 3^n$.\\

At the end, since $a_n = a_n^{(h)} + a_n^{(p)}$,

\begin{equation*}
    a_n = \alpha 2^n + 3^n
\end{equation*}

By using $a_2 = 3$, we find $\alpha = \frac{-3}{2}$

\begin{align*}
    a_n &= \frac{-3}{2} 2^n + 3^n\\
    a_n &= -3 \times 2^{n-1} + 3^n
\end{align*}

\end{document}

