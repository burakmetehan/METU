\section{Closure}
\label{sec:closure}

\textit{\textbf{Closure}} is an abstraction method using the saved environment state in a scope. When a function returns a \textit{local object} or \textit{function} as its result and language keeps the environment state along with the returned value, it becomes a \textit{closure}.

\begin{listing}[H]

\begin{minted}{py}
def newid():
  c = 0  # this is the hidden variable 
         # in the environment
  def incget():
    nonlocal c      #python 3, binds c above
    c += 1
    return c
  return incget

>>> a = newid()
>>> b = newid()
>>> a()
1
>>> b()
1
>>> b()
2
\end{minted}

\caption{}
\label{code:code4}
\end{listing}

\vspace*{\fill}
\columnbreak

Local variables of closures stay alive after call, as long as returned value is alive. \textbf{Closures} can be used for generating new functions as in higher order functions:
\begin{listing}[H]

\begin{minted}{py}
def mult(a):
  def nested(b):
    return a*b
  return nested # a different behaviour,
                # for each a value

twice = mult(2)
tentimes = mult(10)
a=twice(4)+tentimes(50)
\end{minted}

\caption{}
\label{code:code5}
\end{listing}

\noindent Also can be used for prototyping objects. \texttt{Javascript} example:
\begin{listing}[H]

\begin{minted}{js}
function counter() {
  var c = 0 // this is jailed in 
            //local environment, hidden
  var newobj = {} // create a new empty object
  newobj.incr = function () { c++; }
  newobj.get = function () { return c;}
  return newobj
}
a = counter()
b = counter()
a.incr()
a.get()
b.get()  
\end{minted}

\caption{}
\label{code:code6}
\end{listing}

\vspace*{\fill}
\columnbreak

\texttt{C++} 2011 implements closures in lambda expressions by adding a set of captured variables within [ ]. This copy or get reference of auto variables in the environment in an object.

However \texttt{C++} closures do not extend lifetime of captured variables. After exit, the behaviour is undetermined.

\mintinline[bgcolor={}]{cpp}{[a, &b] (int x) { return a+x+b;}} captures $a$ and $b$ from current environment, $a$ is by copy, $b$ by reference.
\begin{listing}[H]

\begin{minted}{cpp}
std::function<int(int)>  multiply(int a) {
  // capture by value
  return [&] (int b) { return a*b;};
};
std::function<int()> cid() {
  int c = 0;
  // capture by copy
  return [=] () mutable { return ++c; };
};
int main() {
  std::function<int(int)> twice = multiply(2);
  std::function<int(int)> three = multiply(3);

  cout << twice(12) << ' ' << three(34) << endl;

  auto c1 = cid();
  auto c2 = cid();

  cout << c1() << ' ' << c2() << endl;
  c1(); c1(); c1();
  cout << c1() << ' ' << c2() << endl;
  return 0;
}
\end{minted}

\caption{}
\label{code:code7}
\end{listing}
