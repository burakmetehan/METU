\section{Ripple Counters}
\label{sec:ripple-counter}

A register that goes through a prescribed sequence of states upon the application of input pulses is called a \textit{counter}. A counter that follows the binary number sequence is called a \textit{binary counter}.

\noindent Counters are available in two categories:
\begin{itemize}
  \item Ripple counters
  \item Synchronous counters
\end{itemize}

In a \textit{ripple counter}, a flip-flop output transition serves as a source for triggering other flip-flops. In other words, the clock input of some or all flip-flops are triggered, not by the common clock pulses, but rather by the transition that occurs in other flip-flop outputs.

In a \textit{synchronous counter}, the clock inputs of all flip-flops receive the common clock.

\subsection{Binary Ripple Counter}
\label{subsec:binary-ripple-counter}

A binary ripple counter consists of a series connection of complementing flip-flops, with the output of each flip-flop connected to the $C$ input of the next higher order flip-flop. The logic diagram of two 4-bit binary ripple counters is shown in Fig. 8.

\vspace*{\fill}
\columnbreak

\begin{figure}[H]
  \centering
  \includegraphics[width=\linewidth]{img/fig-6.8.png}
  \caption{Four-bit binary ripple counter}
  \label{fig:6.8}
\end{figure}

\noindent To understand the operation of the four-bit binary ripple counter, refer to the first nine binary numbers listed in Table 6.4. The count starts with binary 0 and increments by 1 with each count pulse input. The count starts with binary 0 and increments by 1 with each count pulse input. After the count of 15, the counter goes back to 0 to repeat the count

\begin{figure}[H]
  \centering
  \includegraphics[width=.5\linewidth]{img/table-6.4.png}
  \label{table:6.4}
\end{figure}

A binary counter with a reverse count is called a \textit{binary countdown counter}. In a countdown counter, the binary count is decremented by 1 with every input count pulse. The count of a four-bit countdown counter starts from binary 15 and continues to binary counts 14, 13, 12, . . . , 0 and then back to 15.

\newpage

\subsection{BCD Ripple Counter}
\label{subsec:bcd-ripple-counter}

A decimal counter follows a sequence of 10 states and returns to 0 after the count of 9. If the BCD code is used, the sequence of states is as shown in the state diagram of Fig. 9.

\begin{figure}[H]
  \centering
  \includegraphics[width=\linewidth]{img/fig-6.9.png}
  \caption{State diagram of a decimal BCD counter}
  \label{fig:6.9}
\end{figure}

\noindent The logic diagram of a BCD ripple counter using JK flip-flops is shown in Fig. 10.

\begin{figure}[H]
  \centering
  \includegraphics[width=.9\linewidth]{img/fig-6.10.png}
  \caption{BCD ripple counter}
  \label{fig:6.10}
\end{figure}

A ripple counter is an asynchronous sequential circuit. Its state changes are not synchronized to a common clock. Signals that affect the flip-flop transition depend on the way they change from 1 to 0.

The BCD counter of Fig. 10 is a \textit{decade counter}, since it counts from 0 to 9. To count from 0 to 999, we need a three-decade counter. A three-decade counter is shown in Fig. 11. 

\begin{figure}[H]
  \centering
  \includegraphics[width=\linewidth]{img/fig-6.11.png}
  \caption{BCD ripple counter}
  \label{fig:6.11}
\end{figure}
