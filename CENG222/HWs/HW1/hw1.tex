\documentclass[12pt]{article}
\usepackage[utf8]{inputenc}
\usepackage{float}
\usepackage{amsmath}

\usepackage[hmargin=3cm,vmargin=6.0cm]{geometry}
\topmargin=-2cm
\addtolength{\textheight}{6.5cm}
\addtolength{\textwidth}{2.0cm}
\setlength{\oddsidemargin}{0.0cm}
\setlength{\evensidemargin}{0.0cm}
\usepackage{indentfirst}
\usepackage{amsfonts}

\usepackage{multicol}
\usepackage{multirow}
\usepackage{caption}
\usepackage{xparse}
\usepackage{setspace}

\newcommand{\prob}[1]{\textbf{\textit{P}}\{#1\}}
\newcommand{\probb}[2]{P_{(#1)}(#2)}
\newcommand{\expc}[1]{\mathbf{E}(#1)} % in equation
\newcommand{\expec}[1]{$\mathbf{E}(#1)$} % outside equation

\onehalfspacing

\begin{document}

\section*{Student Information}

Name : Burak Metehan Tunçel\\

ID : 2468726\\


\section*{Answer 1}

\subsection*{a)}

In this drawing, there are 8 possible mutually exclusive events. The question states ``\underline{at least} one'' condition. So, the following probabilities need to be considered. ($W$: Choosing White, $B$: Choosing Black)

\begin{multicols}{4}
    \begin{enumerate}
        \item $\prob{WBB}$
        \item $\prob{BWB}$
        \item $\prob{BBW}$
        \item $\prob{WWB}$
        \item $\prob{WBW}$
        \item $\prob{BWW}$
        \item $\prob{WWW}$
    \end{enumerate}
\end{multicols}

\noindent Since these events are mutually exclusive, the sum of the probabilities of these 7 seven events is asked. Instead of calculating and adding the above probabilities, we can calculate
\begin{equation*}
    \left( \prob{\Omega} - \prob{BBB} \right) = \left( 1 - \prob{BBB} \right)
\end{equation*}
The reason is that only condition that violates the `\underline{at least}' statement is the event ``$BBB$'', so subtracting its probability from 1 is enough.

\begin{equation*}
    \prob{\Omega} - \prob{BBB} = 1 - \left( \frac{8}{10} \cdot \frac{11}{15} \cdot \frac{9}{12} \right) = \frac{14}{25} = 0.56
\end{equation*}

\subsection*{b)}

Again there are 8 possible mutually exclusive events. However, this time, only one of them is asked. This event is ``$WWW$''. So the probability of this event,
\begin{equation*}
    \prob{WWW} = \frac{2}{10} \cdot \frac{4}{15} \cdot \frac{3}{12} = \frac{1}{75} = 0.01\overline{3}
\end{equation*}

\subsection*{c)}

I would choose the \textbf{BOX 2}. The reason for this choice is that the highest probability for asked condition is on BOX 2. The probabilities for each box:

\begin{itemize}
    \item \prob{Drawing two white balls from BOX 1}: $\dfrac{2}{10} \cdot \dfrac{1}{9} = \dfrac{1}{45} \approx 0.022$
    \item \prob{Drawing two white balls from BOX 2}: $\dfrac{4}{15} \cdot \dfrac{3}{14} = \dfrac{2}{35} \approx 0.057$
    \item \prob{Drawing two white balls from BOX 3}: $\dfrac{3}{12} \cdot \dfrac{2}{11} = \dfrac{1}{22} \approx 0.045$
\end{itemize}

\subsection*{d)}

\textbf{First BOX 2, then BOX 3}.

\noindent Probability of choosing white ball in first draw:
\begin{itemize}
    \item From BOX 1: $\dfrac{2}{10} = 0.20$
    \item From BOX 2: $\dfrac{4}{15} = 0.2666666667$
    \item From BOX 3: $\dfrac{3}{12} = 0.25$
\end{itemize}

\noindent Since, I would choose the \textbf{BOX 2}, in the first draw, because the highest probability for drawing white ball is in BOX 2. 

After choosing the first white ball from BOX 2, probability of choosing white ball in second draw:
\begin{itemize}
    \item From BOX 1: $\dfrac{2}{10} = 0.20$
    \item From BOX 2: $\dfrac{3}{14} = 0.2142857143$
    \item From BOX 3: $\dfrac{3}{12} = 0.25$
\end{itemize}

\noindent When one white ball is drawn from BOX 2, the highest probability for drawing white balls is in BOX 3. Therefore, in the second draw, I would choose \textbf{BOX 3}.

\subsection*{e)}

To calculate the expected value, we need to calculate the probabilities of the following:

\begin{itemize}
    \item \prob{0 White Balls} = $\prob{BBB}$
    \item \prob{1 White Balls} = $\prob{WBB} + \prob{BWB} + \prob{BBW}$
    \item \prob{2 White Balls} = $\prob{WWB} + \prob{WBW} + \prob{BWW}$
    \item \prob{3 White Balls} = $\prob{WWW}$
\end{itemize}

\noindent After the calculations of these probabilities, we need to use the expactation formula for discrete case because this is a discrete case by using the following:
\begin{equation*}
    \mu = \expc{X} = \sum_x x P(x) 
\end{equation*}

\noindent Starting by calculating the probabilities of the events.
\begin{itemize}
    \item \prob{0 White Balls} = $0.440 = 
        \dfrac{8}{10} \cdot \dfrac{11}{15} \cdot \dfrac{9}{12} = \dfrac{11}{25}$
    \item \prob{1 White Balls} = $0.41\overline{6} = 
        \dfrac{2}{10} \cdot \dfrac{11}{15} \cdot \dfrac{9}{12} + 
        \dfrac{8}{10} \cdot \dfrac{4}{15} \cdot \dfrac{9}{12} + 
        \dfrac{8}{10} \cdot \dfrac{11}{15} \cdot \dfrac{3}{12} 
        = \dfrac{5}{12}$
    \item \prob{2 White Balls} = $0.130 =
        \dfrac{2}{10} \cdot \dfrac{4}{15} \cdot \dfrac{9}{12} + 
        \dfrac{2}{10} \cdot \dfrac{11}{15} \cdot \dfrac{3}{12} + 
        \dfrac{8}{10} \cdot \dfrac{4}{15} \cdot \dfrac{3}{12} 
        = \dfrac{13}{100}$
    \item \prob{3 White Balls} = $0.01\overline{3} = 
        \dfrac{2}{10} \cdot \dfrac{4}{15} \cdot \dfrac{3}{12} 
        = \dfrac{1}{75}$
\end{itemize}

\noindent By using this probabilities, we calculate the expected value:
\begin{equation*}
    \mu = 0 \cdot \dfrac{11}{25} + 1 \cdot \dfrac{5}{12} + 2 \cdot \dfrac{13}{100} + 3 \cdot \dfrac{1}{75} = \dfrac{43}{60} = 0.71\overline{6}
\end{equation*}

\newpage

\subsection*{f)}

We are asked $\prob{B | W}$ if the following denotations are used,

\begin{itemize}
    \item $B$ = \{Choosing a ball from BOX 1\}
    \item $W$ = \{Choosing a white ball\}
\end{itemize}

\noindent Intuitively, the following notion can be used
\begin{quote}
    Suppose that all the balls are in a giant box, and all the balls have a number on them that indicates which box they are from. If we know that drawn ball is white the probability that the ball has number 1 on it is $\dfrac{2}{9}$.    
\end{quote}

Instead, we can also use ``\textit{Bayes Rule}'' and ``\textit{Law of Total Probability}'' to show that intuition was not wrong.

\noindent Before, note that we can acquire the followings:
\begin{itemize}
    \item $\prob{B} = \dfrac{10}{37}$: There are 37 balls in total and 10 of them are in BOX 1.
    \item $\prob{\overline{B}} = \dfrac{27}{37}$: There 37 balls in total and 27 of them are not in BOX 1
    \item $\prob{W | B} = \dfrac{2}{10}$: If a ball is drawn from BOX 1, the probability of it should be 2/10.
    \item $\prob{W |\overline{B}} = \dfrac{7}{27}$: There are 27 balls that are not in BOX 1, and 7 of them are white.
\end{itemize}

\noindent Now, we can start using \textit{Bayes Rule} and \textit{Law of Total Probability}:

\begin{align*}
    \prob{B | W} &= \frac{\prob{W | B} \prob{B}}{\prob{W}} &\textit{(Bayes Rules)}\\
    &= \frac{\prob{W | B} \prob{B}}{\prob{W | B} \prob{B} + \prob{W | \overline{B}} \prob{\overline{B}}} &\textit{(Law of Total Probability)}\\
    &= \dfrac{
            \dfrac{2}{10} \cdot \dfrac{10}{37}
        }{
            \dfrac{2}{10} \cdot \dfrac{10}{37} + \dfrac{7}{27} \cdot \dfrac{27}{37}
        }\\
    &= \dfrac{2}{9} \approx 0.22
\end{align*}

\noindent The answer is ``$\dfrac{2}{9}$''


\section*{Answer 2}

For this question the following event denotations will be used.
\begin{align*}
    &
    D = \left\{ \textnormal{The ring is destroyed} \right\}
    &&
    F = \left\{ \textnormal{Frodo is corrupted} \right\}
    &&
    S = \left\{ \textnormal{Sam is corrupted} \right\}
    &
\end{align*}

\subsection*{a)}

\noindent The followings are given:

\begin{multicols}{3}
    \begin{itemize}
        \item $\prob{D | \overline{S}} = 0.9$
        \item $\prob{D | S} = 0.5$
        \item $\prob{S} = 0.1$
    \end{itemize}    
\end{multicols}

\noindent and the following is asked

$\prob{S | D} = ?$

\noindent By using \textit{Bayes Rule}, we can calculate the $\prob{S | D}$:
\begin{align*}
    \prob{S | D} &= \frac{\prob{D|S} \prob{S}}{\prob{D}}\\
    &(\textnormal{We can use the \textbf{Law of Total Probability} to calculate the $\prob{D}$})\\
    &= \frac{\prob{D|S} \prob{S}}{\prob{D|S} \prob{S} + \prob{D|\overline{S}} \prob{\overline{S}}}\\
    &= \frac{(0.5)(0.1)}{(0.5)(0.1)+(0.9)(0.9)} = 0.05813953488 \approx 0.058
\end{align*}

\subsection*{b)}

\noindent The followings are given:

\begin{itemize}
    \begin{multicols}{3}
        \item $\prob{F} = 0.25$
        \item $\prob{S} = 0.1$
        \item $\prob{D | (F \setminus S)} = 0.2$
        \item $\prob{D | (S \setminus F)} = 0.5$
        \item $\prob{D | (\overline{F} \cap \overline{S})} = 0.9$
        \item $\prob{D | (F \cap S)} = 0.05$
    \end{multicols}
    \item The corruption of Frodo and Sam are independent events.
\end{itemize}

\noindent and the following is asked

$\prob{(F \cap S) | D} = ?$

\noindent Since we know the independence from question, we can calculate $\prob{F \cap S}$
\begin{equation}
    \prob{F \cap S} = \prob{F} \prob{S} = (0.25)(0.1) = 0.025
\end{equation}

\noindent Since $F$ is a union of disjoint events $F \setminus S$ and $F \cap S$, and -similarly- S is a union of disjoint events $S \setminus F$ and $F \cap S$, we can compute,
\begin{align}
    \prob{F \setminus S} &= \prob{F} - \prob{F \cap S} = 0.25 - 0.025 = 0.225\\
    \prob{S \setminus F} &= \prob{S} - \prob{F \cap S} = 0.1 - 0.025 = 0.075
\end{align}

Event $(F \setminus S)$, $(S \setminus F)$, $F \cap S$, and $\overline{(F \cup S)}$ form a partition $\Omega$, because they are mutually exclusive and exhaustive. The last of them is the event of no corruption. Notice that $F$, $S$, and $(F \cap S)$ are neither mutually exclusive nor exhaustive, so they cannot be used for the Bayes Rules. Now orginze the data,

\begin{table}[h!]
    \centering
    \begin{tabular}{l|l}
    \multicolumn{1}{c|}{Corruption}      & \multicolumn{1}{c}{Destroy}               \\ 
    \hline
    $\prob{F \setminus S}       = 0.225$ & $\prob{D | (F \setminus S)}       = 0.2$  \\
    $\prob{S \setminus F}       = 0.075$ & $\prob{D | (S \setminus F)}       = 0.5$  \\
    $\prob{F \cap S}            = 0.025$ & $\prob{D | (F \cap S)}            = 0.05$ \\
    $\prob{\overline{F \cup S}} = 0.675$ & $\prob{D | (\overline{F \cup S})} = 0.9$        
    \end{tabular}
\end{table}

Combining the \textit{Bayes Rules} and the \textit{Law of Total Probability},

\begin{equation*}
    \prob{(F \cap S) | D} = \frac{\prob{D | (F \cap S)} \prob{F \cap S}}{\prob{D}}
\end{equation*}
where
\begin{align*}
    \prob{D} = &\prob{D | (F \setminus S)} \prob{F \setminus S} + \prob{D | (S \setminus F)} \prob{S \setminus F}\\
    &+ \prob{D | (F \cap S)} \prob{F \cap S} + \prob{D | (\overline{F \cup S})} \prob{\overline{F \cup S}}
\end{align*}
Then
\begin{align*}
    \prob{(F \cap S) | D} &= \frac{(0.05)(0.025)}{(0.2)(0.225) + (0.5)(0.075) + (0.05)(0.025) + (0.9)(0.675)}\\
    &= 0.001808318264 \approx 0.0018
\end{align*}


\section*{Answer 3}

By using the data provided in PDF, we can construct a table by organizing the joint pmf of $X$ and $Y$. So, in this way, we get the marginal pmf. (This table is similar to the table in example 3.6 in book.)

% Starting Huge
\renewcommand{\arraystretch}{1.5}
\begin{table}[h!]
\centering
\begin{tabular}{|c|c|c|c|c|c|} 
    \hline
    \multicolumn{2}{|c|}{\multirow{2}{*}{$P_{(A,I)}(a, i)$}} & \multicolumn{3}{c|}{$a$}             & \multirow{2}{*}{$P_{I}(i)$}  \\ 
    \cline{3-5}
    \multicolumn{2}{|c|}{}                           & \textbf{1} & \textbf{2} & \textbf{3} &                          \\ 
    \hline
    \multirow{2}{*}{$i$} & \textbf{1}                & 0.18       & 0.3        & 0.12       & 0.6                      \\
                         & \textbf{2}                & 0.12       & 0.2        & 0.08       & 0.4                      \\ 
    \hline
    \multicolumn{2}{|c|}{$P_A(a)$}                     & 0.3        & 0.5        & 0.2        & 1.0                      \\
    \hline
\end{tabular}
\caption*{Table 1}
\end{table}

\subsection*{a)}

\noindent There are 2 cases that satisfy the conditon \textit{four snowy days}:
\begin{multicols}{4}
    \begin{itemize}
        \item $a = 2,\ i = 2$
        \item $a = 3,\ i = 1$
    \end{itemize}
\end{multicols}

\noindent So we need to calculate the followings:
\begin{multicols}{4}
    \begin{itemize}
        \item $\probb{A,I}{2, 2}$
        \item $\probb{A,I}{3, 1}$
    \end{itemize}
\end{multicols}

\noindent From Table 1, we can read these probabilities.

\begin{equation*}
    \prob{\text{Four Snowy Days}} = \probb{A, I}{2, 2} + \probb{A, I}{3, 1} = 0.2 + 0.12 = 0.32
\end{equation*}

\noindent The answer is ``\textbf{0.32}''.

\subsection*{b)}

To decide on the independence of $A$ and $I$, we need to check if their joint pmf factors into a product of marginal pmfs. If there is at least one pair that violates the equality, it showing violation is enough to porve the dependence.

Therefore, we need to check all pairs.
\begin{multicols}{2}
    \begin{itemize}
        \item $\probb{A,I}{1, 1} = 0.3 \cdot 0.6 = 0.18$
        \item $\probb{A,I}{1, 2} = 0.3 \cdot 0.4 = 0.12$
        \item $\probb{A,I}{2, 1} = 0.5 \cdot 0.6 = 0.30$
        \item $\probb{A,I}{2, 2} = 0.5 \cdot 0.4 = 0.20$
        \item $\probb{A,I}{3, 1} = 0.2 \cdot 0.6 = 0.12$
        \item $\probb{A,I}{3, 2} = 0.2 \cdot 0.4 = 0.08$
    \end{itemize}    
\end{multicols}

\noindent All pairs satisfy the condition for independence. There is no pair to violate the equality mentioned above.

Answer is ``\textbf{Independent}''.

\end{document}
