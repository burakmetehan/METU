\section{Simple and Composite Variables}

\subsection{Simple Variables}

A \textit{\textbf{simple variable}} is a variable that may contain a storable value. \textit{Each simple variable occupies a single storage cell}.

\subsection{Composite Variables}

A \textit{\textbf{composite variable}} is a variable of a composite type. \textit{Each composite variable occupies a group of contiguous storage cells}.

\subsubsection{Total vs Selective Update}

A composite variable may be updated either \textit{in a single step} or \textit{in several steps}, one component at a time. \textit{\textbf{Total update}} of a composite variable means updating it with a new (composite) value in a single step. \textit{\textbf{Selective update}} of a composite variable means updating a single component.

\begin{listing}[H]
\begin{minted}[]{c}
struct Complex { double x, y; } a, b;
...
a = b; // Total update
a.x = b.y * a.x; // Selective update
\end{minted}
\caption{}
\label{code:code-2}
\end{listing}

\subsubsection{Static vs Dynamic vs Flexible Arrays}

We can view an array variable as a mapping from an index range to a group of component variables. How and when a given array variable's index range is determined can be changed. There are several possibilities:
\begin{itemize}
  \item the index range might be fixed at compile-time
  \item the index range might be fixed at run-time when the array variable is created
  \item the index range might not be fixed at all
\end{itemize}

A \textit{\textbf{static array}} is an array variable whose index range is fixed at compile-time. In other words, the program code determines the index range.

A \textit{\textbf{dynamic array}} is an array variable whose index range is fixed at the time when the array variable is created.

A \textit{\textbf{flexible array}} is an array variable whose index range is not fixed at all. A flexible array's index range may be changed when a new array value is assigned to it.
