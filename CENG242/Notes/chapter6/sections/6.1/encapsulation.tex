\subsection{Encapsulation}
\label{subsec:encapsulation}

In order to keep its API simple, a package typically makes only \textit{some of its components visible} to the application code that uses the package; these components are said to be \textit{\textbf{public}}. \textit{Other components are visible only inside the package}; these
components are said to be \textit{\textbf{private}}, and serve only to support the implementation of the public components.

A package with private components hides information that is irrelevant to the application code. This technique for making a simple API is called \textit{\textbf{encapsulation}}.

The \textit{\textbf{package specification}} declares only the \textit{public components}, while the \textit{\textbf{package body}} declares any \textit{private components}.The package specification gives the package's interface, while the package body provides the implementation details.

Building an independent and self complete set of function and variable declarations is \underline{\textit{packaging}}. Restricting access to this set only via a set of interface
function and variables is \underline{\textit{hiding and encapsulation}}.

\begin{figure}[H]
  \centering
  \begin{tikzpicture}[private/.style = {black,thick}, public/.style = {black,thick},
		pubcall/.style = {transparent}, pubtext/.style = {transparent},pricall/.style={transparent},
		pritext/.style={transparent} ]
    \tikzset{private/.style={red,thick}, public/.style = {green,thick}}
    \tikzset{pubcall/.style={green,draw,thick},pubtext/.style={green,fill=white}, pricall/.style={red,draw,thick},pritext/.style={red,fill=white}}
    \node [rectangle split, rectangle split parts=2, text width=9em,rectangle split draw splits=true] (box) {
      interface 
      \nodepart{two} 
      detail \\
      \footnotesize ( functions, variables, algorithm) 
    };
    \node [right of=box,xshift=4cm] (other) {other application};
    \draw (box.text split west) -- (box.text split east);
    \draw [private] (box.north west) -- (box.south west) -- (box.south east) -- (box.north east);
    \draw [public] (box.north west) -- (box.north east);
    \path [pubcall,->] (other) -- +(0,1.2) -| node [pubtext,pos=0.3] {$\surd$} (box.mid);
    \path [pricall,->,bend left=10] (other) to  node [pritext] {$\times$} ($(box.two) +(2cm,-2mm)$);
  \end{tikzpicture}
\end{figure}

\vspace*{\fill}
\columnbreak

\subsubsection*{Advantages of Encapsulation}
\label{subsubsec:adv-encap}

\begin{itemize}
  \item High volume details reduced to interface definitions (\textbf{Ease of development/maintenance})
  \item Many different applications use the same module via the same interface (\textbf{Code re-usability})
  \item Lego like development of code with building blocks (\textbf{Ease of
  development/maintenance}) 
  \item Even details change, applications do not change (as long as interface is kept same) (\textbf{Ease of development/maintenance})
  \item Module can be used in following projects (\textbf{Code re-usability})
\end{itemize}

\subsubsection*{Hiding}
\label{subsubsec:hiding}

A group of functions and variables hidden inside. The others are interface. Abstraction inside of a package:

\begin{table}[H]
  \begin{tabular}{|>{\tt}l|} \hline
    \color{gray}double taylorseries(double);  		\\
    double sin(double x);			\\
    double pi=3.14159265358979;		\\
    \color{gray}double randomseed;	\\
    double cos(double x);			\\
    \color{gray}double errorcorrect(double x); \\ \hline
  \end{tabular}
\end{table}
\begin{listing}[H]
\begin{minted}{Haskell}
{-- only sin, pi and cos are accessible --}
module Trig(sin,pi,cos) where
  taylorseries x = ... 
  sin x = ...
  pi=3.14159265358979
  randomseed= ...
  cos x = ...
  errorcorrect x = ...
\end{minted}
\caption{}
\label{code:code2}
\end{listing}
