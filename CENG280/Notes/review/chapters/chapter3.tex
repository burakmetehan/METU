\section{Chapter 3}

\begin{multicols}{2}
\setlength{\columnsep}{1.5cm}
\setlength{\columnseprule}{0.2pt}

\subsection{Context Free Grammars}

\begin{definition}{}
  A \textbf{context-free grammar} $G$ is a quadruple $(V, \Sigma, R, S)$ where
  \begin{itemize}
    \item $V$ is an alphabet
    \item $\Sigma$ (the set of \textbf{terminals}) is a subset of $V$
    \item $R$ (the set of \textbf{rules}) is a finite subset of $(V - \Sigma) \times V^*$
    \item $S$ (the \textbf{start symbol}) is an element of $V - \Sigma$
  \end{itemize}
\end{definition}

The members of $V - \Sigma$ are called \textbf{nonterminals}. For any $A \in V - \Sigma$ and $u \in V^*$, we write $A \rightarrow_G u$ whenever $(A, u) \in R$. 
  
For any strings $u, v \in V^*$, we write $u \Rightarrow_G v$ if and only if there are strings $x, y \in V^*$ and $A \in V - \Sigma$ such that $u = xAy$, $v = xv'y$, and $A \rightarrow_G v'$. 

The relation $\Rightarrow_G^*$ is the \textit{reflexive}, \textit{transitive closure} of $\Rightarrow_G$. Finally, $L(G)$, the \textbf{language generated} by $G$, is $\left\{ w \in \Sigma^* : S \Rightarrow_G^* w \right\}$; we also say that $G$ \textbf{generates} each string in $L(G)$. A language $L$ is said to be a \textbf{context-free language} if $L = L(G)$ for some context-free grammar $G$.

We call any sequence of the form
\begin{equation*}
  w_0 \Rightarrow_G w_1 \Rightarrow_G \cdots \Rightarrow_G w_n
\end{equation*}
a \textbf{derivation} in $G$ of $w_n$ from $w_0$. Here $w_o, \cdots, w_n$ may be any strings in $V^*$, and $n$, the \textbf{length} of the derivation, may be any natural number, including zero. We also say that the derivation has $n$ \textbf{steps}. 

\end{multicols}

\begin{formula}{}
  \begin{itemize}
    \item $u \Rightarrow v$: $u$ \textbf{\textit{directly yields}} $v$; $A \rightarrow w$: \textbf{\textit{(production) rule}}.
    \item $V$, alphabet, can include symbols such as start symbol, $S$, or $A$.
    \item (From example 3.1.4) The same string may have several derivations in a context-free grammar. Two derivations in this grammar are
    \begin{align*}
      &S \Rightarrow SS \Rightarrow S(S) \Rightarrow S((S)) \Rightarrow S(()) \Rightarrow ()(()) &\textnormal{ and } && S \Rightarrow SS \Rightarrow (S)S \Rightarrow ()S \Rightarrow ()(S) \Rightarrow ()(())
    \end{align*}
    \item Some context-free languages are not regular. However, all regular languages are context-free.
    \item Context-free languages are precisely the languages accepted by certain language acceptors called \textit{\textbf{pushdown automata}}.
  \end{itemize}
\end{formula}


% \begin{multicols}{2}
%   \setlength{\columnsep}{1.5cm}
%   \setlength{\columnseprule}{0.2pt}


% \end{multicols}

