\subsection{Estimating Means with a Given Precision}
\label{subsec:estimating-means-with-given-precision}

When we estimate a population mean, the margin of error is
\begin{equation*}
  \textnormal{margin} = z_{\alpha / 2} \cdot \sigma / \sqrt{n}
\end{equation*}
Solving inequality (7) for $n$ results in the following rule.
\begin{formula}{Sample size for a given precision}
  In order to attain a margin of error $\Delta$ for estimating a population mean with a confidence level $(1 - \alpha)$, a simple of size
  \begin{equation}
    n \geq \left( \frac{z_{\alpha/2} \cdot \sigma}{\Delta} \right)^2
  \end{equation}
  is required.
\end{formula}
When we compute the expression in (8), it will most likely be a fraction. Notice that we can only round it up to the nearest integer sample size. If we round it down, our margin will exceed $\Delta$.

Looking at (8), we see that a large sample will be necessary
\begin{itemize}
  \item to attain a narrow margin (small $\Delta$);
  \item to attain a high confidence level (small $\alpha$); and
  \item to control the margin under high variability of data (large $\sigma$).
\end{itemize}
In particular, we need to quadruple the sample size in order to half the margin of the interval.

\begin{example}{}
  In Example 4,  we constructed a $95\%$ confidence with the center $6.50$ and margin $1.76$ based on a sample of size $6$. Now, that was too wide, right? How large a sample do we need to estimate the population mean with a margin of at most $0.4$ units with $95\%$ confidence?

  \textbf{Solution:}
  We have $\Delta = 0.4$, $\alpha = 0.05$, and $\sigma = 2.2$ (from Example 4). By (8), we need a sample of
  \begin{equation*}
    n \geq \left( \frac{z_{0.05 / 2} \sigma}{\Delta} \right)^2 = \left( \frac{(1.960) (2.2)}{0.4} \right)^2 = 116.2
  \end{equation*}
  Keeping in mind that this is the minimum sample size that satisfies $\Delta$, and we are only allowed to round it up, we need a sample of at least 117 observations.  
\end{example}
