\subsection{\textit{Z}-Test: Standard Normal Null Distribution}
\label{subsec:z-test}
\setcounter{equation}{13}

An important case, in terms of a large number of applications, is when the null distribution of the test statistic is \textit{Standard Normal}.

The test in this case is called a \textbf{Z-test}, and the test statistic is usually denoted by $Z$.

\begin{enumerate}[label=(\alph*)]
  \item A level $\alpha$ test with a \textbf{right-tail alternative} should
    \begin{equation}
      \begin{cases}
        \textnormal{reject } H_0 &\textnormal{if } Z \geq z_{\alpha}\\
        \textnormal{accepts } H_0 &\textnormal{if } Z < z_{\alpha}\\
      \end{cases}
    \end{equation}
    The rejection region in this case consists of large values of $Z$ only,
    \begin{align*}
      \mathcal{R} = \left[ z_{\alpha},\ +\infty \right), & &\mathcal{A} = \left( -\infty,\ z_{\alpha} \right)
    \end{align*}
    Under the null hypothesis, $Z$ belongs to $\mathcal{A}$ and we reject the null hypothesis with probability
    \begin{equation*}
      \prob{T \geq z_{\alpha}\ |\ H_0} = 1 - \Phi(z_{\alpha}) = \alpha
    \end{equation*}
    making the probability of false rejection (type I error) equal $\alpha$.

    For example, we use this acceptance region to test the population mean,
    \begin{align*}
      H_0\ :\ \mu = \mu_0 &&\textnormal{vs}&& H_A\ :\ \mu > \mu_0
    \end{align*}
  
  \item With a \textbf{left-tail alternative} should
    \begin{equation}
      \begin{cases}
        \textnormal{reject } H_0 &\textnormal{if } Z \leq -z_{\alpha}\\
        \textnormal{accepts } H_0 &\textnormal{if } Z > -z_{\alpha}\\
      \end{cases}
    \end{equation}
    The rejection region consists of smal values of $Z$ only,
    \begin{align*}
      \mathcal{R} = \left( -\infty,\ -z_{\alpha} \right], & &\mathcal{A} = \left( -z_{\alpha},\ +\infty \right)
    \end{align*}
    Similarly, $\prob{Z \in \mathcal{R}} = \alpha$ under $H_0$; thus, the probability of type I error equals $\alpha$. For example, this is how we should test
    \begin{align*}
      H_0\ :\ \mu = \mu_0 &&\textnormal{vs}&& H_A\ :\ \mu < \mu_0
    \end{align*}

  \item With a \textbf{two-sided alternative}, we
    \begin{equation}
      \begin{cases}
        \textnormal{reject } H_0 &\textnormal{if } |Z| \geq z_{\alpha/2}\\
        \textnormal{accepts } H_0 &\textnormal{if } |Z| < z_{\alpha/2}\\
      \end{cases}
    \end{equation}
    The rejection region consists of very small and very large values of $Z$,
    \begin{align*}
      \mathcal{R} = \left( -\infty,\ z_{\alpha} \right] \cup \left[ z_{\alpha/2},\ +\infty \right), & &\mathcal{A} = \left( -z_{\alpha/2},\ z_{\alpha/2} \right)
    \end{align*}
    Again, the probability of type I error equals $\alpha$. For example, we use this test for
    \begin{align*}
      H_0\ :\ \mu = \mu_0 &&\textnormal{vs}&& H_A\ :\ \mu \neq \mu_0
    \end{align*}
\end{enumerate}

This is easy to remember:
\begin{itemize}
  \item for a two-sided test, divide $\alpha$ by two and use $z_{\alpha/2}$;
  \item for a one-sided test, use $z_{\alpha}$ keeping in mind that the rejection region consists of just one piece.
\end{itemize}

Now consider testing a hypothesis about a population parameter $\theta$. Suppose that its estimator $\hat{\theta}$ has Normal distribution, at least approximately, and we know $\expc{\hat{\theta}}$ and $\var(\hat{\theta})$ if the hypothesis is true.

Then the test statistic
\begin{equation}
  Z = \frac{\hat{\theta} - \expc{\hat{\theta}}}{\sqrt{\var{\hat{\theta}}}}
\end{equation}
has Standard Normal distribution, and we can use (14), (15), and (16) to construct acceptance and rejection regions for a level $\alpha$ test. We call $Z$ a \textbf{Z-statistic}.
