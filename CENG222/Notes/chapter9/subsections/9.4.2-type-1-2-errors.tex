\subsection{Type I and Type II errors: Level of Significance}
\label{subsec:type-1-2-errors}

When testing hypotheses, we realize that all we see is a random sample. Therefore, with all the best statistics skills, our decision to accept or to reject $H_0$ may still be wrong
\begin{table}[H]
  \centering
  \renewcommand{\arraystretch}{2}
  \begin{tabular}{|c|c|c|} 
  \cline{2-3}
  \multicolumn{1}{c|}{} & \multicolumn{2}{c|}{\textbf{Result of the test}}  \\ 
  \cline{2-3}
  \multicolumn{1}{c|}{} & Reject $H_0$  & Accept $H_0$               \\ 
  \hline
  $H_0$ is true          & Type I error & correct                   \\ 
  \hline
  $H_0$ is false         & correct      & Type II error             \\
  \hline
  \end{tabular}
\end{table}
\begin{definition}{}
  A \textbf{type I error} occurs when we reject the true null hypothesis.\\

  A \textbf{type II error} occurs when we accept the false null hypothesis.
\end{definition}

A type I error is often considered \textit{more dangerous and undesired} than a type II error. For this reason,  we shall design tests that bound the probability of type I error by a preassigned small number $\alpha$. Under this condition, we may want to minimize the probability of type II error.

\begin{definition}{}
  Probability of a type I error is the \textbf{significance level} of a test,
  \begin{equation*}
    \alpha = \prob{\textnormal{reject } H_0\ |\ H_0 \textnormal{ is true}}
  \end{equation*}
  Probability of rejecting a false hypothesis is the \textbf{power of the test},
  \begin{equation*}
    p(\theta) = \prob{\textnormal{reject } H_0\ |\ H_A \textnormal{ is true}}
  \end{equation*}
  It is usually a function of the parameter $\theta$ because the alternative hypothesis includes a set of parameter values. Also, the power is the probability to avoid a Type II error.
\end{definition}
