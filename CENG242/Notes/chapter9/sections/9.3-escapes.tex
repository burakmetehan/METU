\section{Escapes}
\label{sec:escapes}

An \textit{\textbf{escape}} is a sequencer that terminates execution of a textually enclosing command or procedure. In terms of a flowchart, the destination of an escape is always the exit point of an enclosing subchart. With escapes we can program \textit{single-entry multi-exit control flows}.

\noindent Depending on which enclosing block to jump out of:
\begin{itemize}
  \item loop: \textcolor{ex}{break} sequencer.
  \item loops: \textcolor{ex}{exit} sequencer.
  \item function: \textcolor{ex}{return} sequencer.
  \item program: \textcolor{ex}{halt} sequencer.
\end{itemize}

\subsection{Break Sequencer}
\label{subsec:break-seq}

break sequencer in C, C++, Java terminates the innermost
enclosing loop block.

The break sequencer of \texttt{C}, \texttt{C++}, and \texttt{JAVA} allows any composite command (typically a loop or switch command) to be terminated immediately.

\begin{itemize}
  \item \textcolor{ex}{break sequencer} in \texttt{C}, \texttt{C++}, \texttt{Java} terminates the innermost enclosing loop block.
  \item \textcolor{ex}{continue} in \texttt{C}, \texttt{C++} stays in the same block but ends current iteration.
  \item \textcolor{ex}{exit sequencer} in \texttt{Ada} or labeled \texttt{break} in Java can terminate multiple levels of blocks by specifying labels. Java code:
\begin{listing}[H]
\begin{minted}[breaklines]{java}
  L1: for (i = 0; i < 10; i++) {
        for (j = i; j < i; j++) {
          if (...) break;
          else if (...) continue;
          else if (...) break L1;
          else if (...) continue L1;
          s += i*j;
        }
      }
\end{minted}
\caption{}
\label{code:code3}
\end{listing}     

  \item \textcolor{ex}{return sequencer} exist in most languages for terminating the innermost function block.
  \item \textcolor{ex}{halt sequencer} either provided by operating system or PL terminates the program.
\end{itemize}

Consider jump inside of a block or jump out of a block for the function case:
\begin{listing}[H]
\begin{minted}[breaklines]{c}
int f(int n) {
    int a;

L1: if (n<0) goto L2;   @\color{violet}\circtxt{1}@
    else if (n=1) return 1;
    else return f(n-1)*n;
}

int main() {
    ...
    f(12);
L2: ....
    goto L1:             @\color{violet}\circtxt{2}@
    }
\end{minted}
\caption{Jump out of a function block, jump inside of a function block}
\label{code:code4}
\end{listing} 

Jumps update current instruction pointer. But what about environment, activation record (run-time stack)? Jumping outside or inside a function block is possible only for one direction if stack position can be
recovered. Called \textit{non-local jumps}.

% \begin{figure}[T]
%   \centering
%   \begin{tikzpicture}
%     [stack/.style={rectangle split, rectangle split parts=4,draw,
%       text width=7em},
%       scolor/.style={rectangle split part fill={white,blue!20!white,blue!20!white,blue!20!white}}]

%     \node [stack,scolor] (stk1) {
%       \nodepart{one} \rule{0pt}{3em}
%       \nodepart{two} record of f()
%       \nodepart{three} record of g()
%       \nodepart{four} record of main()
%     };

%     \node [stack,scolor,rectangle split parts=3, right=5em of stk1] (stk2) {
%       \nodepart{one} \rule{0pt}{4.7em}
%       \nodepart{two} record of g()
%       \nodepart{three} record of main()
%     };

%     \node [stack,scolor, rectangle split parts=3,right=5em of stk2] (stk3) {
%       \nodepart{one} \rule{0pt}{4.7em}\hspace*{2em} ??
%       \nodepart{two} record of g()
%       \nodepart{three} record of main()
%     };

%     \draw [->,green,thick] (stk1.two east) -- +(1em,0) |- (stk2.two west);
%     \draw [->,red,thick] (stk2.two east) -- +(1em,0) node [fill=white, yshift=.9em] {$\times$} |-  ($(stk3.one west) - (0,1.6em)$);
%   \end{tikzpicture}
%   \caption
% \end{figure}

Non-local jumps can be useful in unexpected error occuring inside of many levels of recursion in order to jump to the outer-most related caller function. Instead of jumping, it can be satisfied by \textbf{exceptions}.
