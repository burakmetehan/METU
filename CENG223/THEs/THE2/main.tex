\documentclass[11pt]{article}
\usepackage[utf8]{inputenc}
\usepackage[dvips]{graphicx}
\usepackage{fancybox}
\usepackage{verbatim}
\usepackage{array}
\usepackage{latexsym}
\usepackage{alltt}
\usepackage{hyperref}
\usepackage{textcomp}
\usepackage{color}
\usepackage{amsmath}
\usepackage{amsfonts}
\usepackage{tikz}
\usepackage{fitch}  % to use fitch
\usepackage{float}
\usepackage[hmargin=3cm,vmargin=5.0cm]{geometry}
%\topmargin=0cm
\topmargin=-2cm
\addtolength{\textheight}{6.5cm}
\addtolength{\textwidth}{2.0cm}
%\setlength{\leftmargin}{-5cm}
\setlength{\oddsidemargin}{0.0cm}
\setlength{\evensidemargin}{0.0cm}
\usepackage[most]{tcolorbox}


\begin{document}

\section*{Student Information } 
%Write your full name and id number between the colon and newline
%Put one empty space character after colon and before newline
Full Name :  Burak Metehan Tunçel \\
Id Number :  2468726 \\

\section*{Q. 1}
Given the sets A and B, prove that
\begin{equation*}
    (A \cup B) \setminus (A \cap B) = (A \setminus B) \cup (B \setminus A)
\end{equation*}
using set membership notation and logical equivalences. Show each step clearly.

\section*{S. 1}
Prove the equation
\begin{equation*}
    (A \cup B) \setminus (A \cap B) = (A \setminus B) \cup (B \setminus A)
\end{equation*}

\begin{align*}
    &(A \cup B) \setminus (A \cap B)\\
    &= \{ x\ |\ x \in (A \cup B) \land x \notin (A \cap B) \} &\text{(Definition of difference)}\\
    &= \{ x\ |\ (x \in A \lor x \in B) \land x \notin (A \cap B) \} &\text{(Definition of Union)}\\
    &= \{ x\ |\ (x \in A \lor x \in B) \land \neg (x \in (A \cap B)) \} & \text{(Definition\ of\ $\notin$)}\\
    &= \{ x\ |\ (x \in A \lor x \in B) \land \neg (x \in A \land x \in B)) \} & \text{(Definition\ of\ Intersection)}\\
    &= \{ x\ |\ (x \in A \lor x \in B) \land (\neg (x \in A) \lor \neg (x \in B)) \} & \text{(De Morgan's for Logical Rules)}\\
    &= \{ x\ |\ (x \in A \lor x \in B) \land (x \notin A \lor x \notin B) \} & \text{(Definition of $\notin$)}\\
    &= \{ x\ |\ ((x \in A \lor x \in B) \land x \notin A) \lor ((x \in A \lor x \in B) \land x \notin B) \} & \text{(Definition of Distributive Law)}\\
    &= \{ x\ |\ ((x \in A \land x \notin A) \lor (x \in B \land x \notin A)) \lor \\ 
    &\quad((x \in A \land x \notin B) \lor (x \in B \land x \notin B)) \} & \text{(Definition of Distributive Law)}\\
    &= \{ x\ |\ (\emptyset \lor (x \in B \land x \notin A)) \lor ((x \in A \land x \notin B) \lor \emptyset)\} &\text{(Domination Law)}\\
    &= \{ x\ |\ (x \in B \land x \notin A) \lor (x \in A \land x \notin B)\} &\text{(Identity Law)}\\
    &= \{ x\ |\ x \in (B \setminus A) \lor x \in (A \setminus B)\} &\text{(Definition of difference)}\\
    &= (B \setminus A) \cup (A \setminus B) &\text{(Definition of Union)}\\
    &= (A \setminus B) \cup (B \setminus A) &\text{(Commutative Law)}
\end{align*}

\newpage

\section*{Q. 2}
Prove that the set

\begin{equation*}
    \{f\ |\ f \subseteq \mathbb{N} \times \{0, 1\} \text{, f is a function}\} \setminus \{f\ |\ f : \{0, 1\} \rightarrow \mathbb{N} \text{, f is a function}\}
\end{equation*}

\noindent is uncountable.

\section*{S. 2}

\subsection*{Proof 1}
Proof of the fact that if $A$ is an \textbf{uncountable} set and $B$ is a \textbf{countable} set, then $A \setminus B$ is \textbf{uncountable}.\\

Suppose for an uncountable set $A$ and a countable set $B$ that $A \setminus B$ is countable. The union of countably many countable sets is countable; thus $(A \setminus B) \cup B$ is countable. But then $A$ is a subset of $(A \setminus B) \cup B$ and thus must be countable itself, which is a \textbf{contradiction}.\\

Therefore, if $A$ is an \textbf{uncountable} set and $B$ is a \textbf{countable} set, then $A \setminus B$ is \textbf{uncountable}.

\subsection*{Proof 2}

Let $S_1$ and $S_2$ be countable sets. From the definition of countable, there exists a \textbf{injection} from $S_1$ to $\mathbb{N}$, and from $S_2$ to $\mathbb{N}$. Hence, there exists an \textbf{injection} $g$ from $S_1 \times S_2$ to $\mathbb{N}^2$.\\

\noindent Now let us investigate the \textbf{cardinality} of N2. From the \textbf{Fundamental Theorem of Arithmetic}, every natural number greater than 1 has a unique prime decomposition. Thus, if a number can be written as $2^n 3^m$, it can be done thus in only one way. So, consider the function $f : \mathbb{N}^2 \rightarrow \mathbb{N}$ defined by:\\

$f(n, m) = 2^n 3^m$\\

\noindent Now suppose $\exists m,n,r,s \in \mathbb{N}$ such that $f(n, m) = f(r, s)$. Then $2^n 3^m = 2^r 3^s$ so that $n = r$ and $m = s$. Thus $f$ is an \textbf{injection}; hence, $\mathbb{N}^2$ is \textbf{countably infinite}.\\

\subsection*{Solution}

Since Cartesian product of two countable sets is countable set \textbf{(Proof 2)}. $\mathbb{N} \times \{0, 1\}$ \text{ is a infinitely countable set}. 

\begin{equation*}
    \{f\ |\ f \subseteq \mathbb{N} \times \{0, 1\} \text{, f is a function}\} \text{ is power set of } \{\mathbb{N} \times \{0, 1\}\}
\end{equation*}

Power set of a infinitely countable set is uncountable according to \textbf{Cantor's Theorem}. Therefore, $\{f\ |\ f \subseteq \mathbb{N} \times \{0, 1\} \text{, f is a function}\}$ is an \textbf{uncountable} set.\\

Also, $\{f\ |\ f : \{0, 1\} \rightarrow \mathbb{N}$ \ is a countable set because $f$ is a function, whose domain is $\{0, 1\}$ while its range is $\mathbb{N}$.\\

According to \textbf{Proof 1},
\begin{equation*}
    \{f\ |\ f \subseteq \mathbb{N} \times \{0, 1\} \text{, f is a function}\} \setminus \{f\ |\ f : \{0, 1\} \rightarrow \mathbb{N} \text{, f is a function}\}
\end{equation*}
is \textbf{uncountable}.

\newpage

\section*{Q. 3}
Prove that the function $f(n) = 4^n + 5n^2logn$ is \textbf{not} $O(2^n)$.

\section*{S. 3}
$f$ is a function. Since $f$ contains $logn$, its domain needs to be the set D = $\{n \in \mathbb{R}\ |\ n > 0 \}$.

\begin{equation}
    4^x > 2^x,\ \forall x \in \mathbb{R}(x > 0)
\end{equation}

The Eq.1 can be proved by using graph of the functions.\\


\textbf{Definition 1 in Book Chapter 3.2.2: } Let $f$ and $g$ be functions from the set of integers or the set of real numbers to the set of real numbers. We say that $f(x)$ is $O(g(x))$ if there are constants $C$ and $k$ such that

\begin{equation*}
    |f(x)| \leq C |g(x)|
\end{equation*}

whenever $x > k$. [This is read as "$f(x)$ is big-oh of $g(x)$."]\\

\noindent We can prove that $f(n) = 4^n + 5n^2logn$ is \textbf{not} $O(2^n)$ by using contradiction. If the function $f$ is $O(2^n)$, then we can say that
\begin{align}
    &\ \ \ \ |f(n)| \leq C |2^n| &\forall n \in \mathbb{R}(n > k)\\
    &=|4^n + 5n^2logn| \leq C |2^n| &\forall n \in \mathbb{R}(n > k)
\end{align}

There is a contradiction because there is not specific $C$ and $k$ pair that satisfies the inequality $\forall n$. Therefore, the function $f(n)$ is \textbf{not} $O(2^n)$

\newpage

\section*{Q. 4}
Given two positive integers $x$ and $n$ such that $x > 2$ and $n > 2$, and the congruence relation
\begin{equation*}
    (2x - 1)^n - x^2 \equiv -x - 1\ (\bmod{(x-1)})  
\end{equation*}
determine the value of $x$.

\section*{S. 4}
\subsection*{Corollary 1}
\begin{align}
    \text{if } a \equiv b \bmod{n} \text{, then } a^k \equiv b^k \bmod{n}
\end{align}

\subsection*{Solution}
\begin{align*}
    (2x - 1)^n - x^2 &\equiv -x - 1\ (\bmod{(x-1)})\\
    (2x - 1)^n &\equiv x^2 - x - 1\ (\bmod{(x-1)})\\
    (2x - 1)^n &\equiv -1\ (\bmod{(x-1)}) &[x^2 - x - 1 &\equiv -1\ (\bmod{(x-1)})]\\
    (2x - 1)^n &\equiv x - 2\ (\bmod{(x-1)}) &[-1 &\equiv x - 2\ (\bmod{(x-1)})]\\
    1 &\equiv x - 2\ (\bmod{(x-1)}) &[2x - 1 &\equiv 1\ (\bmod{(x-1)}) \text{ and by \textbf{Corollary 1}}]\\
    3 &\equiv x\ (\bmod{(x-1)})\\
    3 &\equiv 1\ (\bmod{(x-1)})
\end{align*}

$x - 1 = 2$. Therefore, $x = 3$.

\end{document}
