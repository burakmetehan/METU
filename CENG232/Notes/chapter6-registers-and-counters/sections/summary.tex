\section{Summary}

\subsection{Register}

\begin{itemize}
  \item A register is a special sequential circuit that stores multiple bits of data. Several variations are possible:
    \begin{itemize}
      \item Parallel loading to store data into the register.
      \item Shifting the register contents either left or right.
      \item Counters are considered a type of register too!
    \end{itemize}
  
  \item One application of shift registers is converting between serial and parallel data.
  \item Registers are a central part of modern processors, as we will see in coming weeks.
\end{itemize}

\subsection{Counters}

\begin{itemize}
  \item Counters serve many purposes in sequential logic design.
  \item There are lots of variations on the basic counter.
    \begin{itemize}
      \item Some can increment or decrement.
      \item An enable signal can be added.
      \item The counter's value may be explicitly set.
    \end{itemize}
  
  \item There are also several ways to make counters.
  \begin{itemize}
    \item You can follow the sequential design principles from last week to build counters from scratch.
    \item You could also modify or combine existing counter devices.
  \end{itemize}  
\end{itemize}