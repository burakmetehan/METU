\section{Overloading}
\label{sec:overloading}

An identifier is said to be \textit{\textbf{overloaded}} if it denotes two or more distinct procedures in the same scope. In other words, \textit{\textbf{overloading}} is using same identifier for multiple places in same
scope.

Such overloading is acceptable only if every procedure call is unambiguous, i.e., the compiler can uniquely identify the procedure to be called using only type information.

\textit{Polymorphic function: one function that can process multiple types.}

C++ allows overloading of functions and operators.
\begin{listing}[H]
\begin{minted}{cpp}
typedef struct Comp { double x, y; } Complex;
double @\color{orange}mult@(double a, double b) { return a * b; }
Complex @\color{brown}mult@(Complex s, Complex u) {
  Complex t;
  t.x = s.x*u.x - s.y*u.y;
  t.y = s.x*u.y + s.y*u.x;
  return t;
}
Complex a,b; double x,y;
a=@\color{brown}mult@(a,b) ; x=@\color{orange}mult@(y,2.1);
\end{minted}
\caption{}
\label{code:code3}
\end{listing}

We can characterize overloading in terms of the types of the overloaded functions. Suppose that an identifier $F$ denotes both a function $f_1$ of type $S_1 \ra T_1$ and a function $f_2$ of type $S_2 \ra T_2$. (Recall that this covers functions with multiple arguments, since $S_1$ or $S_2$ could be a Cartesian product.) The overloading may be either \textit{context-independent} or \textit{context-dependent}.

In brief, binding is more complicated. Binding is made not only according to name but \textcolor{red}{according to name and type}.

\begin{formula}{}

\noindent Function type: \\

\texttt{
  \tikz [remember picture,overlay] \matrix [anchor=west,rounded corners=2mm,thick,rectangle,draw=green!50!black,ampersand replacement=\&] (condep) 
  {
    \node [rectangle,draw=orange!80!black,thick] (conind) { 
      name : {\em parameters\/}
      }; \&
    \node {
      $\rightarrow$ \em result\/
    };\\};} \\

\noindent \textcolor{blue}{Context dependent overloading}:
  \begin{tikzpicture} [overlay,remember picture] 
    \node (here) {}; \draw [<-,thick,black,green!50!black] (condep.south east) |- (here) ;
  \end{tikzpicture}\\
\quad Overloading based on function name, parameter type and return type.

\noindent \textcolor{blue}{Context independent overloading} : 
  \begin{tikzpicture} [overlay,remember picture] 
    \node (here) {}; \draw [->,thick,black,orange!80!black] (here) -| +(2.8,2.7) -| (conind) ;
  \end{tikzpicture}\\
\quad Overloading based on function name and parameter type. No return type!

\end{formula}

\vspace*{\fill}
\columnbreak

\subsection{Context-dependent Overloading}
\label{subsec:depend-overload}

\textit{\textbf{Context-dependent overloading}} requires only that $S_1$ and $S_2$ are non-equivalent or that $T_1$ and $T_2$ are non-equivalent. If $S_1$ and $S_2$ are non-equivalent, the function to be called can be identified as below (\hyperref[subsec:independ-overload]{context-independent overloading}). If $S_1$ and $S_2$ are equivalent but $T_1$ and $T_2$ are non-equivalent, context must be taken into account to identify the function to be called. Consider the function call ``$F(E)$'', where $E$ is of type $S_1$ (equivalent to $S_2$). If the function call occurs in a context where an expression of type $T_1$ is expected, then $F$ must denote $f_1$; if the function call occurs in a context where an expression of type $T_2$ is expected, then $F$ must denote $f_2$.

For example:
Which type does the expression calling the function expects (context) ?
\begin{listing}[H]

\begin{minted}{cpp}
int f(double a) { ...@\color{orange}\circtxt{1}@ }
int f(int a) { ...@\color{green!50!black}\circtxt{2}@ } 
double f(int a) { ...@\color{violet}\circtxt{3}@ }  
double x,y; 
int a,b;
\end{minted}
\caption{}
\label{code:code4}

\end{listing}

\begin{itemize}[leftmargin=*]
  \item[] \texttt{a=f(x)}; {\color{orange}\circtxt{1}} (x double)
  \item[] \texttt{a=f(x)}; {\color{orange}\circtxt{1}} (x double)
  \item[] \texttt{a=f(a)}; {\color{green!50!black}\circtxt{2}} (a int, assign int)
  \item[] \texttt{x=f(a)}; {\color{violet}\circtxt{3}} (a int, assign double)
  \item[] \texttt{x=2.4+f(a)}; {\color{violet}\circtxt{3}} (a int, mult double)
  \item[] \texttt{a=f(f(x))}; {\color{green!50!black}\circtxt{2}}({\color{orange}\circtxt{1}}) ( x double, f(x):int, assign int)
  \item[] \texttt{a=f(f(a))}; {\color{green!50!black}\circtxt{2}}({\color{green!50!black}\circtxt{2}}) or {\color{orange}\circtxt{1}}({\color{violet}\circtxt{3}}) ???
\end{itemize}
Problem gets more complicated. (even forget about coercion)

Context dependent overloading is more expensive, complex and confusing. Most overloading languages are context independent.

\subsection{Context-independent Overloading}
\label{subsec:independ-overload}

\textit{\textbf{Context-independent overloading}} requires that $S_1$ and $S_2$ are non-equivalent. Consider the function call ``$F(E)$''. If the actual parameter $E$ is of type $S_1$, then $F$ here denotes $f_1$ and the result is of type $T_1$. If $E$ is of type $S_2$, then $F$ here denotes $f_2$ and the result is of type $T_2$. With context-independent overloading, the function to be called is always uniquely identified by the type of the actual parameter.

\begin{formula}{\textcolor{brown}{Careful}}
Overloading is useful only for functions doing same operations. Two functions with different purposes should not be given same names. Confuses programmer and causes errors.
\end{formula}

\vspace*{\fill}
\columnbreak

We must be careful not to confuse the distinct concepts of overloading (which is sometimes called \textit{ad hoc polymorphism}) and \textit{parametric polymorphism}. Overloading means that a small number of separately-defined procedures happen to have the same identifier; these procedures do not necessarily have related types, nor do they necessarily perform similar operations on their arguments. Polymorphism is a property of a single procedure that accepts arguments of a large family of related types; the parametric procedure is defined once and operates uniformly on its arguments, whatever their type.
