\section{Parameters (\textit{from slides})}

\textbf{Declaration part:}

\texttt{abstraction\_name $(FP_1, FP_2, \ldots, FP_n)$}

\noindent \textbf{Use part:}

\texttt{abstraction\_name $(AP_1, AP_2, \ldots, AP_n)$}

\noindent \textit{\textbf{Formal parameters:}} Identifiers or constructors of identifiers (patterns in functional languages).

\noindent \textit{\textbf{Actual parameters:}} Expression or identifier based on the type of the abstraction and parameter.

\subsection{Parameter passing mechanisms}

Programming language may support one or more mechanisms. 3 basic methods:
\begin{enumerate}
 \item Copy mechanisms (assignment based)
 \item Binding mechanisms 
 \item Pass by name (substitution based)
\end{enumerate}

\subsubsection{Copy Mechanisms}

Function and procedure abstractions, assignment between actual and formal parameter:
\begin{enumerate}
 \item \textbf{Copy In:} \\
	On function call: $FP_i \leftarrow AP_i$
 \item \textbf{Copy Out:} \\
	On function return: $AP_i \leftarrow FP_i$
 \item \textbf{Copy In-Out:} \\
	On function call: $FP_i \leftarrow AP_i$, and\\
	On function return: $AP_i \leftarrow FP_i$
\end{enumerate}
\texttt{C} only allows copy-in mechanism. This mechanism is also called as \textbf{Pass by value}.

\end{multicols}

\begin{minipage}{.29\linewidth}

\begin{listing}[H]

\begin{minted}{c}
int x=1, y=2;
void f(int a, int b) {
  x += a+b;
  a++;
  b=a/2;
}

int main() {
  f(x,y);
  printf("x:%d, y:%d\n",x,y);
  return 0;
}
\end{minted}
\caption{}
\label{code:code6}
\end{listing}

\end{minipage}%
\begin{minipage}{.05\linewidth}
  \vspace*{\fill}
  \hspace*{\fill}
\end{minipage}%
\begin{minipage}{.29\linewidth}

  \underline{\textbf{Copy In:}}

  \begin{tikzpicture}
  \matrix [matrix of nodes,nodes in empty cells,ampersand replacement=\&,column sep=4mm] (m) {
      \underline{x} \&\underline{y} \&\underline{a} \&\underline{b} \\
          {$\not 1$} \& {$2$} \& {$\not 1$} \& {$\not 2$} \\ 
  $4$ \&  \&  $2$ \& $1$ \\
  };
  \draw [bend right=25,->,thick,blue!70!white] (m-2-1) to (m-2-3); 
  \draw [bend left=25,->,thick,blue!70!white] (m-2-2) to (m-2-4); 
  \end{tikzpicture}

  \texttt{x:4, y:2}

  \underline{\textbf{Copy Out:}}

  \begin{tikzpicture}
  \matrix [matrix of nodes,nodes in empty cells,ampersand replacement=\&,column sep=4mm] (m) {
  \underline{x} \&\underline{y} \&\underline{a} \&\underline{b} \\
  $\not 1$ \& $\not 2$ \& $\not 0$ \& $\not 0$ \\
  $\not 1$      \&  {$0$}   \& {$1$}  \& {$0$} \\
  {$1$} \& \& \& \\ 
  };
  \draw [bend left=25,->,thick,blue!70!white] (m-3-3) to (m-4-1); 
  \draw [bend right=25,->,thick,blue!70!white] (m-3-4) to (m-3-2); 
  \end{tikzpicture}

  \texttt{x:1, y:0}

\end{minipage}%
\begin{minipage}{.05\linewidth}
  \vspace*{\fill}
  \hspace*{\fill}
\end{minipage}%
\begin{minipage}{.29\linewidth}

  \underline{\textbf{Copy In-Out:}}

  \begin{tikzpicture}
  \matrix [matrix of nodes,nodes in empty cells,ampersand replacement=\&,column sep=4mm] (m) {
  \underline{x} \&\underline{y} \&\underline{a} \&\underline{b} \\
  {$\not 1$} \& {$\not 2$} \& {$\not 1$} \& {$\not 2$} \\ 
  $\not 4$ \&  {$1$}\&  {$2$} \& {$1$} \\
  {$2$} \\
  };
  \draw [bend right=15,->,thick,blue!70!white] (m-2-1) to (m-2-3); 
  \draw [bend left=15,->,thick,blue!70!white] (m-2-2) to (m-2-4); 
  \draw [bend left=15,->,thick,blue!70!white] (m-3-3) to (m-4-1); 
  \draw [bend right=15,->,thick,blue!70!white] (m-3-4) to (m-3-2); 
  \end{tikzpicture}

  \texttt{x:2, y:1}

\end{minipage}

\subsubsection{Binding Mechanisms}

Based on binding of the formal parameter variable/identifier to actual parameter value/identifier. Only one entity (value, variable, type) exists with more than one names.
\begin{enumerate}
  \item \textbf{Constant binding:} Formal parameter is constant during the function. The value is bound to actual parameter expression value. \\
  Functional languages including Haskell uses this mechanism.
  \item \textbf{Variable binding:} Formal parameter variable is bound to the actual parameter variable. Same memory area is shared by two variable references.\\  
  Also known as \textbf{pass by reference}.
\end{enumerate}
The other type and entities (function, type, etc) are passed with similar mechanisms. 

\begin{minipage}{.4\linewidth}

\begin{listing}[H]
\begin{minted}{c}
int x = 1, y = 2;
void f(int a, int b) {
  x += a + b;
  a++;
  b = a / 2;
}

int main() {
  f(x, y);
  printf("x:%d , y:%d\n",x,y);
  return 0;
}
\end{minted}
\caption{}
\label{code:code7}
\end{listing}

\end{minipage}%
\begin{minipage}{.1\linewidth}
  \vspace*{\fill}
  \hspace*{\fill}
\end{minipage}%
\begin{minipage}{.4\linewidth}
  \underline{\textbf{Variable binding:}}\\
  \begin{tabular}{cp{.5em}c}
    f():a & & f():b \\
    x & & y \\ \cline{1-1} \cline{3-3}
    $\not 1$ 		& &  	$\not 2$	\\
    $\not 4$ 		& &  	$2$	\\
    $5$ 			& &  		\\
  \end{tabular}\\
  \texttt{x: 5, y:2}
\end{minipage}

\subsubsection{Pass by Name}

Actual parameter syntax replaces each occurence of the formal parameter in the function body, then the function body evaluated. \texttt{C} macros works with a similar mechanism (by pre-processor).

\noindent Mostly useful in theoretical analysis of PL's. Also known as \textbf{Normal order evaluation}.

Example (Haskell-like)
\begin{listing}[ht]

\begin{minted}{haskell}
f x y = if (x < 12) then x*x + y*y + x 
        else x + x*x
\end{minted}

\caption{}
\label{code:code8}
\end{listing}

\noindent \textbf{Evaluation:}
\begin{align*}
  &\texttt{f } (3*12 + 7)\ (24 + 16*3) \\
  &\mapsto \texttt{if } ((3*12 + 7) < 12) \texttt{ then } (3*12 + 7) * (3*12 + 7) + (24 + 16*3) * (24 + 16*3) + (3*12 + 7)\\
  &\quad\ \ \ \ \texttt{else } (3*12 + 7) + (3*12 + 7) * (3*12 + 7)\\
  &\stackrel{*}{\mapsto} \texttt{if } (43 < 12) \texttt{ then } ...\\
  &\mapsto \texttt{if } (false) \texttt{ then } ...\\
  &\mapsto (3*12 + 7) + (3*12 + 7) * (3*12 + 7)\\
  &\stackrel{*}{\mapsto} (3*12 + 7) + 43 * (3*12 + 7)\\
  &\mapsto ... \mapsto 1892\quad \texttt{\textcolor{blue}{12 steps}}
\end{align*}

\noindent \textbf{Normal order evaluation} is mathematically natural order of evaluation.

\newpage

Most of the PL's apply \textit{\textbf{eager evaluation}}: Actual parameters are evaluated first, then passed.
\begin{align*}
  &\texttt{f } (3*12+7)\ (24+16*3) \\
  &\mapsto \texttt{f } (36+7)\ (24+16*3) \\
  &\stackrel{*}{\mapsto} \texttt{f } 43\ 72 \\
  &\mapsto \texttt{if } (43 < 12) \texttt{ then } 43*43 + 72*72 + 43 \\
  &\quad\ \ \texttt{else } 43+43*43\\
  &\mapsto \texttt{if } (false)\ \texttt{then } ...\\
  &\mapsto 43 + 43*43 \stackrel{*}{\mapsto} 1892 \quad \texttt{\textcolor{blue}{8 steps}}
\end{align*}

\noindent \textbf{Church–Rosser Property:} 
\begin{quote}
  If an expression can be evaluated at all, it can be evaluated by consistently using normal-order evaluation. If an expression can be evaluated in several different orders (mixing eager and normal-order evaluation), then all of these evaluation orders yield the same result.
\end{quote}

\noindent In $\lambda$-calculus, all orders reduce the same normal form.
\begin{center}
  
\xygraph{[]!{<20mm,0mm>:<0mm,12mm>::}
{\lambda x.(\lambda y.y+(\lambda x.x+1~y)~~(x+2))~~5}
 ( :@{|->}[dll] {\lambda y.y+(\lambda x.x+1~~y)~~(5+2)}
    ( :@{|->}[dl] {5+2+(\lambda x.x+1~~(5+2))}
        ( :@{|->}[ddddrrr] {5+2+5+2+1}="bot" ) ,
      :@{|->}@(dr,ur)[dddl] {\lambda y.y+y+1~~(5+2)} :@{|->}"bot"
    ),
   :@{|->}[dd]  {\lambda x.x+2+(\lambda x.x+1~~(x+2))~5}
    (
    :@{|->}[dl]  {5+2+(\lambda x.x+1~~(5+2))} :@{|->}"bot",
    :@{|->}[dr]  {\lambda x.x+2+x+2+1~~5} :@{|->}"bot"
    ),
   :@{|->}[drr] {\lambda x.(\lambda y.y+y+1~~(x+2))~~5} (
       :@{|->}[ddr] {\lambda y.y+y+1~~(5+2)} :@{|->}@(dr,r)"bot",
       :@{|->}[ddd] {\lambda x.x+2+x+2+1~~5} :@{|->}"bot")
 )
}

\end{center}

\noindent Haskell implements \textbf{Lazy Evaluation} order.
\begin{itemize}
  \item Eager evaluation is faster than normal order evaluation but violates Church-Rosser Property.
  \item Lazy evaluation is as fast as eager evaluation but computes same results with normal order evaluation (unless there is a side effect).
\end{itemize}
Lazy evaluation expands the expression as normal order evaluation however once it evaluates the formal parameter value other evaluations use previously found value:
\begin{align*}
  &\texttt{f } (3*12 + 7)\ (24 + 16*3)\\
  &\mapsto \texttt{if } (x:(3*12 + 7) < 12) \texttt{ then } x:(3*12 + 7) * x:(3*12 + 7) + y:(24 + 16*3) * y:(24 + 16*3) + x:(3*12+7)\\
  &\quad\ \ \texttt{else } x:(3*12 + 7) + x:(3*12 + 7) * x:(3*12 + 7)\\
  &\stackrel{*}{\mapsto} \texttt{if } (x:43 < 12) \texttt{ then } x:43 * x:43 + y:(24 + 16*3) * y:(24 + 16*3) + x:43\\
  &\quad\ \ \ \texttt{else } x:43 + x:43 * x:43\\
  &\mapsto \texttt{if } (false) \texttt{ then } ...\\
  &\mapsto x:43 + x:43 * x:43\\
  &\mapsto x:43 + 1849 \mapsto 1892 \quad \texttt{\textcolor{blue}{7 steps}}
\end{align*}
In lazy evaluation, parameters are passed by name but compiler keeps evaluation state of them. Parameter value is store once it is evaluated. Further evaluations use that.v

Python implementation. First delay evaluation of expressions. Convert to functions:

\lstinline!exp! $\rightarrow$ \lstinline[language=Python]!lambda : exp! \\
$\eta$ expansion. Function version is also called \textbf{thunk}.

Inside function, call these functions to evaluate the expression
\begin{listing}[ht]

\begin{minted}{py}
def E(thunk):
  if not hasattr(thunk,"stored"):
    thunk.stored = thunk()      # evaluate and store
  return thunk.stored           # use stored value
def f(x,y):
  if E(x) < 10:                # call E() on all evaluations
    return E(x)*E(x)+E(y)
  else:
    return E(x)*E(x)+E(x)
f(lambda : 3*32+4, lambda: 4/0)     # call by converting to function
\end{minted}

\caption{}
\label{code:code9}
\end{listing}v

Delayed evaluation in normal order or lazy evaluation enables working on infinite values.
