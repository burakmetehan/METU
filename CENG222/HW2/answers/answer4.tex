\section*{Answer 4}

% Central Limit Theorem applies to all these distributions with sufficiently large $n$ in the case of Binomial, k for Negative Binomial, and α for Gamma variables.


% According to Central Limit Theorem, as long as $n$ is large, one can use Normal distribution to compute probabilities about $S_n$. So, in here $n = 30.000$, then we can use 

%%%%%%%%%%%%%%%%%%%%%%%%%%%%%%%%%%%%%%%%%%


%variables are Binomial; in other words, they are Binomial distribution. For small 

% Also, the Central Limit Theorem is applicable to Binomial variables with sufficiently large $n$. So, we can use the Binomial($n, p$). since p is not small or large enough. Instead, $0.05 \leq p \leq 0.95$; therefore, we can 

%%%%%%%%%%%%%%%%%%%%%%%%%%%%%%%%%%%%%%%%%%

% In here, let 
% \begin{itemize}
%   \item $X_1 = 70$ people from undergraduate
%   \item $X_2 = 71$ people from undergraduate
%   \item[] $\cdot$
%   \item[] $\cdot$
%   \item[] $\cdot$
%   \item $X_{31} = 100$ people from undergraduate   
% \end{itemize}
% We need to find the sum of the probability of each one. So, the Central Limit Theorem is applicable to Binomial variables with sufficiently large $n$. 
% So, $S_n = X_1 + \cdots + X_{31}$. For small $p$, we can use Poisson, but out $p$is not small. In such situation, $0.05 \leq p \leq 0.95$, we can use the following
% \begin{equation*}
%   \textnormal{Binomial}(n, p) \approx Normal \left( \mu = np, \sigma = \sqrt{np(1-p)} \right)
% \end{equation*}

%%%%%%%%%%%%%%%%%%%%%%%%%%%%%%%%%%%%%%%5

According to Central Limit Theorem, if $n$ is sufficiently large and $p$ satisfies the condition $0.05 \leq p \leq 0.95$, then all distribution can be thought as Normal distribution. Since all variables are Binomial, we can use the following
\begin{equation*}
  \textnormal{Binomial}(n, p) \approx \textnormal{Normal}\left( \mu = np, \sigma = \sqrt{np(1-p)} \right)
\end{equation*}

\subsection*{a)}

We need to determine $n$, $p$ and, by using these, we need to find $\mu$, $\sigma$

\begin{multicols}{2}
  \begin{itemize}
    \item $n = 100$
    \item $p = 0.74$
    \item $\mu = np = 74$
    \item $\sigma = \sqrt{np(1-p)} \approx 4.39$
  \end{itemize}
\end{multicols}

And we are asked the following, ($X$ = the number of undergraduate students in the group.)

\begin{align*}
  \prob{X \geq 70} &= \prob{X \geq 69.5} &(\textit{Continuity correction})\\
  &= 1 - \prob{X \leq 69.5}\\
  &= 1 - \prob{\dfrac{X - \mu}{\sigma} \leq \dfrac{69.5 - \mu}{\sigma}}\\
  &= 1 - \prob{\dfrac{X - 74}{4.38634244} \leq \dfrac{69.5 - 74}{4.38634244}}\\
  &= 1 - \prob{Z \leq \dfrac{69.5 - 74}{4.38634244}}\\
  &= 1 - \prob{Z \leq - 1.025911693}\\
  &= 1 - \Phi(- 1.025911693)\\
  &= 1 - 0.1525 = 0.8475 \approx 0.85
\end{align*}

\newpage
\subsection*{b)}

We need to determine $n$, $p$ and, by using these, we need to find $\mu$, $\sigma$

\begin{multicols}{2}
  \begin{itemize}
    \item $n = 100$
    \item $p = 0.10$
    \item $\mu = np = 10$
    \item $\sigma = \sqrt{np(1-p)} = 3$
  \end{itemize}
\end{multicols}

And we are asked the following, ($X$ = the number of people pursuing a doctoral degree in the group.)

\begin{align*}
  \prob{X \leq 5} &= \prob{X \leq 5.5} &(\textit{Continuity correction})\\  
  &= \prob{\dfrac{X - \mu}{\sigma} \leq \dfrac{5.5 - \mu}{\sigma}}\\
  &= \prob{\dfrac{X - 10}{3} \leq \dfrac{5.5 - 10}{3}}\\
  &= \prob{Z \leq - 1.50}\\
  &= \Phi(- 1.50)\\
  &= 0.066807 \approx 0.07
\end{align*}



% \begin{align*}
%   \prob{0 \leq X \leq 5} &= \prob{\dfrac{0 - \mu}{\sigma} \leq \dfrac{X - \mu}{\sigma} \leq \dfrac{5 - \mu}{\sigma}}\\
%   &= \prob{\dfrac{0 - 10}{3} \leq \dfrac{X - 10}{3} \leq \dfrac{5 - 10}{3}}\\
%   &= \prob{- 3.33 \leq Z \leq - 1.67}\\
%   &= \Phi(- 1.67) - \Phi(- 3.33)\\
%   &= 0.0475 - 0.0004 = 0.0471
% \end{align*}
