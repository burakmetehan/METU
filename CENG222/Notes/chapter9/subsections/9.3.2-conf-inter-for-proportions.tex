\subsection{Confidence Intervals for Proportions}
\label{subsec:conf-inter-for-proportions}

In particular, we surely don't know the variance when we estimate a population proportion.
\begin{definition}{}
  We assume a subpopulation $A$ of items that have a certain \textit{attribute}. By the \textbf{population proportion} we mean the probability
  \begin{equation*}
    p = \prob{i \in A}
  \end{equation*}
  for a randomly selected item i to have this attribute.

  A \textbf{sample proportion}
  \begin{equation*}
    \hat{p} = \frac{\textnormal{number of sampled items from A}}{n}
  \end{equation*}
  is used to estimate $p$.
\end{definition}
\noindent Let us use the \textit{indicator} variables
\begin{equation*}
  X_i = \begin{cases}
    1 &\textnormal{if } i \in A\\
    0 &\textnormal{if } i \notin A\\
  \end{cases}
\end{equation*}
Each $X_i$ has Bernoulli distribution with parameter $p$. In particular
\begin{align*}
  \expc{X_i} = p &&\textnormal{and}& &\var{X_i} = p (1 - p)
\end{align*}
Also,
\begin{equation*}
  \hat{p} = \frac{1}{n} \sum_{i=1}^{n} X_i
\end{equation*}
is nothing but a sample of $X_i$.

\noindent Therefore,
\begin{align*}
  \expc{\hat{p}} = p &&\textnormal{and}& &\var{\hat{p}} = \frac{p (1 - p)}{n}
\end{align*}

\vspace*{\fill}
\columnbreak

\noindent We conclude that
\begin{enumerate}
  \item a sample proportion $\hat{p}$ is unbiased for the population proportion $p$;
  \item it has approximately Normal distribution for large samples, because it has a form of
  a sample mean;
  \item when we construct a confidence interval for $p$, we do not know the standard deviation
  Std($\hat{p}$).
\end{enumerate}

Indeed, knowing the standard deviation is equivalent to knowing p, and if we know p, why
would we need a confidence interval for it?

Thus, we estimate the unknown standard error
\begin{equation*}
  \sigma(\hat{p}) = \sqrt{\frac{p (1-p)}{n}}
\end{equation*}
by
\begin{equation*}
  s(\hat{p}) = \sqrt{\frac{\hat{p} (1 - \hat{p})}{n}}
\end{equation*}
and use it in the general formula
\begin{equation*}
  \hat{p}\ \pm\ z_{\alpha/2} \cdot s(\hat{p})
\end{equation*}
to construct an approximate $(1 - \alpha)100\%$ confidence interval.
\begin{formula}{Confidence interval for a population proportion}
  \begin{equation*}
    \hat{p} \ \pm\ z_{\alpha/2} \sqrt{\frac{\hat{p} (1 - \hat{p})}{n}}
  \end{equation*}
\end{formula}

Similarly, we can construct a confidence interval for the difference between two proportions. In two populations, we have proportions $p_1$ and $p_2$ of items with an attribute. Independent samples of size $n_1$ and $n_2$ are collected, and both parameters are estimated by sample proportions $\hat{p_1}$ and $\hat{p_2}$.

Summarizing, we have
\begin{align*}
  \textnormal{Parameter of interest:} &\theta = p_1 - p_2 \\
  \textnormal{Estimated by:}          &\hat{\theta} = \hat{p_1} - \hat{p_2} \\
  \textnormal{Its standard error:}    &\sigma(\hat{\theta}) = \sqrt{\frac{p_1 (1 - p_1)}{n_1} + \frac{p_2 (1- p_2)}{n_2}} \\
  \textnormal{Estimated by:}          &s(\hat{\theta}) = \sqrt{\frac{\hat{p_1} (1 - \hat{p_1})}{n_1} + \frac{\hat{p_2} (1- \hat{p_2})}{n_2}}
\end{align*}

\begin{formula}{Confidence interval for the difference of proportions}
  \begin{equation*}
    \hat{p_1} - \hat{p_2} \ \pm \ z_{\alpha/2} \sqrt{\frac{\hat{p_1} (1 - \hat{p_1})}{n_1} + \frac{\hat{p_2} (1 - \hat{p_2})}{n_2}}
  \end{equation*}
\end{formula}

\begin{example}{ (Pre-election poll)}
  A candidate prepares for the local elections. During his campaign, 42 out of 70 randomly selected people in town A and 59 out of 100 randomly selected people in town B showed they would vote for this candidate. Estimate the difference in support that this candidate is getting in towns A and B with $95\%$ confidence. Can we state affirmatively that the candidate gets a stronger support in town A? \\

  \textbf{Solution:}
  We have $n_1 = 70$, $n_2 = 100$, $\hat{p_1} = 42/70 = 0.6$, and $\hat{p_2} = 59/100 = 0.59$. For the confidence interval, we have
  \begin{equation*}
    \textnormal{center} = \hat{p_1} - \hat{p_2} = 0.01
  \end{equation*}
  and
  \begin{align*}
    \textnormal{margin} &= z_{0.05/2} \sqrt{\frac{\hat{p_1} (1 - \hat{p_1})}{n_1} + \frac{\hat{p_2} (1 - \hat{p_2})}{n_2}}\\
    &= (1.960) \sqrt{\frac{(0.6) (0.4)}{70} + \frac{(0.59) (0.41)}{100}} = 0.15
  \end{align*}
  Then
  \begin{equation*}
    0.01 \ \pm \ 0.15 = \left[ -0.14,\ 0.16 \right]
  \end{equation*}
  is a $95\%$ confidence interval for the difference in support ($p_1 - p_2$) in the two towns.\\

  So, is the support stronger in town A? On one hand, the estimator $\hat{p_1} - \hat{p_2} = 0.01$ suggests that the support is $1\%$ higher in town A than in town B. On the other hand, the difference could appear positive just because of a sampling error. As we see, the $95\%$ confidence interval includes a large range of negative values too. Therefore, the obtained data \textit{does not} indicate affirmatively that the support in town A is stronger.
\end{example}
