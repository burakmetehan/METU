\subsection{Packages}
\label{subsec:packages}

A \textit{\textbf{package}} is a group of several components declared for a common purpose. In other words, a group of declarations put into a single body.

These components may be \textit{types}, \textit{constants}, \textit{variables}, \textit{procedures}, or indeed \textit{any entities} that may be declared in the programming language.

\vspace*{\fill}
\columnbreak

\texttt{C} has indirect way of packaging per source file. \texttt{Python} defines modules per source file. \texttt{C++} has namespaces.
\begin{listing}[H]
\begin{minted}[]{cpp}
namespace Trig {
  const double pi=3.14159265358979;
  double sin(double x) { ... }
  double cos(double x) { ... }
  double tan(double x) { ... }
  double atan(double x) { ... }
  ...
};
\end{minted}
\caption{}
\label{code:code1}
\end{listing}
\noindent These functions can be used as 
  \mint{cpp}{Trig::sin(Trig::pi/2+x) + Trig::cos(x)}
\noindent In \texttt{C++}, (::) is \textit{Scope} operator. In this way, \textit{identifier overlap is avoided}. There is no name collisions in \mintinline{cpp}{List::insert(...)} and \mintinline{cpp}{Tree::insert(...)}.
